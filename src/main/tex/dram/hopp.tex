% Dramenanalyse: Hoppla, wir leben!

\HeadingOne{Zwischen Revolutionslogik und Irrsinn:\\
  \Cite{Hoppla, wir leben!}}{\Cite{Hoppla, wir leben!}}

Das erste dramatische Werk Ernst Tollers nach seiner Haftentlassung entstand
1927 -- in einer Zeit, die man literarhistorisch oft mit dem Begriff der
\Cite{Neuen Sachlichkeit} zu beschreiben versucht.\Footnote{\Abr{Vgl} dazu
  auch den Aufsatz von Stefan Neuhaus: Ernst Toller und die Neue
  Sachlichkeit. Versuch über die Anwendbarkeit eines problematischen
  Epochenbegriffs. In: \BibRef{NEUHAU.99}{135-154}.}  
Der bis dahin durch seine \Quote{spätexpressionistischen} Stücke bekannt
gewordene Autor und Revolutionär konnte sich diesem literarischen
Paradigmenwechsel ebensowenig entziehen, wie dessen zeitgeschichtlichen
Hintergründen. So sind in \Title{Hoppla, wir leben!} -- durch besondere
Entstehungsumstände noch verstärkt -- gewisse \Quote{Restbestände} des
expressionistischen Veränderungspathos mit \Quote{sachlich} grundierter
Gegenwartsanalyse zusammengeflossen.

\HeadingTwo{Bemerkungen zur Textedition}

Die Analyse des Dramas erhält eine gewisse Komplexität durch die Existenz
zweier Fassungen, von denen sich eine aus der Kooperation Tollers mit
dem Regisseur Erwin Piscator ergab. Toller hatte Piscator, der nach einem
geeigneten Stück für die Eröffnung seines neuen Theaters in Berlin
suchte, seine bereits vollständig ausgearbeitete Erstfassung
vorgelegt. Diese ursprungliche Textform, die das originäre Werk Tollers
darstellt, ist nie publiziert worden und steht der Forschung heute nur noch in
Form zweier Einzelexemplare zur Verfügung.\Footnote{Laut
  \BibRef{leydec.98}{123}, sind dies eine im Privatbesitz John Spaleks
  befindliche Kopie mit handschriftlichen Ergänzungen Tollers und eine (lange
  unbekannt gebliebene) Kopie in der Theatersammlung der Staats- und
  Universitätsbibliothek Hamburg (Thalia-Archiv, \Abr{Nr} 5232). Letztere war
  Grundlage der Inszenierung an den Hamburger Kammerspielen.}

Piscator war an dem Stück interessiert, sah jedoch erheblichen
Änderungsbedarf, da Toller \Cite{dem Gefühl} eine zu \Cite{selbstherrliche}
Stellung eingeräumt hätte.\Footnote{\Abr{Vgl} \BibRef{piscat.29}{146-147}. Im 
  Rückblick beschreibt Piscator die Kooperation mit Toller
  geradezu als Reparaturarbeit an einem nur bedingt ausgereiften Rohtext:
  \Cite{Das ursprünglich gesteckte Ziel der Revue wurde am nächsten erreicht
    durch einen Entwurf, den Toller mir im Frühjahr gegeben hatte. [..] Aber wie
    immer bei Toller ging das Dokumentarische mit dem Dichterisch-Lyrischen
    durcheinander. Alle unsere Bemühungen im Lauf der weiteren Arbeit gingen
    dahin, dem Stück den realistischen Unterbau zu geben.}}  
Die bei den Vorarbeiten für die Berliner Inszenierung gemachten Änderungen
brachten eine bedeutende Verschiebung der politischen Aussage des Dramas mit
sich. Langwierige Debatten führten Toller und Piscator insbesondere über die
Gestaltung des Dramenschlusses.\Footnote{Die Werkausgabe von Spalek und
  Frühwald verzeichnet beide Varianten des Schlusses, lässt aber andere wichtige
  Änderungen unerwähnt.}

Die fast gleichzeitige Uraufführung an zwei deutschen Theatern brachte die
Varianz auch auf die Bühne: Während die Piscatorbühne in Berlin die
überarbeitete Version (sogar mit noch zusätzlichen Änderungen Piscators)
spielte, wurde in Hamburg nahezu Tollers Erstfassung
inszeniert.\Footnote{Die Hamburger Kopie der Erstfassung weist
  Ergänzungen auf, die wahrscheinlich von Toller und/oder dem dortigen Regisseur
  Hans Lotz stammen. \Abr{Vgl} \BibRef{leydec.98}{123 und 131}.}  
Der gemeinsam mit Piscator erarbeitete Text wurde von Toller noch 1927
als Druckfassung autorisiert und dadurch bald kanonisiert. Die später von
beiden kritisch kommentierte, konfliktreiche Berliner
Zusammenarbeit prägte dem Stück jedoch eine nachhaltige Ambivalenz
auf.\Footnote{Toller schrieb
  \AbrPair{z}{B}. 1930 in \Quote{Arbeiten}: \Cite{Ich bedaure
  heute, daß ich, von einer Zeitmode befangen, die Architektonik des
  ursprünglichen Werkes zugunsten der Architektonik der Regie zerbrach.}
  Zitiert nach \LongSourceRef{I}{146}.}

Für die Interpretation des Dramas blieben dadurch wichtige Fragen offen. Die
von Piscator erwirkten Änderungen betreffen neben dem Schluss insbesondere die
Kontrastierung des Protagonisten mit den Figuren, die sozusagen für
alternative politische Konzepte stehen. In der Erstfassung wird
der gefühlsmäßigen Sichtweise des Protagonisten größeres Gewicht gegeben
und der dramatische Konflikt motiviert sich stärker aus dieser subjektiven
Wahrnehmung. Die Werkausgabe von Spalek und
Frühwald weist die entsprechenden Passagen der Erstfassung leider nicht nach.

Basierend auf der ungenügenden Textinformation entstand ein
literaturwissenschaftlicher Streit, der im Kern darum geht, ob und wie sich
Tollers Distanzierung von expressionistischen Revolutionskonzepten
konstatieren lässt. Auch die für die vorliegende Arbeit wichtigen Fragen nach
einer Pluralisierung der Dramenperspektive sowie nach der Reflexionsweise und
konzeptionellen Bewältigung der in diesem \Cite{Zeitstück} thematisierten
\Quote{verspäteten} Moderneerfahrung sind von der genannten Textvarianz
entscheidend berührt.

Erst 1998 konnte Karl Leydecker durch Recherche an den erhaltenen Exemplaren
der Erstfassung zeigen, dass jene Forschungspositionen, die im wesentlichen
auf der Werkausgabe und der Einbeziehung späterer Selbstkommentare Tollers
beruhen, nicht haltbar sind. In seiner eingehenden Variantenanalyse vertritt
er die Einschätzung,

\begin{BlockQuote}
  that \Cite{Hoppla, wir leben!} is not Toller's first clear-headed
  post-Expressionist play but, with its recrudescence of Expressionist
  motifs and structure, is instead eloquent testimony to the extent to
  which, as late as 1927, Toller's literary imagination was in thrall to
  that movement.\Footnote{\BibRef{leydec.98}{123}.} 
\end{BlockQuote}
Dass Toller tatsächlich \Cite{sklavisch} an den Revolutionsideen des
Expressionismus geklebt haben soll, wird in einem neueren und umfangreichen
Beitrag von Kirsten Reimers allerdings
bestritten.\Footnote{Nach ihrer Meinung \Cite{geht Karl
  Leydecker in seiner Behauptung zu weit, Toller habe sich 1927 vom
  Expressionismus noch nicht genügend distanziert [..]}. Siehe
\BibRef{reimer.00}{207}.} 
Da im Rahmen der vorliegenden Arbeit
leider keine Quellenanalyse an den Exemplaren der Erstfassung möglich war,
beziehe ich mich hauptsächlich auf die Argumentationen bei Reimers und
Leydecker, um zu erörtern, inwiefern sich die Konzepte Tollers und
Piscators in einem dialektischen Prozess verbunden haben und was dies für
Tollers (politische) Modernereflexion bedeutete.

Die Darstellung orientiert sich deswegen hauptsächlich an der kanonisierten
Druckfassung und zitiert Textstellen gemäß der Werkausgabe. An einigen
relevanten Stellen wird auch auf die Varianten der Erstfassung hingewiesen.  
Erst in der abschließenden Erörterung der \Quote{zwiespältigen Autornorm} wird
die eben angerissene Deutungsdebatte noch einmal aufgenommen, wobei
Aspekte der Erstfassung gleichsam als Kontrastmittel zur Beleuchtung des
politischen Gehalts des Stückes herangezogen werden. 

\HeadingTwo{Inhaltliche Zusammenfassung}

Die Revolutionsthematik, die schon in der \Title{Wandlung} und in \Title{Masse
Mensch} von zentraler Bedeutung ist, bildet in \Title{Hoppla, wir leben!} den
perspektivierenden Eingangskontext der Dramenhandlung:

\begin{BlockQuote}
  Das Stück spielt in vielen Ländern. Acht Jahre nach einem niedergeworfenen
  Volksaufstand. Zeit: 1927. \SourceRef{II}{11}\Footnote{Die pathetische Rede
    aus dem Nebentext der \Title{Wandlung} von der
    \Cite{Wiedergeburt} der Menschheit \Cite{in Europa} erscheint
    hier versachlicht als \Cite{Volksaufstand}, der seinen übernationalen
    Kontext mit dem dezenten Hinweis auf die \Cite{vielen Länder} erhält.}
\end{BlockQuote}
In einem \Cite{Vorspiel} verbinden sich zeitliche Rückblende und
Figurenexposition. Eine Gruppe zum Tode verurteilter Revolutionäre erscheint
als Schicksalsgemeinschaft in den Fängen der staatlichen Macht. Nach ihrer
Begnadigung gehen die \Cite{Genossen} jeweils unterschiedliche Lebenswege. Die
Figur des Karl Thomas, der bei der Begnadigung in Wahnsinn verfällt, wird im
Folgenden zum wesentlichen Handlungsträger.

Nach mehreren Jahren im \Cite{Irrenhaus} wird Thomas \Cite{geheilt} in die
Welt der Republik entlassen \SourceRef{III}{26}, in der die Haftgenossen
mittlerweile ihre Verortung gefunden haben: Einer von ihnen ist als Minister
in die Sphären der Macht aufgestiegen, die anderen mühen sich in politischer
Kleinarbeit für Arbeiterinteressen. Aus der Perspektive des unangepassten
\Quote{Zu-spät-Kommenden} empfindet Thomas ihren politischen Pragmatismus als
feige Halbherzigkeit und beschließt, \Cite{ein Beispiel [zu] geben}, damit
\Cite{die Lahmen rennen} \SourceRef{III}{66}.  Da die ehemaligen Gefährten ihm
-- auf jeweils verschiedene Weise -- entfremdet sind, will er als Einzeltäter
ein Attentat auf den in seinen Augen verräterischen Minister verüben.

Das Drama bebildert die Eindrücke von der Vielgestalt und den Disparitäten der
gesellschaftlichen und politischen Gegenwart als ein Kaleidoskop
widersprüchlicher Tendenzen und \Quote{Teilräume}, mit denen der Protagonist
in Berührung kommt, ohne sich orientieren zu können. Moderne Katastrophen wie
Währungskrise oder Börsensturz stehen für ihn unvermittelt neben den
überwältigenden Möglichkeiten moderner Technik und den weltverbesserischen
Debatten dekadenter Intellektueller. Macht und Profit von Karrieristen
kontrastieren mit demokratischen Hoffnungen in Arbeiterkreisen. Die
Gemeinschaft von einst scheint verloren und seine frühere Geliebte ist
inzwischen eine emanzipierte Frau, die seinen Schwärmereien eine Absage
erteilt. 

In der extremen Zuspitzung des
politischen Mordvorhabens überkreuzt sich die revolutionäre Perspektive des
Protagonisten dann mit ihrem Gegenteil: Reaktionäre Nationalisten vollziehen
zeitgleich das gleiche Attentat auf den Minister unter umgekehrten Vorzeichen.
Was dem seiner Idee beraubten und zu Unrecht des Mordes angeklagten Thomas
bleibt, ist seine Verwirrung und die Infragestellung der erlebten Welt, in der
ihm Irrsinn und Normalität verschwimmen und die Psychiatrie nur als
systemstabilisierende Eingliederungshilfe erscheint.

Zum Schluss ist die alte Gruppe der Gefangenen noch einmal in Haft, nun in
Einzelzellen. Und diesmal kommt die Entlassungsmeldung zu spät. Die
aussichtslose Vereinzelung wird übermächtig und treibt Thomas, kurz bevor
seine Unschuld bekannt wird, zum tragischen Selbstmord in der Zelle.

\medskip

Die Erstfassung Tollers bleibt stärker auf den Protagonisten und dessen
emotionale Weltsicht fixiert. Der Minister wird
wesentlich negativer gezeichnet und zur nahezu dämonischen Gegenfigur
ausgestaltet. Die eher psychologisch geleitete Darstellung bewirkt eine 
engere Perspektive auf die republikanische Welt und legt weniger Wert auf die
Handlungsoption der \Quote{politischen Kleinarbeit}. Die Figuren Eva Berg und
Albert Kroll sind schwächer und weniger engagiert dargestellt. Der Aspekt der
nationalistischen Gefährdung der Republik fehlt, die Handlung ist weniger auf
das Attentat zugespitzt und der Minister überlebt. Der Schlussakt im Gefängnis
fehlt in der Erstfassung. Karl Thomas ist am Ende in der Irrenanstalt,
begeht keinen Selbstmord und es es bleibt unbekannt, dass er nicht
der Attentäter war. Der \Quote{Irrsinn der Welt} wird hier stärker betont und
zum Schluss demonstriert sogar \Cite{das Volk} für den Protagonisten
\SourceRef{III}{324-325}.

\HeadingTwo{Die Inhaftierten als Schwundform der revolutionären\\
  Menschengemeinschaft}

Die Ausgangssituation des Dramas präsentiert eine Gemeinschaft von
Revolutionären in Haft: \Cite{Zeit 1919} \SourceRef{III}{9}. Der Kontext der
gemeinsamen Zwangslage und des anzunehmenden gemeinsamen politischen Ziels
suggeriert zunächst ihre Einheit in Idee und Schicksal. Der Bezug zur
revolutionären Situation ist durch ein vorhergehendes \Cite{Filmisches
  Vorspiel} schemenhaft angedeutet.\Footnote{Der Dramentext umreißt die
  Einspielung so: \Cite{Geräusche: Sturmglocken / Streiflichter knapp / Szenen
    eines Volksaufstandes / Seine Niederwerfung / Des dramatischen Vorspiels /
    Figuren / auftauchend ab und zu} \SourceRef{III}{11}. Der ausgiebige
  Einsatz von Film- und Toneinblendungen zeigt für Falko Schneider, wie
  Piscator \Cite{auf die Herausforderung durch die Massenmedien produktiv
    antwortete}. Siehe \BibRef{schnei.95}{465}.}

Schon im Gespräch der Figuren zeigt sich die Heterogenität der Gruppe. Karl
Thomas und der klassenbewusste Arbeiter Albert Kroll drängen todesmutig auf
einen Ausbruchsversuch. Der Parteifunktionär Wilhelm Kilman hingegen mahnt,
man solle noch \Cite{überlegen} \SourceRef{III}{15}. Ein farblos bleibender
\Cite{sechster Gefangener} distanziert sich völlig und verlangt nach dem
Beistand eines Geistlichen. Der Streit über die Unzuverlässigkeit Kilmans
schwächt den Zusammenhalt. Kroll beschimpft Thomas als \Cite{Bürgersöhnchen}
\SourceRef{III}{19} und verhöhnt seine Beziehung zur naiven Eva Berg. In der
Erstfassung stellt Kroll darüber hinaus in Frage, was Eva Berg eigentlich
\Cite{in unsern Kampf} getrieben habe, denn sie sei von \Cite{reich[er]}
Herkunft und \Cite{hatte es nicht nötig.}\Footnote{\Abr{Vgl}
\BibRef{leydec.98}{125}.}

Die individuelle Charakterisierung der Figuren und ihrer Motive lässt
erkennen, dass hier kein Gemeinschaftsideal pathetisch verbrüderter Menschen
bebildert wird oder widerstreitende Führungsalternativen einer lenkbaren
\Cite{Masse} in Diskurs treten. Statt dessen wird die generelle
Fragwürdigkeit der Revolutionsteilnahme durch Karl Thomas sogar selbstreflexiv
zum Ausdruck gebracht:

\begin{BlockQuote}
  Ja, warum kämpfen wir? Was wissen wir? Für die Idee, für die Gerechtigkeit --
  sagen wir. [..] Blick dich um, was stürzt sich alles zur Idee, in
  Revolution, in Krieg. [..] Immer sind es wenige, die müssen aus innerstem
  Zwang. \SourceRef{III}{18}
\end{BlockQuote}
Der spätere Protagonist reflektiert bereits in dieser
Eingangsszene eine Pluralität von individuellen Handlungsmotiven, die in den
früheren Dramen Tollers vergeblich zu suchen ist. Damit zeichnet sich für
dieses Drama ein neues Subjektkonzept ab, das nicht mehr in den Hülsen
gemeinschaftlicher Einheit und Ganzheit oder in Form absolut gesetzter Ideen
daherkommt. Das Moment des Kampfes aus innerlich gefühlter Notwendigkeit, das
in den expressionistischen Verkündungsdramen normativ dominierte, ist hier
schon eingangs in seiner Beschränkung auf \Cite{wenige} erkannt. Diese
Reduzierung ist in Thomas' Aussage von einer gewissen Wehmut begleitet und
unschwer ist zu erkennen, dass er sich selbst zu jenem Kern der innerlich
Bewegten zählt.

Als Gegenpol muss in diesem Verständnis der Typus des Verräters gelten, den
statt \Cite{innerstem Zwang} Opportunismus treibt und dem beim Taktieren auch
Lüge und Heuchelei recht sind. Am Ende des Vorspiels wird der bereits als
Volksredner und Feigling\Footnote{\SourceRef{III}{19}: \Cite{\DramName{Wilhelm
      Kilman}: Hab ich nicht zu den Massen gesprochen vom Balkon des Rathauses?
    \DramName{Albert Kroll}: Ja, als wir die Macht hatten. Vorher nicht für, nicht
    wider.}}  
dargestellte Wilhelm Kilman dem Publikum als ein solcher Antipode
enthüllt: Per selektiver Informationsvergabe\Footnote{In Abwesenheit der
  anderen Dramenfiguren wird der Zuschauer Zeuge eines kurzen Zwiegesprächs
  zwischen Kilman und dem Militärbeamten, der das Gnadenurteil übermittelt.}
erfährt man, wie sein geheimes \Cite{Gnadengesuch} bewilligt wird, da er
\Cite{gegen [seinen] Willen in die Reihen der Aufrührer [kam]}
\SourceRef{III}{21}.

Das Vorspiel endet mit der offiziellen Begnadigung der ganzen Gruppe. Karl
Thomas lässt diese unverhoffte Nachricht wahnsinnig werden. Die scheinbare
Willkür der Regierung passt nicht zu seinem Wahlspruch \Cite{Guter Sieg oder
guter Tod} \SourceRef{III}{15}, das Warten in der Zelle erscheint ihm
nachträglich als mentale Todesfolter. Den angeblichen \Cite{Willen zur
Versöhnung} \SourceRef{III}{20} des Präsidenten kann er nicht ernst
nehmen. Die von diesem ersten Zusammentreffen republikanischer
(Schein-)Humanität mit drangvoller Revolutionslogik bewirkte Wahnreaktion
führt zugleich den Topos des \Quote{Irrens an der Welt} ein, der für die
spezifische Form der Modernereflexion in \Title{Hoppla, wir leben!}
kennzeichnend ist.

\HeadingTwo{Die republikanische Welt: Pluralität und Widersprüche}

Mit der Entlassung des Protagonisten aus der Irrenanstalt wird die Perspektive
der Hauptfigur Karl Thomas gewisssermaßen \Quote{sichtleitend} für die
Darstellung der republikanischen Gegenwart.\Footnote{\Abr{Vgl}
  \BibRef{benson.87}{119}: \Cite{Karl Thomas ist Tollers Ideenvermittler, dem
    durch die wechselnden Situationen, in die er gestellt wird, die
    Möglichkeit zur kritischen Analyse der untergehenden Weimarer Republik
    gegeben wird.}}
Die \Cite{Grundidee} des Dramas,
über die zwischen Toller und Piscator weitgehend Einigkeit herrschte, war
\Cite{der Zusammenprall eines Revolutionärs, der acht Jahre im Irrenhaus
  zugebracht hatte, mit der Welt von 1927.}\Footnote{Siehe 
  \BibRef{piscat.29}{146}. In Tollers Rückblick von 1930 ist die Rede vom
  \Cite{Zusammenprall des Menschen, der das Absolute unbedingt, noch heute
    verwirklichen will, mit den Kräften der Zeit und den Zeitgenossen, die die
    Verwirklichung [..] aufgeben oder [..] für spätere Tage vorbereiten.}
  \SourceRef{I}{145}.}

In der Figur des Karl Thomas ist eine spätexpressionistisch revolutionäre
Weltsicht konserviert. In der nahezu ereignislosen Separation in der
Psychiatrie blieb sein Leben zirkulär. In exakter Wiederholung bewahrte sich
der Hass des Ausgesonderten auf die \Quote{gnädige} neue
Staatsführung.\Footnote{\Abr{Vgl} \SourceRef{III}{25}:
  \Cite{\DramName{Professor Lüdin}: [..] jedes Jahr einmal fingen Sie an zu
    toben. Man musste Sie isolieren. [..]  \DramName{Karl Thomas}: Am Tag der
    Begnadigung \ldots [..] Ich haßte die Begnadigung. Ich haßte den
    Präsidenten.}}

In der Konfrontation dieser Figur mit der um Jahre \Quote{vorausgeeilten} und
durch gesellschaftliche Modernisierungsprozesse veränderten Gegenwart
entwickelt das Drama seine Zeitkritik als eine fassungslose 
\Quote{Außensicht}. Der Protagonist irrt durch die verschiedenen
gesellschaftlichen Teilräume und Sphären, sammelt auf der Suche nach den alten
Weggefährten gleichsam die Widersprüchlichkeiten der Gegenwart in sich auf und
artikuliert seine gefühlsmäßige Wahrnehmung als Zweifel an der Sinnhaftigkeit
des Erlebten.

Pluralisierung, Pragmatismus und Sachlichkeit -- das unpathetische
Nebeneinander und Durcheinander des republikanischen Lebens -- können aus
dieser \Quote{unzeitgemäßen} Perspektive nicht mehr ordnend erfasst
werden. Aus Sicht von Karl Thomas scheint der Welt die innere Kohärenz
abhanden gekommen zu sein. Seine frühere psychiatrische
Aussonderung invertierend, macht er seiner Umgebung den wiederholten Vorwurf der
\Quote{Verrücktheit}.\Footnote{So \AbrPair{z}{B} \Cite{Bin ich in ein
    Tollhaus geraten?} \SourceRef{III}{70} oder \Cite{Ist die Welt ein Irrenhaus
    geworden?} \SourceRef{III}{95}.}  
Die Bewältigung dieser Realität scheint ihm nur per Umsturz oder Flucht
möglich.\Footnote{Sein Tatendrang ist die aktivistische Variante eines
  Harmonieideals, das er in schwachen Momenten als Sehnsucht nach
  einem naiv \Quote{ursprünglichen} Leben zu erkennen gibt: \Cite{Es müssen irgendwo
    noch Menschen leben, kindliche, die sind, nur sind. [..] Die nichts von
    Politik wissen, die leben, nicht immer kämpfen müssen.} \SourceRef{III}{50}.}

Zentrum und Zuspitzungsort seiner \Quote{Odyssee der Eindrücke} ist das
\Cite{Grand Hotel}, in dem er -- gewappnet mit der grotesken \Cite{Fassade}
eines Dauerlachens\Footnote{Zu seinem aufpolierten Äußeren gehört die
  Befolgung eines Ratschlags zeitgenössischer \Cite{Chefs}: \Cite{In unserer
    Zeit muß man lachen, immer lachen.}  \SourceRef{III}{79}.}  
-- eine Arbeit als Kellner angenommen hat. In den Räumen, Zimmern und
Einrichtungen des Hotels kommt die urbane Vielschichtigkeit der Hauptstadt in
dichter Überblendungsfolge zur Darstellung. Karl Thomas erlebt die finanziell
notdürftige Lage der Dienstboten, die zu Geldentwertung und schlechten
Arbeitsbedingungen \Quote{gute Miene} machen müssen, als direkten Kontrast zu
den opulenten Diners der politisch-wirtschaftlichen \Quote{Macher} und elitär
räsonnierenden Intellektuellen. Während ein Hausdiener mit dem
konsternierenden Verlust seiner jahrelang ersparten Habe durch die
Währungskrise fertig werden muss,\Footnote{\SourceRef{III}{90}:
  \Cite{Sechshundert Wochen habe ich gespart. Zwölf Jahre. Und was bekam ich
    zuguterletzt? Einen Dreck!}}  
verhandeln \Cite{geistige Kopfarbeiter} in einem \Cite{Klubzimmer}, wie das
Proletariat \Cite{geistig zu erlösen} sei \SourceRef{III}{84-85}.
Innenminister und Ex-Revolutinär Kilman, der sich nur noch lose mit dem Ziel
einer sozialeren Gesellschaft assoziiert, speist mit Finanzmagnaten, die ihn
zu bestechen suchen, während seine Tochter mit einem Drahtzieher der
nationalistischen Reaktion im Bett liegt und aus sexueller Langeweile die
Vorzüge von \Cite{Koks} und \Cite{Cordon rouge} gegeneinander abwägt
\SourceRef{III}{89}.

Eine quasi globale, medial vermittelte Übersteigerung erhält die Darstellung,
als der Protagonist in der \Cite{Radiostation} des Hotels Zeuge der
\Cite{Nachrichten aus aller Welt} \SourceRef{III}{82} wird: Werbung für
Luxusmode ertönt unvermittelt neben Meldungen über Katastrophen und
kriegstechnische Neuerungen. Von den bisher ungekannten technischen
Möglichkeiten überwältigt, staunt Thomas über die zerstörerische und
widersinnige Anwendung des Fortschritts.\Footnote{\SourceRef{III}{83}:
  \Cite{\DramName{Karl Thomas}: Wie wundervoll ist das alles! Und was machen die
    Menschen damit \ldots [..].} Der nüchterne Kommentar des Technikers:
  \Cite{\DramName{Telegraphist}: Vorläufig dienen diese Apparate dazu, damit die
    Menschen sich desto raffinierter totschlagen. [..] Wir werdens nicht
    ändern.}.}

Die \Quote{Einblicke} in die Miniaturwelt des Hotels als lokale Verdichtung
widersprüchlicher Tendenzen werden durch den globalen \Quote{Ausblick} auf die
gefährliche Rasanz von Wirtschaft, Technik und Militarismus zusätzlich
akzentuiert. Das Gesamtbild einer sich bewusstlos krisenhaft entwickelnden
Welt forciert beim Protagonisten den Eindruck der Irrationalität der
Verhältnisse und nährt sein Bedürfnis, mit \Quote{beispielhafter} Tat dagegen
anzugehen.

Die der Darstellung immanente kritische Norm kann bezeichnet werden als eine
Mischung aus revolutionärem Gerechtigkeitsideal, emotionalem
Ehrlichkeitsdrang und dem Bedürfnis nach einer durchschaubaren, vernünftig
geregelten Welt. Die Anwendung dieser kritischen Norm auf die vorwiegend sozialen
und technisch-ökonomischen Ungereimtheiten des republikanischen Lebens ist in
erster Linie im beschriebenen Szenario des \Cite{Grand Hotel} zu erkennen.  In
Tollers Erstfassung ist dies noch stärker expliziert, indem Karl Thomas am
Dramenende über seine Eindrücke zusammenfassend Auskunft gibt:

\begin{BlockQuote}
Ich war Kellner. Einen Abend lang. Ich schmorte in einem Hexenkessel. Es stank
nach Korruption, nach Geilheit, nach Dünkel, nach Dreck. Die Kollegen fanden
es in Ordnung und waren stolz darauf. \SourceRef{III}{319}
\end{BlockQuote}
Wie hier schon angedeutet ist, vollzieht sich die Konfrontation des
Protagonisten mit der Welt als eine sukzessive Enttäuschung, die bis zum
\Cite{Ekel} \SourceRef{III}{66} geht. Seine Empörung über die
Selbstgefälligkeit der gehobenen Kreise wird durch die unkritische Haltung der
\Quote{niederen Chargen} noch verstärkt und zu allgemeinem Unverständnis
erweitert.

In politischer Hinsicht artikuliert sich dieses allgemeine Unverständnis am
deutlichsten bei den Zusammentreffen des Protagonisten mit den verschiedenen
Ex-Revolutionären. Um der politischen Aussage von \Title{Hoppla, wir leben!}
auf den Grund gehen zu können, wird deshalb im folgenden Abschnitt die
dramatische Problematisierung der politischen Sphäre erörtert.

\HeadingTwo{Die politische Sphäre zwischen Pragmatismus und Reaktion}

Zentrales Moment der politischen Charakterisierung der \Cite{Welt von 1927},
wie sie in \Title{Hoppla, wir leben!} zum Ausdruck kommt, ist der Eindruck
einer \emph{vordergründigen Demokratisierung}. Ehemalige Arbeiterfunktionäre
wie Wilhelm Kilman können bis zu Ministerposten aufsteigen, was im
Vergleich zur relativ hermetischen politischen Klasse des Kaiserreiches als
Fortschritt erscheinen muss. Außerdem finden Wahlen statt, in denen Kandidaten
verschiedener sozialer Provenienz scheinbar gleichberechtigt
konkurrieren.\Footnote{\Abr{Vgl} \BibRef{schnei.95}{454}: \Cite{Die neue
    Verfassung erhebt das Volk zum Souverän; die Masse wandelt sich zum
    umworbenen Wahlvolk.} In \Title{Hoppla, wir leben!} wird dieses
  \Quote{Umwerben} durch die marktschreierische Darstellung der Wahlhelfer
  schon fast als Rummelplatz-Belustigung karikiert. \Abr{Vgl}
  \SourceRef{III}{60-61}.}
Die Arbeiter beteiligen sich zudem konstruktiv an den demokratischen Verfahren
und können sich auch sonst auf Verfassungsrechte
berufen.\Footnote{\SourceRef{III}{28}: \Cite{\DramName{Eva Berg}: Ich übe die
    Rechte aus, die die Verfassung mir gewährt.}}
Auch zeigen sich gewisse Emanzipationserscheinungen im Arbeitsleben, die
weibliches Engagement in betrieblichen Zusammenhängen
ermöglichen,\Footnote{Dies wird am Beispiel Eva Bergs gezeigt.} ohne sich aber
politisch-institutionell weiter auszuwirken.

Diese vordergründigen Neuerungen werden durch eine Reihe von tatsächlichen Defiziten
und Verfälschungen des republikanischen Staatssystems konterkariert. Ein großer
Teil des Dramas, insbesondere die Szenen in Kilmans Ministerium und im
\Cite{Wahllokal} der Arbeiter, bebildern genau diese Mängel.
Kilman, der sich als \Cite{demokratisch} \SourceRef{III}{40} bezeichnet,
rechnet es sich gleichzeitig als \Cite{Mut} an, gelegentlich \Cite{gegen das
  Volk zu regieren}. Als Staatsräson gilt ihm sein \Cite{gesunder
  Menschenverstand} \SourceRef{III}{37}, mit dem er sogar die Verfassung
relativiert\Footnote{\SourceRef{III}{28}: \Cite{\DramName{Wilhelm Kilman}: Die
    Verfassung ist für ruhige Zeiten gemacht. \DramName{Eva Berg}: Leben wir nicht
    in ihnen? \DramName{Wilhelm Kilman}: Der Staat kennt selten ruhige Zeiten.}}
und den abstrakten \Cite{Mechanismus des Staates} \SourceRef{III}{41} über die
Interessen der Arbeitenden stellt. Der \Cite{Wille des ganzen Volk}
\SourceRef{III}{40} ist der biegbare Titel seiner Handlungsmaxime, mit der er
sich zugleich über die Belange einer womöglich unzufriedenen \Cite{Masse}
erhebt, die er für \Cite{unfähig} und der \Cite{Erziehung} bedürftig befindet
\SourceRef{III}{42}. Wahlmanipulationen,\Footnote{\SourceRef{III}{40}:
  \Cite{\DramName{Albert Kroll}: Kilman hat den Arbeitern bei den Chemischen
    Werken das Stimmrecht gestohlen!}}  der Verdacht der Bestechlichkeit und sein klares
Bewusstsein von der verlogenen Doppelbödigkeit staatsmännischer Rhetorik
\SourceRef{III}{36} vervollständigen die Charakterisierung Kilmans als
autoritäre, heuchlerische Politikerfigur.

Während die in der Person Kilmans fokussierten scheindemokratischen Momente
dem Protagonisten Karl Thomas zum Großteil unmittelbar erkenntlich werden,
bleibt der Aspekt der zusätzlichen \emph{nationalistischen Unterwanderung} des
Staatsapparates für ihn und die anderen regierungskritischen Figuren zunächst
im Dunkeln und erscheint nur als unbestimmter
Verdachtsmoment.\Footnote{\SourceRef{III}{41}: Karl Thomas zu Kilman:
  \Cite{Ihr verhelft der Reaktion in den Sattel.}}
Die Darstellung dieses Entwicklungsstrangs als geheime Verschwörung
\SourceRef{III}{75-77} erzeugt so
einen Informationsvorsprung beim Zuschauer, der bereits auf die dramatische
Zuspitzung der Attentatshandlung hinzielt.

Karl Thomas sieht den Verrat an den früheren Revolutionsidealen zwar wesentlich
durch Kilman verkörpert, tritt aber auch zu den anderen früheren
\Cite{Genossen} in Konflikt, indem er ihre konstruktive Haltung gegenüber der
demokratischen Ordnung als \Cite{falsch} \SourceRef{III}{66} und \Cite{feige}
\SourceRef{III}{67} abwertet. 
Er vermisst \Cite{Taten}, wo Albert Kroll sich über
\Cite{Wahlstimmen} freut \SourceRef{III}{70}, und verflucht die \Cite{Ruhe}
\SourceRef{III}{72}, mit der Kroll seinen Aktionismus \Cite{bremst}
\SourceRef{III}{73}. 

Mit dieser argumentlosen Unbändigkeit, die lediglich mit Appellen an den
\Cite{Glauben} \SourceRef{III}{65} aus Revolutionstagen zu überzeugen
versucht, äußert sich eine Spaltung der Autornorm in einen Wahrnehmungs- und
einen Handlungaspekt: Es wird spürbar, dass Karl Thomas die Unzulänglichkeiten der
Gegenwart zwar intensiv empfinden, für ihre Bewältigung aber kein angemessenes
Konzept entwickeln kann.\Footnote{Sigurd Rothstein sieht in  dieser Mischung aus
  starkem Empfinden und Mangel an Handlungskompetenz ein Charakteristikum der
  \Cite{neuen Hauptfiguren Tollers}, zu denen er neben Karl Thomas auch den
  Protagonisten des \Title{Hinkemann}-Dramas von 1923 zählt. Siehe
  \BibRef{rothst.87}{157}.} 

Demgegenüber wird den pragmatisch agierenden Arbeiterfiguren ein höheres Maß
an Handlungskompetenz zugeordnet. Um dies klarer zu konturieren, haben Toller
und Piscator insbesondere an der Figurencharakteristik von Eva Berg erhebliche
Änderungen vorgenommen. Es gehört zu den hervorzuhebenden Leistungen Karl
Leydeckers, diese Änderungen anhand der Erstfassung rekonstruiert zu haben. 
Er kommt zu dem Resultat, dass die Charakterisierung Bergs als zielstrebige,
disziplinierte Parteikämpferin erst durch den Einfluss Piscators entstanden
sei. Die Szene, in der Karl Thomas ihr seine Gefühle offenlegt und von der
gemeinsamen Realitätsflucht schwärmt,\Footnote{\SourceRef{III}{50}:
  \Cite{\DramName{Karl Thomas}: [..] Du sollst mir Morgen sein und Traum der
    Zukunft. Dich, dich will ich, nichts weiter. \DramName{Eva Berg}: Flucht
    also?}}  
ist laut Leydecker Beleg, wie aus einer negativ gezeichneten,
emotionslosen Figur, die bei Toller offensichtlich die verhärtete Sachlichkeit
des republikanischen Lebens illustrieren sollte, eine im positiven Sinne harte
Vorzeigefigur gezimmert worden ist, die damit allerdings ihre Stimmigkeit
verlor:
 
\begin{BlockQuote}
Whereas Toller had presented this hardened character in a negative light in
the first version, for Piscator this hardness was a positive
characteristic. The result of their collaboration was a deeply ambivalent
character in the second version and this has led to the diverging judgements
of her by literary critics.\Footnote{\BibRef{leydec.98}{127}.}
\end{BlockQuote}
Ist Karl Thomas in der Erstfassung noch der empfindsame Idealist, der von
einer verhärteten Frau geradezu bemitleidenswert verstoßen wird, so erscheint
er in der Druckfassung als naiver Faulpelz, der eine Vorkämpferin des
Proletariats von ihrer Arbeit abhält. Letztere Darstellung trägt naturgemäß
dazu bei, ihn als politische Figur zu diskreditieren und die Handlungsnorm des
Stückes auf eben jene disziplinierte Parteiarbeit auszurichten.

Die politische Lage, die in \Title{Hoppla, wir leben!} als vordergründiges
demokratisches Kräftegleichgewicht und scheinbar neutrale Ordnung kritisch
beleuchtet wird, stellt sich aus der Perspektive des Protagonisten insgesamt
als unerträgliche bis eklige\Footnote{\Abr{Vgl} dazu auch \SourceRef{III}{64}:
  \Cite{\DramName{Karl Thomas}: [..] Kein frischer Hauch. Die Luft modert vor
    Ordnung. [..] Es schimmelt nach Bürokratie.}} 
Ruhe dar. Der pragmatischen, mittelfristig
angelegten Parteiarbeit der früheren Genossen steht sein emotionaler
Aktionsdrang, sein subjektiver Spontaneismus entgegen. Als empfindsamer und
noch in Ganzheitsidealen denkender \Quote{Seismogrpah} einer widersinnigen
Welt ist die Figur Karl Thomas zwar durchaus als wesentlicher Träger der
kritischen Norm des Stückes zu sehen. Seine Handlungsperspektive, \emph{die
ungeduldige singuläre Gewalttat}, kann dagegen sowohl in Tollers
Originalfassung als auch in der \Quote{Piscatorfassung} nur als Verfehlung
verstanden werden.

Im folgenden Abschnitt wird der Weg der Hauptfigur
zu dieser Verfehlung als fortschreitende Isolation und Verunsicherung
nachgezeichnet. Erst die Zuspitzung dieser Separation am Dramenschluss
liefert dann den letzten Baustein zur hermeneutischen
Erschließung der Dramenaussage.\Footnote{Dabei wird sich zeigen, wie die
  unterschiedlichen Dramenschlüsse zu divergenten Analysen führen.} In jedem
Fall wird sich aber zeigen, dass die Überwindung einer vorläufigen und von
reaktionären Umtrieben unterwanderten Scheindemokratie die politische
Zielrichtung des Dramas bestimmte. Ernst Toller und Erwin Piscator waren in
vielen Punkten zerstritten, aber über die Unzulänglichkeit der Republik von
1927 waren sie sich zweifellos einig.

\HeadingTwo{Die Isolation des Protagonisten}

Die \Cite{Grundidee} des Dramas, die in der Konfrontation des Protagonisten
mit der gesellschaftlichen Gegenwart besteht, impliziert einerseits dessen
Rand- bzw. Außenstellung, andererseits erhält seine kritische Wahrnehmung der
Welt dabei einen hervorgehobenen Wert. 

Ein grundlegender Faktor für die Sonderstellung von Karl Thomas ist sein von
allen anderen Figuren abweichender \emph{Erfahrungshintergrund}. Die
Kriegserlebnisse, die seine Revolutionsbeteiligung motivierten, sind ihm noch
voll präsent und die Leitidee von der Verbrüderung aller Menschen setzt er
offenbar als selbstverständlich voraus.\Footnote{Interessanterweise ist sein
  Gespräch mit zwei Kindern \SourceRef{III}{53-57} die einzige Stelle, an der er
  überhaupt eine reflektierte Begründung seiner unbedingten Umsturzideale
  gibt. Bei allen anderen Figuren geht er scheinbar davon aus, dass diese Dinge
  klar sein müssten.}  
Die acht Jahre im Irrenhaus, die in seiner Erinnerung \Cite{wie ausgelöscht}
\SourceRef{III}{24} sind, haben ihn aber vom veränderten Lebensgefühl seiner
Altersgruppe entfremdet. Die dadurch entstandene Distanz ist zugleich
Gradmesser für die Rasanz der gesellschaftlichen Entwicklung und
Anzeichen für den prägenden Effekt, dem die in ihr \Quote{alltäglich} Lebenden
ausgesetzt sind:

\begin{BlockQuote}  
\DramName{Eva Berg}: Ich merke, wenn ich mit dir spreche, die letzten acht
Jahre, in denen du \Quote{begraben} warst, haben uns stärker verwandelt als
sonst ein Jahrhundert.\\
\DramName{Karl Thomas}: Ja, ich glaube mitunter, ich komme aus einer
Generation, die verschollen ist. \SourceRef{III}{51}
\end{BlockQuote}
Während Thomas latent daran leidet, Teil eines nur noch in der Vorstellung
oder Erinnerung existenten Sinnkontextes zu sein, deklarieren und verhalten
sich alle anderen Figuren streng gegenwartsbezogen. Ihr unaufgeregtes Leben in
der Republik wird als Ergebnis von Lernprozessen dargestellt, über die sie
selbst nur vage oder unwillig Auskunft geben. Es ist demnach eben kein einzeln
benennbares \Cite{Erlebnis} gewesen, das Eva Berg \Cite{verhärtet} hätte
\SourceRef{III}{52}. Vielmehr erscheinen der republikanische Alltag und das
Arbeitsleben als die Hauptfaktoren der
Gewöhnung.\Footnote{\SourceRef{III}{53}: \Cite{ \DramName{Eva Berg}: Seit acht
    Jahren arbeite ich, wie früher nur Männer arbeiteten. Seit acht Jahren
    entscheide ich über jede Stunde meines Lebens. Darum bin ich wie ich bin.}}

Den Vorwurf, er spreche resigniert \Cite{wie ein alter Mann}, beantwortet auch
Albert Kroll mit dem Hinweis auf die besondere Prägekraft der vergangenen
Jahre, die \Cite{zehnfach} zählten \SourceRef{III}{64}. Die Härte des
\Cite{Alltag[s]} \SourceRef{III}{63} rege
ihn inzwischen nicht mehr auf. Sein nüchterner Kommentar lautet: \Cite{Man
  lernt.}  \SourceRef{III}{64}. 

Die revolutionären Erlebnisse sind insgesamt
wie romantische Kindheitstage in die Ferne gerückt und Karl Thomas wird
empfohlen, endlich erwachsen zu werden:

\begin{BlockQuote}
\DramName{Eva Berg}: [..] Wir können es uns nicht mehr leisten, Kinder zu
sein. Wir können Hellsichtigkeit, Wissen, das uns zuwuchs, nicht mehr in die
Ecke werfen wie Spielzeug, das wir nicht mögen. \SourceRef{III}{52-53}
\end{BlockQuote} 
In der reflektierten Aufarbeitung von Erfahrungen, die hier nahe gelegt wird,
ist der Prozess des \Quote{Erwachsenwerdens} positiv konnotiert. Für Karl
ist Ruhe, Vernunft und Selbstkontrolle aber zu sehr 
mit dem Beigeschmack der Anpassung, Unterordnung und
Selbstverleugnung verbunden. Denn auch der verhasste Wilhelm Kilman
relativiert die gemeinsame Vergangenheit mit dem Hinweis, \Cite{was für Kinder
wir waren}, und von Kilmans Tochter muss sich Thomas gar sagen lassen, dass
jeder einmal \Cite{groß} werde -- zum Beispiel als Minister \SourceRef{III}{39}.

Karl Thomas kann diese Varianten der Einpassung nicht
auseinanderhalten.\Footnote{Toller kommentierte dies in \Quote{Arbeiten}:
  \Cite{Karl Thomas versteht beide nicht, setzt ihre Motive und Taten gleich
    und geht unter.} \SourceRef{I}{145}.}  
Aus seiner Perspektive gibt es nur zwei Handlungsalternativen: Akzeptanz oder
aktive Zerschlagung der Verhältnisse. Da
er jegliche Mitarbeit an oder in dem bestehenden System als unerträglich
verwirft, ist die direkte revolutionäre Tat seine einzige Option.

In den Dialogen des Dramas wird kaum über politische Zielsetzungen und Inhalte
reflektiert. Zwar sind sowohl Berg als auch Kroll als politische Aktivisten
dargestellt, doch über Sinn und Ziele ihrer Arbeit machen sie nur vage
Äußerungen.\Footnote{Eva Bergs Zielsetzungen sind vom
  Willen \Quote{der Partei} bestimmt: \Cite{Dann kam die
    Arbeit. Die Partei brauchte mich.} \SourceRef{III}{53}, und Kroll
  wartet auf unbestimmte zukünftige Chancen: \Cite{Weil ich mit Volldampf
    fahren will, wenns Zeit ist.} \SourceRef{III}{73}.}
Die Auseinandersetzung der Figuren bleibt
also, grob gesagt, auf die Alternativen von entschlossener Revolution oder
taktierendem Abwarten beschränkt. Politisches Handeln ist in dieser Logik eine
Frage des \emph{Gefühls}, der charakterlich-emotionalen
Disposition.

Der Charakter von Karl Thomas wird bereits im Vorspiel hinreichend
exponiert. Schon dort ist er als unbändiger Aktionist, spontan und emotional
gezeichnet\Footnote{Schon seine ersten Worte sind eine Beschwerde darüber,
  dass nichts Wahrnehmbares geschieht: \Cite{Verfluchte
    Stille!} \SourceRef{III}{13}.}
und erhebt sein Umsturzideal zur positiven Norm,\Footnote{Auch das macht er schon
  im Vorspiel klar: \Cite{Wagen muß man, Genosse!} \SourceRef{III}{15}.}  
Die frühere revolutionäre Gemeinschaft ist ihm verbindlicher
Anknüpfungspunkt für seinen aktionistischen Imperativ und das
zugehörige Gemeinschaftsideal: Wer seine Ziele teilt, muss
bedingungslos seinen Aktionismus teilen, sonst ist er ein \Cite{Feigling}
\SourceRef{III}{66}. Mit dieser Sichtweise zieht er sich immer mehr auf sich
als einzig wahren Kämpfer zurück, der auch vor dem Selbstopfer nicht zurück
schreckt.\Footnote{Siehe \SourceRef{III}{66}:
  \Cite{\DramName{Karl Thomas}: [..] Einer muß sich opfern. [..] Jetzt weiß
    ich, was ich zu tun habe.}}
Die bedingungslose Opferbereitschaft, die in der \Title{Wandlung} und in
\Title{Masse Mensch} noch als besonderes Merkmal messianischer Führer galt,
ist allerdings als isolierender Irrweg dargestellt.\Footnote{\Abr{Vgl}
  \BibRef{chong.98}{147}: \Cite{Die politische Aktion, der Selbstopfergedanke
    als politisches Mittel, führt zum blinden Fanatismus.}}  

Die emotionale Charakterisierung des Protagonisten bestimmt auch seine
kommunikativen und sprachlichen Merkmale. Er ist unfähig, mehr als seine Gefühle zu
kommunizieren. Es ist nicht zuletzt seine Ausdrucksweise, die mit ihren Pathos
und den \Cite{Begriffen, die nicht mehr stimmen} \SourceRef{III}{52}, dafür
sorgt, dass es zwischen
ihm und denen, die er der Tatenlosigkeit bezichtigt, zu keiner inhaltlichen
Verständigung kommt.\Footnote{Kirsten Reimers weist auf die besondere
  Verknüpfung von Erfahrungshintergrund und Kommunikationsproblemen
  hin. \Abr{Vgl} \BibRef{reimer.00}{169}.} 
Die divergenten Sinnkontexte von Gegenwart und Revolutionszeit werden auch
über den Sprachstil markiert. Reste expressionistischer Diktion finden sich
im Vorspiel und bleiben ansonsten Karl Thomas vorbehalten.\Footnote{\Abr{Vgl}
  \BibRef{reimer.00}{169-170}.}

Einen skurrilen \Quote{Begleiter} hat Thomas in einer Figur namens
\Cite{Pickel}. Dieser steht in einer gewissen Analogie zu ihm, da auch er
ein Außenseiter ist, der keinen Anschluss findet und die republikanische Welt
nicht versteht.\Footnote{\Abr{Vgl} \BibRef{chong.98}{138}: \Cite{Die
    Gemeinsamkeit von Thomas und Pickel besteht in ihrem Fixiertsein auf alte
    Denkmuster und in ihrer Verständnislosigkeit gegenüber der technischen
    Entwicklung.}}
Seine dörfliche Herkunft und sein naiver Glaube an angebliche 
Errungenschaften der neuen Staatsform bilden eine groteske Verstärkung und
Überzeichnung der kritischen \Quote{Außensicht} des Protagonisten. Dem
unverstandenen Patheten Karl Thomas ist mit Pickel gewissermaßen ein
\Cite{karikiertes Gegenbild}\Footnote{\BibRef{klein.68}{141}. \Abr{Vgl} auch
  \BibRef{reimer.00}{180f.}.}
zur Seite gestellt.

Als Ausweg aus seiner Isolation wird Karl Thomas mehrfach empfohlen, sich eine
Arbeit zu suchen.\Footnote{\SourceRef{III}{73}: \Cite{\DramName{Karl Thomas}:
    Was soll ich denn tun, um euch zu verstehen? \DramName{Albert Kroll}:
    Arbeite Irgendwo. [..] Du mußt in den Alltag hinein.} oder
  \SourceRef{III}{49}: \Cite{\DramName{Eva Berg}: Ja, es wäre Zeit, daß du
    Arbeit fändest.}}
Er befolgt den Ratschlag  mit innerem Widerwillen und hält es trotz einer
Maske des Lachens, die seine Abwehrhaltung überdecken soll, nur für ein paar
Stunden aus. Aus dem Gefühl der inneren Abscheu heraus, fasst er den diffusen Plan,
\Cite{alle} zu \Cite{wecken} \SourceRef{III}{92} und damit den
ignoranten \Quote{Schlafzustand}\Footnote{\Abr{Vgl} \SourceRef{III}{91}: \Cite{Ihr schlaft!
    Ihr schlaft!}}
des Alltagstrotts zu durchbrechen. Mit der Waffe in der Hand schreitet er zum
Attentat auf den Minister.

Die schrittweise Andeutung einer reaktionären Gefährdung der Demokratie wird
mit dem  Paralleleffekt des Attentats schlagartig verdeutlicht und als
Kennzeichnung des Irrwegs von Karl Thomas ausgedeutet. Als der tatsächliche
Attentäter sich nicht als \Cite{Genosse} sondern als Nationalist
erweist,\Footnote{Verschiedene Autoren betonen die Entsprechungen zwischen
  Thomas und dem nationalistischen Studenten hinsichtlich ihrer Überbewertung
  der singulären \Cite{Tat}. Damit erhalte Thomas ein Gegenstück, dass die
  Inhaltslosigkeit seiner politischen Perspektive aufdecke. Siehe \AbrPair{z}{B}
  \BibRef{grunow.94}{102}, oder \BibRef{herman.81}{168}.}
der den von ihm getöteten Kilman als \Cite{Revolutionär}, \Cite{Bolschewiki}
und Judenfreund beschimpft, erkennt Thomas den Widersinn des eigenen Mordvorhabens:

\begin{BlockQuote}
\DramName{Karl Thomas}: Du siehst ein Haus brennen, packst den Eimer, willst
löschen, und anstatt Wasser gießt du Öl fuderweise in die Flammen \ldots
\SourceRef{III}{95}
\end{BlockQuote}
Der emotionsgeleitete Aktionismus hat sich für Karl Thomas gleichsam als
Schlag ins eigene Gesicht erwiesen und lässt sein Vertrauen in die Kraft des
Gefühls schwinden. Zur Isolation kommt die Selbstentfremdung hinzu, was
 unmittelbar nach dieser Einsicht zum völligen Bezugsverlust führt:

\begin{BlockQuote}
\DramName{Karl Thomas}: Ich bin der Welt abhanden gekommen / Die Welt ist mir
abhanden gekommen
\SourceRef{III}{96}
\end{BlockQuote}
Die aktionistische Vorgehensweise von Thomas ist damit eindeutig in ein
negatives Licht gesetzt. Daraus ergibt sich ein Wendepunkt des Dramas. Die
Endstufe im Isolationsweg des Protagonisten, der nunmehr jegliche
Vorbildfunktion eingebüßt hat, ist eingetreten.
Die Verunsicherung des bisher von seiner einzigartigen Sensibilität
überzeugten Karl Thomas erhält abermals die Form des Wahnsinns, des
psychischen Identitätsverlusts. Diese höchste Isolationsstufe findet 
ihre endgültige Auflösung in der folgenden \Quote{Schlussbetrachtung} des
Dramas.

\HeadingTwo{Politische Bewältigungskonzepte und zwiespältige Autornorm}

In der Druckfassung, die als die für Öffentlichkeit und Forschung
letztlich gültige Textform verstanden werden muss und deshalb auch die
wesentliche Grundlage der vorliegenden Analyse bildet, wird die
Isolationsentwicklung des Protagonisten nach dem Attentat stufenweise
fortgesetzt. Der Verhaftete fühlt sich sogar von seiner Waffe
verhöhnt\Footnote{\SourceRef{III}{96}: \Cite{\DramName{Karl Thomas}: Was weiß
    ich? Was weißt du? Sogar der Revolver kehrt sich gegen den Täter, und aus
    dem Lauf spritzt Gelächter.}} 
und kann sich gegenüber der Polizei nicht klar ausdrücken. \Cite{Das Volk} ist
gegen ihn als vermeintlichen Ministermörder aufgebracht und will ihn
\Cite{lynchen} \SourceRef{III}{96, 99}. Sein inhaltsloses Aktionsprimat hat
zwar dafür gesorgt, dass die unerträgliche \Cite{Ruhe} vorübergehend durchbrochen
ist, doch der ausgelöste Volkszorn kehrt sich gegen ihn selbst und damit wird
sein \Cite{Selbstopfer-Idealismus ad absurdum
  geführt}.\Footnote{\BibRef{buetow.75}{327}.}

In den Verhören der Polizei ist es ausgerechnet \Cite{Traugott Pickel}, die lebende
Karikatur des verwirrten Außenseiters, der als Komplize von Karl Thomas in
Verdacht gezogen wird. Die dissoziierte Randstellung des Protagonisten wird
dadurch noch betont.
Auch vor Gericht ist es nicht Karl Thomas, der den Überführungsversuchen des
Untersuchungsrichters etwas entgegen zu setzen hat. Während die
\Cite{Genossen} Eva Berg und Mutter Meller souverän und solidarisch
auftreten und ihr klares Rechtsbewusstsein artikulieren, hat er für die
Verhandlung nur seinen notorischen Irrsinnsvorwurf
parat.\Footnote{\SourceRef{III}{96}: \Cite{\DramName{Untersuchungsrichter}:
    [..] Thomas, was sagen Sie zu den Aussagen? \DramName{Karl Thomas}: Daß
    ich allmählich den Eindruck bekomme, ich befinde mich in einem
    Irrenhaus.}}
Die \Cite{Kameraden} stärken ihm zwar moralisch und durch ihre Aussagen auch
beweistechnisch den Rücken, doch sein Weg in die Sonderrolle des
\Quote{Verrückten} wird davon nicht aufgehalten. Seine Verzweiflung am
\Cite{Irrsinn der Welt} bleibt verquickt mit einem generellen Gefühl der
Vereinzelung. 

Nach der Untersuchung in der Psychiatrie wird dies in den
abschließenden Einzelhaftszenen im Gefängnis zu Ende geführt.
Die Satzfetzen seines rasenden Schlußmonologs dokumentieren ein
Wahrnehmungsschema, das die Mitmenschen als Teile eines bewusstlosen,
zerstörerischen Zirkulärdaseins empfindet und aus der völligen Bezugslosigkeit 
nur noch den Ausweg des Suizids zu wählen weiß:

\begin{BlockQuote}
\DramName{Karl Thomas}: [..] Du, werde ich dich nie verstehen? \ldots Du,
wirst du mich nie begreifen?
\ldots Nein! Nein! Nein! \ldots Warum zertrümmert, verbrennt, vergast ihr die
Erde ? \ldots Alles vergessen? \ldots Alles umsonst? \ldots So dreht euch
weiter im Karussell, tanzt, lacht, weint, begattet euch -- viel Glück! Ich
springe ab \ldots [..] \SourceRef{III}{115}
\end{BlockQuote}
Nach Auffassung Erwin Piscators sollte mit dem tragischen Ende des
\Cite{klassenmäßig entwurzelten} Karl Thomas der \Cite{anarchisch
  sentimentale} Typus in seinem zwangsläufigen Scheitern gezeigt werden. Den
positiven Part erhalten in dieser Konsequenz die \Cite{klassenbewussten}
proletarischen \Cite{Gegenspieler} in Gestalt von Eva Berg, Albert Kroll und
Mutter Meller.\Footnote{\Abr{Vgl} \BibRef{piscat.29}{148 und 153}.}
In Piscators Berliner Regiebuch bleibt diesen drei Figuren denn
auch die explizite Schlusskommentierung des Protagonisten vorbehalten:

\begin{BlockQuote}
\DramName{Kroll}: Das durfte er nicht tun, so stirbt kein Revolutionär.\\
\DramName{Eva}: Der Alltag hat ihn zerbrochen.\\
\DramName{Meller}: Verdammte Welt! -- Man muß sie
ändern. \SourceRef{III}{326}\Footnote{Es ist nicht ganz klar, ob exakt diese
  Repliken auch zur Aufführung kamen, denn in seinen Nachbetrachtungen ordnet
  Piscator der Mutter Meller die (zusammenfassendere) Aussage zu: \Cite{Es
    gibt nur eines -- sich aufhängen oder die Welt verändern.} Siehe
  \BibRef{piscat.29}{156}.}
\end{BlockQuote}
Die Zielsetzung des Stückes zeigt sich damit letztlich als Kombination zweier
Hauptanliegen: Einerseits sollte \Cite{der Irrsinn der bürgerlichen
  Weltordnung} durch die Konfrontation eines \Quote{konservierten}
Revolutionsidealisten mit der Realität der republikanischen Gegenwart
aufgezeigt oder sogar \Cite{bewiesen} werden.\Footnote{\Abr{Vgl}
  \BibRef{piscat.29}{148}.}
Andererseits galt es vor allem Piscator als vordringlich, dies mit der
Propagierung einer parteisozialistischen Handlungsperspektive zu verbinden.

Daraus erklärt sich die merkwürdige \Quote{Doppelfrontstellung} des
Protagonisten: Als empfindsamer Träger der kritischen Norm des Stückes begehrt
er gegen die \Quote{irrsinnigen} gesellschaftlichen Verhältnisse auf und wendet sich
gegen die
herrschende Politik -- vor allem in Gestalt Wilhelm Kilmans. Als unorganisierter, in
seinem Scheitern vorgeführter  Aktionist ist er dagegen \emph{ex negativo} ein
Beweisfall für die Notwendigkeit der mittelfristig ausgelegten, durchdachten und
disziplinierten Parteiarbeit und muss deshalb in einen
Kontrast zu den entsprechend handelnden \Cite{Genossen} gestellt werden. 

Dass die Hauptfigur in diesen doppelten Gegensatz gesetzt ist, ohne sich
darüber bewusst zu werden, ist für die Logik des Stückes essentiell. 
Die Erkenntnispotentiale sind auf verschiedene Figuren verteilt und es gibt
keine \Quote{messianische} Verkünderfigur.
Die Analyse des Stückes und das Verständnis der impliziten Autornorm ist
dadurch sicher erschwert. Piscator scheint der \Quote{Beweiskraft} der
Komposition nicht vollends vertraut zu haben und setzte nicht ohne Grund seine
zusätzliche explizite Politdidaxe an das Ende des Dramas. 

\HeadingThree{Bemerkungen zur Erstfassung}
 
Während in der \Quote{Piscatorfassung} Karl Thomas \Cite{logischerweise
  zerbricht},\Footnote{\Abr{Ebd}} 
da er die Funktion erfüllen soll, den Irrsinn der Welt \emph{und}
die Abwegigkeit seines Gefühlssozialismus' zu demonstrieren, bleibt in
Tollers Erstfassung die Hauptfigur am Leben. Der fünfte Akt, der in der
Druckfassung die finalen Gefängnisszenen und den Suizid beinhaltet, fehlt hier. 
Das Stück endet mit einer Szene im Irrenhaus, die den Topos vom Irrsinn der
scheinbar normalen Welt aufgreift und umfassend bebildert. Durch diese Szene,
die in der Druckfassung nur wesentlich verkürzt enthalten ist, bleibt der
Protagonist in der Erstfassung bis zuletzt Träger der kritischen Norm des
Stückes und wird mit einer Massendemonstration am Ende noch darin bestärkt. 

Nach meiner Auffassung ist der Massenaufmarsch am Ende der Erstfassung der
stärkste Hinweis darauf, dass das Motiv der massenhaften Volksbewegung in
Tollers dramatischer Gestaltungsphantasie als relativ unreflektierter
\Quote{Restposten} aus der expressionistischen Zeit überlebt hatte. Es
erscheint relativ unmotiviert, dass an dieser
Stelle eine solche Bewegung mit \Cite{Hoch Karl Thomas!}-Rufen
\SourceRef{III}{324} zustande kommt. Es scheint, dass dieser \emph{deus ex
  machina} das Defizit an Handlungsperspektive auffüllen musste, das in der
Erstfassung aufgrund der blass bleibenden Arbeiterfiguren festzustellen ist.

Der Schwerpunkt in Ernst Tollers Erstfassung bestand offensichtlich in
der \emph{Darstellung} des \Cite{Irrsinns der Welt} und weniger in der Propagierung
einer klaren Verhaltensdevise für die \emph{Überwindung} derselben. Da die
handlungsperspektivische Gegenüberstellung mit den Arbeiterfiguren weitgehend
fehlt und außerdem die Verfehlung des Attentats sehr viel schwächer als in der
Druckfassung herausgestellt ist, sind Protagonisten- und Autorperspektive
in der Erstfassung nur schwer zu differenzieren. Die Normalitätsfrage, die
wiederholt das kritische Reflexionsmuster des Protagonisten darstellt, erhält
dabei eine besondere Bedeutung, die sich symbolhaft in Motiv des Gelächters
widerspiegelt.  

Als kommunikatives Symbol kommt das Lachen des Karl Thomas mehrfach vor.
Als Reaktion auf das Gnadenurteil des Präsidenten signalisiert es den
Normverlust, den die praktische Aufhebung der revolutionären
Konfrontationslogik mit sich bringt. Auch in der Hotelszene, als Karl Thomas
seine Arbeit antritt, begleitet das Lachen den vorübergehenden Versuch, die
eigene Norm der Eingliederungsverweigerung aufzugeben.
Daraus könnte gefolgert werden, dass auch das (irrsinnige) Lachen in der
Schlusszene die Umwertung eines Normsystems bedeutet.
Nachdem Karl Thomas erkannt hat, dass der Psychiater Lüdin die Verkehrung von
Irrsinn und Normalität systematisch betreibt, beginnt sein Lachen als Ausdruck
der Wiedererlangung seiner kritischen Norm.\Footnote{\SourceRef{III}{324}:
  \Cite{\DramName{Karl Thomas}: Ich Narr! Jetzt seh ich die Welt wieder
    klar. Ihr habt sie in ein Irrenhaus verwandelt. [..] Hahaha \ldots}}
Als der Professor mit seiner (Un-)Logik auch die
marschierenden Volksmassen für verrückt erklärt, bricht Thomas in ein noch
stärkeres Lachen aus. 
Da sich in Form der Menschenmenge gewissermaßen eine Macht des Faktischen Bahn
bricht, die Karl Thomas in \emph{seiner} Vorstellung von Normalität bestärkt
und die Verdrehungen des Professors in Frage stellt, kann das sich
steigernde Lachen als Signal einer zweifach umgekehrten und damit wieder
richtig gestellten Weltsicht aufgefasst werden: Der Protagonist lacht --
sozusagen mit der faktisch normsetzenden Kraft der Volksmehrheit im Rücken -- die
Interpretationen des Professors in Grund und Boden. 

Diese Interpretation der Verrücktheit als einer vordenkenden, zunächst ins
Abseits gestellten Korrekturkraft, die dann durch massenhafte
Unterstützung normativ \Quote{gerettet} wird, würde das positive Potential des
Protagonisten bedeutend erhöhen. Da die symbolisch aufgefasste Bedeutung des Lachens
allerdings ohne den
Kunstgriff des plötzlich aufmarschierenden Volkes ohne praktische Relevanz
bliebe, scheint es zutreffend, wenn Kirsten Reimers die politische
Aussage der Erstfassung als \Cite{sehr schwammig} bezeichnet und das
Auftreten der Massen in seiner \Cite{politischen Zielsetzung} für
\Cite{unklar} befindet.\Footnote{Siehe \BibRef{reimer.00}{204}.} 

Nach ihrer Einschätzung liegt in der Erstfassung \Cite{der Schwerpunkt des
  Dramas auf dem Konflikt zischen Karl Thomas und Kilman} und die politische
Bandbreite der Druckfassung werde nicht
erreicht. Es handele sich eher um eine \Cite{psychologische Studie der
  Hauptfigur}. Die Darstellung des Karl Thomas lade zwar \Cite{zum Mitleiden
  und Einfühlen} ein, doch er sei zweifellos \Cite{kein eindeutig positiver
  Held}.\Footnote{\Abr{Ebd}, \Page{203-204}.}

Die Behauptung Leydeckers, dass Ernst Toller literarische Imaginationskraft
noch 1927 \Cite{sklavisch} an expressionistische Ideen geknüpft gewesen sei,
lässt sich meines Erachten nur bedingt aufrecht erhalten. Die politischen
Umsturzwünsche, die Karl Thomas leiten, hatte Toller schon lange vorher
relativiert. Bereits in einem Brief von 1920 schrieb er:

\begin{BlockQuote}
Uns fällt die undankbare Aufgabe zu, auch die minimalen Errungenschaften zu
verteidigen, wir können es nicht verantworten, mit großer Geste auf sie zu
verzichten, weil wir \Quote{aufs Ganze} gehen. \SourceRef{V}{49}
\end{BlockQuote}
Tollers Lösung von expressionistischen Revolutionsidealen verlief sicher
\Cite{nicht glatt und eindeutig}.\Footnote{\Abr{Ebd}, \Page{207}.} 
Doch allein die sehr kurze Frist,\Footnote{Laut der Biographie von Richard
  Dove kam Toller am 20. Juli von einem Kurzurlaub nach Berlin zurück und
  erfuhr von den Überarbeitungswünschen Piscators. Am 11. August hatte er die
  endgültige Fassung dann bereits fertig gestellt. \Abr{Vgl} \BibRef{dove.93}{195}.}
in der er die \Cite{Unklarheit} der Erstfassung zu einem vielschichtigen
\Cite{Aufriß einer ganzen Epoche}\Footnote{\BibRef{piscat.29}{147}.} 
umgestaltete, zeigt, wie gut sich Toller künstlerisch auf der Höhe der Zeit
bewegen konnte. Sein Gestaltungsvermögen wurde durch die spannungsreiche
Zusammenarbeit mit Piscator offensichtlich herausgefordert, sich auf
deutlich postexpressionistischem Niveau zu beweisen.  

\HeadingTwo{Zur \Quote{Modernität} von \Cite{Hoppla, wir leben!}}

\Title{Hoppla, wir leben!} wird von der Forschung überwiegend als Beleg für
\Cite{Tollers Abschied von der expressionistischen
  Dramatik}\Footnote{\BibRef{rothst.87}{136}.}  
eingestuft. Insbesondere, wenn man den Einfluss Piscators wie eine
katalytische Kraft auffasst, die Toller in seinem dramatischen
Gestaltungsprozess lediglich zur Präzisierung ohnehin vorhandener Einsichten
herausforderte, verlieren die Einwände, die Interpeten wie Leydecker aufgrund
der Erstfassung gegen eine solche Innovationsdiagnose erheben, an Gewicht. 
Wie beispielsweise Sigurd Rothstein in seiner Analyse von
\Title{Hoppla, wir leben!} und Tollers nächstfrüherem Stück \Title{Hinkemann}
nachweist, ist schon in
Tollers Dramatik von 1923 eine klare Ablösungstendenz vom Expressionismus zu
erkennen.\Footnote{Der Beitrag untersucht beide Dramen parallel als
  Varianten dieser Tendenz. \Abr{Vgl} \BibRef{rothst.87}{136-171}.}  
Weitere starke Belege für diese Einschätzung liefert auch Kirsten
Reimers.\Footnote{\Abr{Vgl} \BibRef{reimer.00}{207}.} 

Auf jeden Fall ergeben sich -- auch ganz ohne Rückgriff auf andere Werke und
Verlautbarungen Tollers -- aus der Analyse der Druckfassung eine Reihe von
Kennzeichen für die Weiterentwicklung der Tollerschen Dramaturgie in Richtung
einer \Quote{moderneren} Konzeption. Dazu gehört wesentlich das Aufbrechen der
unipolaren \Abr{bzw} bipolaren Figurenkonstellation, die noch in der
\Title{Wandlung} oder in \Title{Masse Mensch} vorherrschte. Statt
messianischen Ideenträgern, die vorwiegend monologisch oder verkündend
sprechen, treten in \Title{Hoppla, wir leben!} individuell konturierte Figuren
auf, die tatsächlich \emph{miteinander} reden. Jost Hermand fasst dies so
zusammen: 

\begin{BlockQuote}
In \Title{Hoppla, wir leben!} werden keine
Manifeste erlassen oder Kanzelpredigten ins Publikum geschleudert, sondern
eine Reihe von Grundhaltungen zur Revolution auf höchst antithetische Weise
\Quote{im Dialog} durchexerziert.\Footnote{\BibRef{herman.81}{166}.}
\end{BlockQuote}
Diese Pluralisierung der Figurenperspektiven geht einher mit einer
\Cite{ungewöhnlich hohe[n] Anzahl differenzierter Haupt- und
  Nebenpersonen}\Footnote{\BibRef{klein.68}{135}.}, die zur Darstellung einer
vielschichtigen, in ihrer sozialen Komplexität abgebildeten Welt beitragen. 
Dazu tragen \Cite{realistische
  Partnerdialoge}\Footnote{\BibRef{buetow.75}{336}.} bei.
Den \Cite{visionären Monolog} gibt es allenfalls noch als Ausdruck der
selbstbezogenen Verwirrung einer isolierten Figur wie Karl
Thomas\Footnote{Beispielsweise in der Szene direkt nach dem Attentat
  \SourceRef{III}{95} oder in seiner präsuizidalen Verzweiflungsrede
  \SourceRef{III}{115}.}
und dadurch wird auch diese Sprechweise \Cite{Instrument des
  Realismus}.\Footnote{\BibRef{buetow.75}{337}.}
Die Figurensprache ist außerdem weitgehend von expressionistischem Pathos
befreit und in reiner Prosa gehalten.\Footnote{\Abr{Vgl} \BibRef{klein.68}{146}:
  \Cite{\Title{Hoppla, wir leben!} ist eine  eindeutige
  Absage an das Pathos der Frühphase.}}
 
Insgesamt sind diese Änderungen in der Figurenkonzeption und Dialoggestaltung
Ausdruck eines neuen Subjektverständnisses: Sowohl die Eigenständigkeit und
Selbstbestimmungsfähigkeit der Mitspieler als auch die autonome
Erkenntnisfähigkeit der Rezipienten wird viel höher eingestuft als in den
belehrenden und missionierenden Verkündigungsdramen. Die auf verschiedene
Figuren verteilten Erkenntnispotentiale ermöglichen dem Zuschauer eine
selbständige Erschließung der Dramenaussage -- wenn man von dem Piscatorschen
Epimythion einmal absieht. Jost Hermand schwärmt angesichts dieses
\Quote{Ensemble-Charakters} der Erkenntniskonstruktion gar von
\Cite{dialektischen} Kräften: 

\begin{BlockQuote}
In schroffem Gegensatz zu jedem Ansporn- oder
Einfühlungstheater gibt es hier keinen \Quote{Helden}, sondern bloß
Ideenträger, die sich entweder dialektisch ergänzen oder gedanklich ad
absurdum führen. so gesehen ist \Quote{Hoppla, wir leben!} zwar noch kein
\Quote{episches}, aber doch schon ein \Quote{dialektisches}
Theaterstück.\Footnote{\BibRef{herman.81}{166}. Eine ähnliche Einschätzung
  vetritt auch \BibRef{klein.68}{135}. }
\end{BlockQuote} 
Die schematische Rollenverteilung zwischen Führergestalten und einem
\Quote{massenhaften} oder \Quote{volksmäßigen} Gefolge wird in \Title{Hoppla,
  wir leben!} nicht nur durch eine Vielzahl selbständig agierender Figuren
abgelöst,\Footnote{\Abr{Vgl} \BibRef{kim.98}{285}: \Cite{In \Cite{Hoppla, wir
      leben!} ist nicht die Rolle eines elitären Führers, sondern die
    Teilnahme aller an jedem kleinen Schritt zur Revolution gewünscht. Damit
    wird die hierarchische Beziehung zwischen Führer und Geführten im
    expressionistischen Sinne abgebaut.}} 
sondern am Beispiel des Protagonisten und seinem widersinnigen
\Quote{Erweckungsversuch} sogar als gefährlicher Extremismus entlarvt. Der
Parallelismus des Attentats macht dies deutlich:

\begin{BlockQuote}
Der messianische Anspruch des expressionistischen Helden führt zu einem
selbstgerechten Märtyrertum und dient der Rechtfertigung eines Verbrechens,
die expressionistische Idee eines geistigen Führertums deckt sich mit der
faschistischen Ideologie.\Footnote{\BibRef{reimer.00}{187}.}
\end{BlockQuote}
Damit problematisiert Toller eben jene Ideologie eines prädestinierten
Erweckertums, die er selbst noch in der \Title{Wandlung} zum Leitideal
gemacht und parareligiös erhöht hatte. Jener \Cite{Glaube an die Menschheit},
der \Cite{die Frau} in \Title{Masse Mensch} noch in jenseitsgewandte
Entrückung versetzte, wird auch dann kritisiert, wenn Albert Kroll seinen
politischen \Quote{Diesseitsbezug} klarstellt:

\begin{BlockQuote}
\DramName{Karl Thomas}: Nur der Glaube zählt.\\
\DramName{Albert Kroll}: Wir wollen keine Seligkeit im Himmel. Man muß sehen
lernen und sich dennoch nicht unterbekommen lassen. \SourceRef{III}{65}
\end{BlockQuote}
Die Distanzierung Tollers vom Landauerschen Mythos der
berufenen \Quote{Geistigen} zeigt sich auch in der sarkastischen
Darstellung der \Cite{geistigen Kopfarbeiter}, die neben ihren faschistoiden
Diskussionen über richtige \Cite{Zuchtwahl} \SourceRef{III}{84} ausgerechnet
\Cite{die proletarische Gemeinschaft der Liebe und die Aufgabe der Geistigen}
\SourceRef{III}{86} auf ihre Tagesordnung
setzen.\Footnote{Sigurd Rothstein kommentiert das so: \Cite{Der zentrale
    Befund ist dabei der, daß der messianische Expressionimus engültig
    überholt ist. Seine Schlagwörter [..]
    geraten in die Obhut von \Quote{geistigen Kopfarbeitern}}. Siehe
  \BibRef{rothst.87}{170}.}

Es lässt sich also in der Summe durchaus konstatieren, dass Ernst Tollers
\Cite{Zeitstück} einen relativ modernen Subjektbegriff mit einer
konsensorientierten Handlungsnorm verknüpft und damit gegenüber den
\Quote{messianischen} Stücken seines Frühwerks einen -- im Sinne des hier
verwendeten normativen Modernitätsbegriffs -- erheblichen konzeptionellen
Modernisierungsschritt darstellt. 
Selbst die in \Title{Hoppla, wir leben!} zum Ausdruck gebrachte politische
Handlungsperspektive der \Quote{kleinschrittigen Parteiarbeit} ist nicht so
aufdringlich präsentiert, dass man von einem \Quote{Tendenzstück}
sprechen müsste. Auch hinsichtlich der Verhandelbarkeit von
Wahrheitsansprüchen setzt Toller also eher \Quote{moderne} Akzente, indem er
statt kommunistischer Propaganda eine undogmatische Aufforderung zur
politischen Einmischung formte. 

%%% Local Variables: 
%%% mode: latex
%%% TeX-master: "~/TOLLER/MAIN"
%%% End:






















