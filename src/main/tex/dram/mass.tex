% Examensarbeit: Dramenanalyse "Masse Mensch"

\HeadingOne{Das integre Subjekt als verhinderter Massenführer:\\
  \Cite{Masse Mensch}}{\Cite{Masse Mensch}}

\Title{Masse Mensch} ist das erste von Tollers \Cite{Gefängnisdramen}, in
denen er die Erlebnisse der Münchner Räterevolution verarbeiten
konnte.\Footnote{Vgl. \BibRef{altenh.81}{130}: \Cite{\Quote{Masse Mensch} kann
    durchaus als direkte Antwort auf die politischen Ereignisse von 1918/19
    verstanden werden [..].}}
Es entstand als \Cite{erste Niederschrift} bereits wenige Monate nach der
Inhaftierung, nach Tollers eigenen Angaben im Oktober 1919.\Footnote{Alle
 Zitate und Angaben zur Textgestalt sind den \Title{Gesammelten Werken}, Band II,
 \Page{63-112}), entnommen.} 

Das Werk kann als unmittelbare Aufarbeitung gewisser Komplikationen des
revolutionären
Geschehens verstanden werden, an denen Toller besonders litt: Das Ideal der
friedlichen Menschheitsrevolution hatte in der proletarisch-revolutionären
Praxis Schaden genommen. Die Gewissheit der messianischen Vision der
\Title{Wandlung} musste der Problematisierung einer Revolution weichen, in
der \Cite{der Mensch} tragisch vom Phänomen der \Cite{Masse} überdeckt
ist.\Footnote{Laut Hugh Michael Fritton war \Cite{für Toller nach der
    Zerschlagung der Räterepublik die zentrale Frage nicht, wie diese
    Niederlage hätte verhindert werden können, sondern wie er, Toller, als
    Individuum die politischen Kämpfe hätte bestehen können, ohne schuldig zu
    werden.}  Siehe \BibRef{fritto.86}{196}.}

\HeadingTwo{Inhaltliche Zusammenfassung}

Hintergrund dieses Stücks \Cite{aus der sozialen Revolution des
20. Jahrhunderts} \SourceRef{II}{63} ist ein Arbeiteraufstand, an dessen
Spitze \Cite{die Frau}
als Hauptfigur des Dramas und ihr Gegenspieler, \Cite{der Namenlose}, um die
Führungsrolle konkurrieren. Die im Titel zusammengefassten Aspekte
\Cite{Masse} und \Cite{Mensch} stehen für Handlungsprinzipien, deren
Widerstreit den Konflikt der beiden Hauptfiguren ausmacht. Während \Cite{die Frau}
die zu einende Gesamtheit \Cite{der Menschen} als Träger und Gegenstand einer
friedlichen Revolution versteht, vertritt der \Cite{Namenlose} das Recht der
wütenden \Cite{Masse} auf rücksichtslosen und gewalttätigen Kampf gegen alle
Revolutionsgegner. Im Gegensatz zu ihr unterscheidet er zwischen zum Aufstand
berechtigten \Cite{Massenmenschen} und gegnerischen \Cite{Staatsmenschen}
\SourceRef{II}{107}.

Als eine aus innerer Verantwortung handelnde Vertreterin der
Menschheitsrevolution tritt \Cite{die Frau} zusätzlich in Konflikt zu ihrem
Ehemann, der als Beamter das Staatssystem repräsentiert. Die persönliche
Beziehung kann vor ihrem radikalen humanen Verantwortungsdenken nicht bestehen
und wird vom Gegensatz zwischen Menschlichkeit und Staatssystem überlagert.

Die Protagonistin, die sowohl der gewaltbereiten \Cite{Masse} als auch dem
Ausbeutung und Krieg betreibenden Staat, den geldgierigen Spekulanten und der
Rechtfertigung liefernden Kirche Vergehen \Cite{am Menschen} vorwirft, wird
schließlich von den Revolutionären als Verräterin verstoßen und vom Staat als
\Cite{Führerin} \SourceRef{II}{99} verhaftet.

Im albtraumhaften \Cite{Käfig} eines \Cite{Menschenschauhauses}
\SourceRef{II}{100} verzweifelt
sie fast an der im Kampf um Menschheitsrevolution unvermeidlich erscheinenden
schuldhaften Verstrickung. \Cite{Geheilt} wird sie erst, als sie ihre laute
Anklage gegen Gott und sein \Cite{ungeheuerlich Gesetz der Schuld} wendet
\SourceRef{II}{103}.

In bestärktem Glauben an die Ewigkeit ihrer Vision, an die Reinheit und
Schuldlosigkeit der Menschheitsidee und den Urmythos eines heiligen
\Cite{Werk-Volkes} \SourceRef{II}{110} widersteht die Frau zuletzt noch
einmal allen Anfechtungen, lehnt es ab, gewaltsam befreit zu werden, und wird
schließlich hingerichtet.

\HeadingTwo{Anmerkungen}

Wie die \Title{Wandlung} ist auch \Title{Masse Mensch} explizit in
einen zeitgeschichtlichen Zusammenhang gerückt. Es ist nicht nur als
\Cite{Stück aus der sozialen Revolution des 20. Jahrhunderts} gekennzeichnet,
sondern wird in diesem Kontext auch perspektivisch positioniert, indem es
\Cite{den Proletariern} gewidmet ist und die Angabe der Entstehungszeit
\Cite{im ersten Jahr der deutschen Revolution} die Fortsetzbarkeit des
revolutionären Geschehens suggeriert \SourceRef{II}{64-65}. 

Der Dramenhandlung sind allerdings Geleitverse vorangestellt, die ein
zwiespältiges Licht auf die \Cite{Weltrevolution} werfen. Diese ist
zwar als \Cite{Gebärerin} einer Erneuerung apostrophiert, doch dazu
mischt sich ein Eindruck des Schreckens: 

\begin{BlockQuote}
Weltrevolution. / Gebärerin des neuen Schwingens. / Gebärerin der neuen
Völkerkreise. / Rot leuchtet das Jahrhundert / Blutige Schuldfanale. / Die
Erde kreuzigt sich.
\SourceRef{II}{65}
\end{BlockQuote}
So ist schon hier markiert, dass das Revolutionsmotiv im folgenden Stück
keine reine Quelle heilbringender Erweckung mehr ist, sondern mit tragischer
Schuld konnotiert wird.

Im Zentrum der Dramenhandlung steht \Cite{die Frau}, die auch als
einzige im Figurenverzeichnis einen Eigennamen trägt. Als \Cite{Sonja Irene
  L.} konzipierte Toller sie offensichtlich in Anlehnung an die Aktivistin
Sonja Lerch, die im Münchener Munitionsarbeiterstreik auftrat, deswegen von
ihrem bürgerlichen Mann verlassen wurde und sich später umbrachte. Alle
anderen Figuren sind als Typen gestaltet, die nur durch ihre soziale Rolle
oder Position charakterisiert sind. Ähnlich dem leidenden \Cite{Volk} der
\Title{Wandlung} äußern sich die Arbeitermassen in \Title{Masse
  Mensch} chorartig. Die im Figurenverzeichnis aufgeführten \Cite{Arbeiter}
und \Cite{Arbeiterinnen} treten entweder als \Cite{Massenchöre} oder in Form
einzelner typisierter Stimmen in Erscheinung.\Footnote{Vgl. \BibRef{hohend.67}{128}:
 \Cite{Selbst wo einzelne Gestalten auftreten, fungieren sie als
    Repräsentanten einer Gruppe und ihrer Anschauung.}}

Der revolutionäre autobiographische Hintergrund Tollers ist in dem Stück
sozusagen in \Quote{verschmelzender Raffung} eingeflossen: Sind die
anfänglichen Streikszenen noch aus dem Kontext des (vorrevolutionären) 
pazifistischen Munitionsarbeiterstreiks zu
Kriegszeiten entnommen, so trägt der spätere Umschlag der Handlung in einen
proletarischen Umsturzversuch eher die Züge der räterevolutionären Phase Tollers.   

Das Stück ist eine Reihung von sieben Bildern. Das dritte, fünfte und
siebte Bild sind zwar nach Tollers Regieanweisungen \Cite{in visionärer
  Traumferne} zu denken, jedoch enthalten sie zusammen mit dem ersten Bild die
primäre, eher realistisch gestaltete Handlung des Dramas. Sie alternieren
mit den allegorisch, traumartig verfremdeten Szenen des zweiten, vierten und
sechsten Bildes, die auch explizit als \Cite{Traumbilder} gekennzeichnet sind. 
Diese bilden eine Ebene der dramatischen Gestaltung, in der auch
phantastische Elemente verwendet werden: Figuren tauchen plötzlich auf und
verschwinden wieder, verändern ihr \Cite{Antlitz} oder verkörpern die Schatten
von Getöteten.\Footnote{So erscheinen \AbrPair{z}{B} im sechsten Bild kopflose
  Figuren, die die Frau des Mordes anklagen \SourceRef{II}{100}.} 
Diese Traumbilder sind mit den \Cite{traumfernen} Bildern der
\Title{Wandlung} vergleichbar. Die Tatsache, dass Toller in \Title{Masse Mensch}
auch die realistischeren Szenen in \Cite{visionäre Traumferne} verlegt, weist
auf die insgesamt sehr abstrakte und auf Ideenreflexion zielende Struktur des
Stückes hin, das deswegen schon früh der Rubrik \Cite{Ideendrama} zugerechnet
wurde.\Footnote{So \AbrPair{z}{B} \BibRef{bebend.90}{72}. Bei zeitgenössischen
  Kritikern wurde die ideenlastige Form sehr verschieden
  bewertet. Einerseits \BibRef{droop.22}{30}: \Cite{Das Drama wurzelt ganz im
    Gedanklichen [..] Herzblut tropft aus jeder Szene.}. Dagegen
  \BibRef{bab.26}{43-44}: Die \Cite{bloß verstandesmäßige Erkenntnis} des
  Dramas sei ein Zeichen dichterischer Schwäche und \Cite{jede Szene Szene
    bleibt in Tollers Text schattenhaft, blaß, allgemein.}}

Die Sprache aller Figuren ist expressionistisch reduziert. Parolenhaft
verkürzte Sätze und intensivierende Reihungen geben den Repliken das typisch
expressionistische Pathos.\Footnote{\Abr{Vgl} dazu auch \BibRef{reimer.00}{53}.}

\HeadingTwo{Bekenntnis zur Revolution aus Verantwortung}

Die Ausgangssituation des Stückes ist der Vorabend eines Arbeitskampfes, dem sich
\Cite{die Frau} als Wortführerin angeschlossen hat. Das erste Bild zeigt sie
in einer \Cite{Arbeiterschänke} \SourceRef{II}{67}, wo man von morgiger
\Cite{Entscheidung}
spricht. Die Frau sieht sich \Cite{bereit}, die Führung zu übernehmen. Sie
\Cite{sehnt} sich danach, ihr \Cite{Herzblut} durch die Kraft der Rede in die
Taten der Arbeiter einfließen zu lassen. Ihr Antrieb, zum Streik aufzurufen,
wird als innerer Zwang vorgestellt. Sie empfindet \Cite{Scham und Qual}
angesichts des Krieges und der Propaganda, die seine Fortführung fordert.  Die
Notwendigkeit, dem ein Ende zu machen, ist ihr so offensichtlich, dass sie sich
geradezu als Stimme eines natürlichen Aufschreis versteht:

\begin{BlockQuote}
Bin \emph{ich} es noch, die Streik verkünden wird? Mensch ruft Streik, Natur
ruft Streik! Mir ists, als bellts der Hund, der an mir aufspringt, [..].
\SourceRef{II}{67}
\end{BlockQuote}
Ihre innere Sicherheit wird als starkes \Cite{Wissen} suggeriert, das sie
an \Cite{Armeen der Menschheit} glauben lässt, die \Cite{das Friedenswerk zum
unsichtbaren Dome türmen} werden.  Die Frau artikuliert sich in dieser
Eingangssequenz wie ein Messias der Menschheitsverbrüderung kurz vor dem
Verkündungstag. Die Vision der \Cite{Menschheitskathedrale}, wie sie in
der \Title{Wandlung} entworfen wurde,\Footnote{Vgl. \SourceRef{II}{51} und
  die Erörterung auf Seite \pageref{mkath} der vorliegenden Arbeit.} 
ist hier voll präsent.

Doch im Gegensatz zur \Title{Wandlung} ist der rhetorische Akt
nicht mehr der übertragene Ausdruck einer Verschmelzung von Mensch und
Menschheit in Wille und Tat, sondern nur eine Bekundung von Wunsch und Vision
der Protagonistin vor sich selbst. Sie \emph{empfindet sich} als Zuständige
für die Belange der Menschheitsverbrüderung, während Friedrich zu seiner
Mission explizit \emph{durch höhere Kräfte ausgesandt} wurde.   

Im Gegensatz zur \Title{Wandlung} gibt es in \Title{Masse Mensch}
ernst zu nehmende Gegner der Revolution, und die Frau ist auf besondere Weise
mit diesen verbunden. Ihr Ehemann, der im ersten Bild in die Szenerie gleichsam
hereinplatzt, vertritt als überzeugter Beamter ein bürgerlich-staatstragendes
Wertesystem und will ihre Vorhaben unterbinden. Sie besteht darauf, dass
\Cite{Not aus Menschsein} \SourceRef{II}{70} sie dazu treibe. Staatsvertreter
hätten die einfachen Menschen \Cite{erniedrigt} und sich damit \Cite{selbst
  geschändet} \SourceRef{II}{70-71}.
Aus ihrer Perspektive würde die friedliche
Verbrüderung aller Menschen auch die Erlösung der Herrschenden bedeuten. Damit
vertritt sie das umfassende Menschheitsideal, das in der \Title{Wandlung} noch als
Grundlage einer friedlichen Revolution taugte. In \Title{Masse Mensch}
steht die Protagonistin mit diesen Vorstellungen allerdings ziemlich
allein da.

Tatsächlich ist die Frau noch auf der Suche nach einer sinngebenden
Gemeinschaft. In der Rolle der Streikführerin strebt sie eine Identität an,
die ihr soziales Gewissen und ihren inneren Drang nach Frieden zur Geltung
kommen lässt. Sie versucht sich von ihrer bürgerlichen Herkunft, die
durch ihren Mann verkörpert wird, abzugrenzen. In dieser Isoliertheit gleicht
sie dem verwirrten Friedrich zu Beginn der \Title{Wandlung}, der sich von
seiner Familie abzulösen suchte:

\begin{BlockQuote}
Die Ankündigung einer gewaltsamen politisch-gesellschaftlich entscheidenden
Veränderung (Krieg, Streik) wird in beiden Expositionen mit der
Neuorientierung, ja Identitätsfrage eines Individuums verknüpft, das dabei
ist, seine Zugehörigkeit zu einer Gruppe oder - nach Luhmann - zu einem
sozialen System zu wechseln.\Footnote{\BibRef{schrei.97}{123}.}
\end{BlockQuote}
Die Protagonistin von \Title{Masse Mensch} steht zwischen den
Zugehörigkeiten des Staates und der Arbeiterklasse. Einerseits treibt sie ihr
Menschheitsideal dazu, dem Staat \Cite{die Maske von der Mörderfratze zu
  reißen} \SourceRef{II}{72}
und sich den Arbeitermassen, die ihr Mann als \Cite{innren Feind}
\SourceRef{II}{71} bezeichnet, 
anzuschließen. Andererseits begehrt sie trotz ihrer selbstlosen Idee
von Menschenliebe das nächtliche Beisammensein mit ihrem Mann und geht am
Ende des ersten Bildes auch mit ihm. Sie empfindet das offensichtlich selbst als
Widerspruch und nennt sich \Cite{schamlos in meinem Blut} \SourceRef{II}{73}.

Die Frau ist somit schon aus dieser Disposition heraus als eine
(tragisch) verhinderte Menschheitserlöserin konzipiert. In ihrem Wunsch,
soziale Zugehörigkeit mit einer überindividuellen Mission zu
verbinden, trägt sie einen zentralen Zug vieler dramatischer Figuren im Werk
Ernst Tollers, die in ihrer messianischen Ausprägung mit dem Typus des
intellektuellen Heilbringers zusammenfallen. Die Gespaltenheit der potentiellen
Führerfigur deutet hier schon die spezifische \Cite{Tragik} von
\Title{Masse Mensch} an. 

\HeadingTwo{Kritik der kapitalistischen Pervertierung von Liebe und Güte}

Das Gegensatzpaar aus universeller \Cite{brüderlicher} Liebe und persönlicher
Liebe mit geschlechtlichem Verlangen, an dem die Befangenheit der Frau im
ersten Bild vorgeführt wurde, wird im zweiten Bild zur Grundlage einer kritischen
Offensive gegen den Zynismus des Börsenkapitalismus ausgebaut. Im traumartigen
Szenario einer \Cite{Effektenbörse} \SourceRef{II}{73f.} wird das Geschäftsmodell der
\Cite{Kriegserholungsheim A.G.} \SourceRef{II}{76} entwickelt, das den Liebesmangel der
kämpfenden Truppe in Form von staatlichen Bordells gewinnträchtig kompensieren
soll. Der Liebesmangel war als Ursache für schleppende Kriegserfolge festgestellt
worden und hatte deswegen für Unzufriedenheit bei den kriegsgewinnlerischen
Bankiers gesorgt.

Als die Frau in die Szene einbricht und den Herren den Hinweis
gibt, dass es um \Cite{Menschen} gehe \SourceRef{II}{79}, wird das von den
Spekulanten als
Meldung eines unglücklichen Zwischenfalls umgedeutet. Sie nehmen es zum
Anlass, ein \Cite{Wohltätigkeitsfest} zu feiern, das als grotesker
\Cite{Tanz ums Börsenpult} stattfindet und zu dem die Bankiers
sich \Cite{fünfhundert raffinierte Mädchen} spendieren \SourceRef{II}{79-80}.

Die Funktionalisierung von Liebe als Hilfsmittel für Krieg und
Ausbeutung wird so durch die eigennützige Zelebrierung von \Cite{Wohltätigkeit} 
ergänzt. Die Darstellung kritisiert die Aktivitäten des Finanzkapitalismus als
menschenverachtende, verlogene Machenschaften. Sie dient der Illustration des
Befundes, dass der moderne Staat eine Machthülse für inhumane Verfahren der
ökonomischen Ausbeutung sei. Die Darstellung impliziert eine erhebliche Zuspitzung
der Gesellschaftskritik der \Title{Wandlung}, wo nur am Rande die Rede
gewesen ist von \Cite{Reichen, die prassen} \SourceRef{II}{38} und am Krieg
verdienen.

Handlungslogisch fungiert die Szene als Kommentar, der -- wie so oft in
Tollers früher Dramatik -- als \Cite{Traumszene} deklariert ist und damit quasi
neben die eigentliche Dramenhandlung treten kann. Es wird unterstrichen, dass
das System des Staates durch den Krieg und die damit verbundenen
Machenschaften diskreditiert ist und als sinngebender Kontext für die
Protagonistin ausscheidet. Außerdem wird die Inkonsequenz der Frau gegenüber
ihrem Mann anhand des zwiespältigen Liebesbegriffs motivisch aufgenommen
und mit menschenfeindlicher Praxis verknüpft. Das \Cite{schamlose} Verhalten
der Frau bekommt so eine zusätzlich schuldbehaftete Konnotation. 

Die Kontrastierung von Erotik und idealisierter Brüderlichkeit ist ein
wiederkehrendes Thema der Tollerschen Dramatik. Das Verhältnis der beiden
Kräfte wird meist zu einem polaren Gegensatz ausgestaltet, in dem die
persönliche oder \Quote{nur} körperliche Liebe als Bedrohung des
verantwortungsvollen, mitmenschlichen Handelns aufgefasst wird. Am
Beispiel des Kriegsbordells soll ofensichtlich gezeigt werden, dass im
Extremfall das Sexuelle sogar als geplanter funktionaler Ersatz für
brüderliche Liebe herzuhalten hat.

Erstmals tritt in diesem Bild auch die Figur des \Cite{Begleiters} auf, der der
Frau unterstützend zur Seite steht, als sie die Bankiers
vorübergehend in ihrer Selbstzufriedenheit stört. Der \Cite{Begleiter}
kommt nur in den Traumszenen vor und übernimmt dabei stets eine leitende, aber
ambivalent gezeichnete Rolle gegenüber der
  Frau.\Footnote{\Abr{Vgl} die Regieanweisung für seinen ersten Auftritt:
  \Cite{Sein Gesicht: ein Verwobensein von Zügen des Todes und Zügen
    angespanntesten Lebens. Er führt die Frau.} \SourceRef{II}{78}. 
  \Abr{Vgl} dazu auch \BibRef{reimer.00}{52}.}

\HeadingTwo{Menschheitserlösung oder Massenaufruhr}

Die Tendenz der Frau zur sozialen Ablösung und Neuzuordnung wird im
Hauptkonflikt des Dramas zusätzlich relevant. Außer dem Identitätskonflikt
einer bürgerlichen Menschenfreundin erwächst ihr nämlich auch ein Problem aus
ihrem Bemühen, als Arbeiterführerin Akzeptanz zu finden. Sie hat zwar im
dritten Bild Gelegenheit, ihre Vision eines friedlichen Streiks vor den
Arbeitermassen zu entwerfen und diesen als konsequenten Kriegsboykott zu
motivieren, doch ihre Führungsrolle wird durch einen Konkurrenten
angefochten. \Cite{Der Namenlose} beansprucht die alleinige Wortführerschaft
und die Repräsentation des Willens der \Cite{Masse} für sich \SourceRef{II}{84f.}. 

Die Position der Frau beruht auf dem Postulat der Solidarität aller
\Cite{Schwachen} \SourceRef{II}{82} in der Verweigerung ihrer Dienste gegenüber dem
Staatssystem. Ähnlich wie in der \Title{Wandlung} wird hier das Volk als die
Gesamtheit der Leidenden angesprochen, denen ein verständnisvoller und
mitfühlender Vordenker die Verbrüderungsidee nahe bringen muss. 

Die Revolution wird aber nicht mehr als automatische Folge rhetorisch-mentaler
Vorgänge konstruiert. Stattdessen wird die politische Einflussnahme der
Arbeitermassen als eine Zielsetzung behandelt, für die geeignete Mittel
einzusetzen sind. Die Verbrüderungsidee wird mit der Strategie des friedlichen
Streiks verknüpft und bekommt darin ihre konkrete Handlungsperspektive.
Die Revolution als bloße \Cite{Erweckung} aller Menschen scheint nicht mehr vertretbar,
der Staat und seine Organe sind als heftige Gegner innenpolitischer Veränderung
dargestellt. Die friedliche Verweigerung stellt somit die Adaption des
Verbrüderungsgedankens an reale Kräfteverhältnisse und Widerstände dar. 

Diesem friedlichen Konzept, das durch Gewaltvermeidung den \Cite{Menschen} im
politischen Gegner respektieren und sich damit ein Verbrüderungspotential 
erhalten will, steht das kämpferische Revolutionsverständnis des
\Cite{Namenlosen} entgegen. Er will mit Gewalt einen \Cite{Krieg} gegen Staat
und Kapital ausrufen, der die \Cite{Fundamente} der Ausbeutung
beseitige \SourceRef{II}{84,85}.
Wirkungslose \Cite{schöne Reden} \SourceRef{II}{85} sind seine Sache nicht. Statt
einer langfristigen Verbrüderung mit den Herrschenden, betont er den Kampf der
\Cite{Masse} als Angelegenheit von \Cite{Schicksal}, dem man durch seine soziale
Herkunft ausgesetzt sei. Das gewaltige rhetorische Aufbrausen des Redners
überzeugt die anwesende \Cite{Masse}, seine Beschwörung von \Cite{Kraft} und
\Cite{Tat} euphorisiert die Zuhörer und lässt die Frau verstummen. Sie ist
gefühlsmäßig überwältigt und ächzt schließlich zustimmend: 

\begin{BlockQuote}
\DramName{Die Frau}: Du \ldots bist \ldots Masse / Du \ldots bist \ldots
Recht.\\
\DramName{Der Namenlose}: Die Brückenpfosten eingerammt, Genossen! Wer in den Weg
sich stellt, wird überrannt. Masse ist Tat!\\
\DramName{Masse im Saal} (\DramHint{hinaus stürmend}): Tat!!!
\SourceRef{II}{86}\Footnote{Die formelhaft verkürzten
  Nominalpradikationen sind typisch für
  den Sprachstil der meisten Dialoge des Dramas. Der Streit der beiden
  Ideenträger um die Gefolgschaft der Aufständischen manifestiert sich
  geradezu in dem verbalen Kampf um die endgültige
  Prädikation dessen, was \Cite{Masse} ist.}
\end{BlockQuote}
Das Schema der revolutionären Führung durch erfolgreiche Volksreden, das in
der \EmphCite{Wandlung} aufgebaut wurde, ist hier beibehalten. Die
wortgewaltigere Position setzt sich durch, und
der Gegensatz zwischen \Cite{Mensch} und \Cite{Masse} ist mit diesem Rededuell bereits
entschieden. Aus der Perspektive der Autornorm mutieren die Arbeiter vom
erlösbaren \Cite{Volk} zur haltlosen \Cite{Masse}, indem sie sich dem falschen
Führer und dessen Aufforderung zur Gewalttätigkeit anschließen und vom Konzept
des friedlichen Streiks aus solidarischer Menschenliebe abweichen. 

Die Frau wird aufgrund ihrer Position der Schwäche und mit Verweis auf ihre
bürgerliche Herkunft als ungeeignete Führerin abgelehnt. Ihr Menschheitsideal
findet also weder im Bereich des Staatssystems noch im selbstgewählten Umfeld der
rebellierenden Arbeiter ernsthaften Anklang. Sie ist eine verhinderte Führerin
ohne Gefolge, eine Prophetin ohne Publikum. Im Folgenden wird sie zur
ohnmächtigen Begleiterin der Kampfhandlungen degradiert, gegen die sie immer
wieder Einspruch erhebt, ohne jedoch ihren Überzeugungen Geltung verschaffen
zu können. Sie steht in der Logik des Stückes sozusagen für die ständig
präsente, aber immer wieder missachtete Option der \Cite{Menschlichkeit}.
 
Die \Cite{Masse} ist im weiteren Verlauf nur noch ausführendes Organ der
Anweisungen des \Cite{Namenlosen}. Führer und Geführte werden sogar explizit
gleichgesetzt: Mehrmals heißt es, der \Cite{Namenlose} \emph{sei} \Cite{Masse}
\SourceRef{II}{85}
und umgekehrt gilt auch die Parole: \Cite{Masse ist namenlos}
\SourceRef{II}{88}.
Das Gemeinschaftsideal der Verschmelzung von Subjekt und sozialem
Umfeld, das im Menschheitskult der \Title{Wandlung} als positive Norm
generalisiert und realisiert wurde, ist hier als negativer Herdentrieb, als
blindes Massenphänomen gezeichnet. 
Die Passivität der Zuhörenden, ihre auf widerspiegelndes Echo
beschränkten Kommunikationsanteile und die schematische Rollenverteilung
zwischen vordenkendem Rhetor und leidender, kanalisierbarer Menge sind in
beiden Fällen gleich. Entscheidend für die unterschiedliche Wertung, die sich
jeweils aus der Autornorm ergibt, ist die Zielrichtung der
Bewegung: Menschliche Verbrüderung ist das befürwortete Ziel, gewaltsamer
Kampf die abgelehnte Verirrung. Tollers implizite Kritik der
\Cite{Masse} stößt sich also nicht an ihrer Lenkbarkeit an sich, sondern
beklagt lediglich ihre Fehlleitung. 

Es ist in der Forschung vielfach erörtert worden, ob die Position der Frau mit
der Autornorm gleichgesetzt werden könne. Mit dem Hinweis, dass Toller
an anderer Stelle revolutionären Gewalteinsatz bedingt befürwortet hat, wurde
eine völlige Identifizierung der Positionen abgelehnt. Immerhin steht die
Protagonistin aber für ein Prinzip, das in der \Title{Wandlung} noch als
heilbringende Autornorm entwickelt worden ist. 
Dass Toller in \Title{Masse Mensch} den Misserfolg der
ganzheitlichen Verbrüderungsidee als tragisches Scheitern gestaltete, kann als
wehmütige Verarbeitung seiner eigenen Revolutionserfahrungen verstanden
werden. Es bleibt aber zu fragen, ob und wie er seine erweckungsrevolutionäre 
Position damit relativiert hat.

\HeadingTwo{Revolution als System des Todes}

Mit dem Triumph des \Cite{Namenlosen} ist die Revolution zum System der Gewalt
geworden. Die Revolutionäre agieren mit Rachsucht. sie vergelten Tod durch
neuen Tod. In einem Traumbild lässt der Namenlose Arbeiter und todgeweihte
Geiseln den \Cite{Totentanz der Zeit} \SourceRef{II}{88f.} miteinander
tanzen. Damit nähert sich
die Darstellung dem kriegsbezogenen Todesmotiv der \Title{Wandlung} an,
das dort als \Cite{Feind des Geistes} den Gegenpol zur geistig verbrüdernden
Erweckung bildete.
Sowohl den Staatsverfechtern als auch den Kämpfern der Masse ist die
\Cite{Sache}, für die sie streiten, \Cite{heilig} \SourceRef{II}{109-110}. Aus dieser
Perspektive gilt es als heldenhaft, für seine Sache zu sterben. Der Tod erhält
eine Heilsbedeutung. Die staatliche Logik des
Heldentodes wird von den Revolutionären also lediglich mit anderen Vorzeichen
versehen.

Seitens der Frau wird dieser Todes- und Tötungsbereitschaft das Prinzip der
Vergebung
entgegengesetzt. Sie beharrt auf dem Verbrüderungsideal, nach dem jeder
Gewalttäter sich auch gegen sich selbst vergehe.\Footnote{Vgl. \SourceRef{II}{92}: 
\Cite{\DramName{Die Frau}: Der heute an der Mauer steht. Mensch [..] Erkenn
  dich doch: Das bist du.}.}
Allerdings zieht sie Verdacht auf sich, weil sie
die Vergebung ausgerechnet für ihren gefangenen Mann einfordert. 

In der revolutionären Kommandozentrale setzt sie ihre Einwände fort, ohne
damit Einfluss nehmen zu können. Während der Namenlose zwischen staatlichem
Krieg und Arbeiterkampf unterscheidet,\Footnote{Vgl. \SourceRef{II}{94}:
  \Cite{\DramName{Der Namenlose}: Im Kriege gestern warn wir Sklaven. [..] Im
    Kriege heute sind wir Freie.}} 
sieht sie stets nur Menschen gegen Menschen kämpfen und fühlt sich schuldig: 

\begin{BlockQuote}
Einhaltet Kampfverstörte! [..] Masse soll Volk in Liebe sein. / Masse soll
Gemeinschaft sein. / Gemeinschaft ist nicht Rache. [..] Mensch, der sich
rächt, zerbricht. [..] Zerbrecht das System! / Du aber willst die Menschen
zerbrechen. [..]  Menschen ewige Brüder \ldots \SourceRef{II}{95-96}
\end{BlockQuote}
Die weitere Streitrede bringt keine neuen Aspekte. Während man von der
schrittweisen Niederlage der kämpfenden Arbeiter erfährt, tauschen die Frau
und der Namenlose ihre einschlägigen Argumente in immer neuen
Variationen aus. Sie will Erlösung und Frieden, er den gewaltsamen Sieg. Die
Frau gerät dabei weiter in einen Gestus der Unterlegenheit. Sie macht sich zum
\Cite{bittend Kind} \SourceRef{II}{97}, das um ein Ende der
Rachsucht bettelt. Schließlich will der Namenlose sie
sogar verhaften lassen, doch das geht im Chaos der Niederlage unter. Während
die Arbeiter trotzig die Internationale singen, wird die Frau vom staatlichen
Militär \Cite{als Führerin} \SourceRef{II}{99} verhaftet.

\HeadingTwo{Schuld und Überwindung}

Der Ausgang der Revolte verdeutlicht die isolierte Stellung der
Protagonistin. Sie, die die Verbrüderung mit allen wollte, ist nun von allen
Seiten ins Abseits gestellt. Die Arbeiter haben sie verstoßen, die
Staatsmacht hat sie als Aufrührerin gefangen genommen und auch sie selbst
fühlt sich schuldig. Im sechsten und siebten Bild beherrscht die
Schuldthematik, die aus der Perspektive der Frau mehrfach angedeutet wurde,
das Geschehen.\Footnote{\Abr{Vgl} \BibRef{buetow.75}{106}: 
  \Cite{Die Prämisse von der Unvereinbarkeit von Gewalt und Gewissen führt in
    die personale Schuld- und Erlösungsproblematik Sonjas.}}

In der Welt der \Cite{Wandlung} gab es keine Schuld, sondern nur vorläufige
Irrwege. Das Menschheitsideal hatte dort noch die Verbindlichkeit einer
Vorsehung, die durch den Protagonisten nur erkannt und umgesetzt werden
musste. Im Gegensatz dazu ist das Menschheitsideal der Frau ein subjektives
Gefühl, das ihr Gewissen ihr nahelegt und wodurch Außen- und Innenwelt in
Widerspruch geraten. Das Innere der Protagonistin ist nicht
mehr -- wie in der \Title{Wandlung} -- als Widerspiegelung der Menschheit
gestaltet, sondern bildet den Vexierspiegel einer \Quote{eigentlichen}
Idealität, die sich mangels Ausstrahlungserfolgen gleichsam selbst verbrennt.   
Da der empathische Automatismus des \Cite{Ich weiß um euch} nicht mehr greift.
bleiben die Erweckungsversuche auf sich selbst zurückgeworfen und lassen die
Gescheiterte am Mangel ihrer Redegewalt leiden. Das ohnmächtige Verzagen vor
der Stimmgewalt des Namenlosen wird als schuldhaftes Versäumnis ausgedeutet:
In der Traumszenerie des \Cite{Menschenschauhauses} beschuldigen die
kopflosen Toten des Aufstandes sie des Mordes durch Schweigen. Bei allen
entscheidenden Momenten des gewaltsamen Kampfes habe sie 
fahrlässig geschwiegen \SourceRef{II}{100}.

Die weitere Erörterung dieser Vorwürfe, die von dem Ideal des
eigentlich allmächtigen Wortes der wahren Führerperson ausgehen, führt zu der Frage
nach der Zwangsläufigkeit solchen Schuldigwerdens. Da die \Cite{Masse} in ihrem
Verhalten als zwanghaft und damit schuldunfähig aufgefasst wird, leitet die
Schuldfrage auf höhere Instanzen über. Das \Cite{Menschenschauhaus}, in dem 
die Frau -- von \Cite{dem Begleiter} in Gestalt eines \Cite{Wärters} \SourceRef{II}{99}
bewacht -- in einem Käfig sitzen muss, wird zum Ort einer seherischen Redefolge, in der
sie gewissermaßen mit der \Quote{Erkenntnishilfe} des orakelnden
\Cite{Begleiters} die Schuldfragen erörtert. Schließlich weist sie die
individuelle Verantwortung von sich und klagt stattdessen Gott und dessen
\Cite{ungeheuerlich Gesetz der Schuld} \SourceRef{II}{103} an.
Indem sie Gott \Cite{überwindet} und sich quasi selbst von der
unmittelbaren Pflicht erfolgreicher Volksführung freispricht, wird die Frau
\Cite{geheilt}.\Footnote{Siehe \SourceRef{II}{103}:
  Die \Quote{Heilung} wird von der Wärtergestalt festgestellt und führt zur
  Entlassung der Frau aus ihrem \Cite{Käfig}. Das \Cite{Menschenschauhaus} 
  ist somit eine Art \Quote{Läuterungsanstalt}, die die Leitfunktion der
  \Cite{Begleiter}-Figur institutionell verstärkt.}

Eine Ambivalenz von \Cite{schuldloser} Schuld \SourceRef{II}{105} und
\Cite{unfreier} Freiheit \SourceRef{II}{103} bleibt aber bestehen, bis die
Protagonistin in der letzten
Szene ihre messianische Endabrechnung mit allen Beteiligten des
Kampfgeschehens vollziehen kann.

\HeadingTwo{Der messianische Tod der Führerin ohne Volk}

Die \Quote{Heilung} der Protagonistin im \Cite{Menschenschauhaus} hat in \Title{Masse
Mensch} eine ähnliche Funktion, wie die \Cite{Wandlung} Friedrichs in der
\Cite{großen Fabrik}. Während Friedrich seine fehlerhafte Existenz durch
\Quote{Selbstkreuzigung} gleichsam auslöscht, überwindet die Frau bei ihrer
Heilung das Prinzip der Selbstbezichtigung. Was bei Friedrich der Beginn
einer Erweckerkarriere war, ist für sie eine Reinigung des Gewissens,
die sie befähigt, ihre \Cite{schuldhaften} Verflechtungen mit den Systemen der
Gewalt aufzulösen. Die derart befreite Führungsgestalt bedeutet allerdings
keine durchgreifende Veränderung für Tollers dramatische Subjektkonstitution:
Das Erlösungsziel bleibt erhalten, nur die \Quote{Zielgruppe} wird (vorerst)
für unreif erklärt.\Footnote{In ihrer Attribuierung der \Cite{Masse} als
  \Cite{zerstampfter Acker} und \Cite{verschüttet Volk} \SourceRef{II}{107}
  wird deutlich, dass die Frau auch zuletzt noch von einem (verdeckten)
  Läuterungspotential der \Cite{Massenmenschen}
  ausgeht und diese als hilflose Opfer einer Fehlleitung begreift.}
  
So verabschiedet die Frau ihren Mann als \Cite{Bruder}, nachdem sie ihm noch
einmal ihre Kritik am unterdrückerischen Staat entgegen geschleudert
hat. Sie habe \Cite{überwunden} \SourceRef{II}{106}, was sie mit ihm persönlich
verband.\Footnote{Nach Einschätzung von Walter Sokel vollzieht die Frau damit
  damit eine klare Gewichtung des polarisierten Liebesmotivs: \Cite{Die Liebe als
    Agape, die sie zu den leidenden Menschen zieht, siegt über die Liebe als
    Eros, als selbstisch geschlechtliches Glück.} Siehe
  \BibRef{sokel.81}{30}.}
Nachdem die Frau damit ihre persönliche Verbindung zum Umfeld des Staatssystems
abgebrochen hat, tritt sie in eine letzte Redeschlacht mit dem Namenlosen, in
der sie sich auch von seinem System der Gewalt vollends loslöst. Mit starker
Stimme weist sie sein Angebot, sie gewaltsam aus der Haft zu befreien, von
sich, fasst noch einmal ihre Kritik des revolutionären \Cite{Krieges}
zusammen und bringt in einer Reihung die Quintessenz ihrer
Überzeugungen auf den Punkt:

\begin{BlockQuote}
Nur sich selbst opfern darf der Täter\\
Höre: kein Mensch darf Menschen töten / Um einer Sache willen.\\
Unheilig jede Sache, dies verlangt,\\
Wer Menschenblut um seinetwillen fordert, ist Moloch:\\
Gott war Moloch. Staat war Moloch. Masse war Moloch. \SourceRef{II}{110}
\end{BlockQuote}
Das Gewalt verweigernde Selbstopfer ist ihr somit das einzig legitime Mittel gegen
die großen Mächte der \Cite{heiligen} Sachen, für die über Leichen gegangen
wird. Auf die Frage, was ihr \Cite{heilig} sei,
schlägt sie den Bogen zum säkularen Urmythos der Menschengemeinschaft, die als
\Cite{Werk-Volk} eine ewige metaphysische Institution darstelle
\SourceRef{II}{110}. 
Da die Welt für diese \Cite{Wahrheiten} (noch) nicht reif erachtet wird, rückt das
Jenseits in den Rang eines Wartestands der Erleuchteten: Durch ihren Tod werde
sie einst \Cite{reine, schuldlose, Menschheit} sein. 

Der Glaube der Frau an den säkularen Menschheitsmythos wird abgesichert durch die
Ablehnung Gottes und kirchlicher Sündenthesen. Einen Pfarrer, der die ewigen
Komplementärkräfte bösen Menschseins und göttlicher Erlösung postuliert, weist sie
entsprechend in die Schranken. So nimmt sie ihren Abschied von einer Welt,
die aus ihrer Sicht \Cite{den Menschen} \SourceRef{II}{111} verkennt und ihr nur die
\Quote{Selbstopferung} ermöglicht. Damit hat sie ihren letztlich doch
messianischen Ausweg verkündet. 

Ihre Hinrichtung ist der Schlusspunkt eines Dramas, das seine 
Gemeinschaftsvision in nebulöse Jenseitigkeit verlagert, weil es die
Sphäre der Revolution an die Irrungen der zwischenmenschlichen Gewalttätigkeit
abgetreten hat. Die entrückte Gewissheit ewigen \Cite{Werk-Volkes} ist hier der
pathetische Nachhall einer Messiasfigur, die ohne diesseitige
Glaubensgemeinschaft auszukommen hat. Die Wahrung des eindeutigen Standpunktes
des integren, \AbrPair{d}{h} pazifistischen Subjekts geht auf Kosten jeglichen 
Gemeinschaftsbezugs und letztlich des eigenen Lebens. Das Motiv der
diesseitigen Verschmelzung von Subjekt und
Gemeinschaft, das in der \Title{Wandlung} noch dominierte, ist hier ersetzt durch
eine singuläre Vision und die Autonomie der Selbstopferung. Der Gestus des
Messianischen bleibt erhalten, doch er ist nur der Verweis auf das
eigentlich Gute, das in einem schuldverhafteten Diesseits (fast) keine Chance hat:
Eine Spur anrührender Nachwirkung findet der Tod der Protagonistin in der Scham zweier 
Frauen, die ihr Wühlen in den Sachen der Verurteilten beim Ertönen der
Todessalve jäh unterbrechen: \Cite{Schwester, warum tun wir
  das?} \SourceRef{II}{112} 
fragen sie sich bestürzt, und man kann sich hier nur ein leises \Cite{Wir sind
  doch Menschen.} dazu denken, das in der \Title{Wandlung} noch
revolutionärer Aufschrei war.\Footnote{Klaus Bebendorf sieht in diesem
  Dramenschluss eine ernüchterte Kommentarlosigkeit: \Cite{Kein weiser Rat
    des Dichters, kein kluges Fazit, nur eben Ratlosigkeit steht am Schluß des
    Werks. Tollers hinzugewonnene politische Erfahrung ist es, die sich als
    Antithese in den von seiner Phantasie vorgezeichneten Lauf der Historie
    zum Menschheitsparadies mischt.} Siehe \BibRef{bebend.90}{82}.}

\HeadingTwo{Zur diskursiven Problematik des Dramas}

\HeadingThree{Das Dilemma der Verbrüderungsidee}

Was in \Cite{Masse Mensch} als verhängnisvolles Aufeinanderprallen
gegensätzlicher Revolutionsvorstellungen dargestellt und als Schuldproblematik
verhandelt wird, ist zum Großteil den inneren Widersprüchen des messianischen
Menschheitskultes geschuldet, der sein soziales \Quote{Modell} als abstrakte
Verbrüderungsszenerie imaginiert und ein passives \Cite{Volk} durch wundersam
erleuchtete Messiasgestalten \Quote{erwecken} lassen will. In der
ideologischen Projektion selbstloser Liebe und konfliktloser Gemeinschaft auf
die Fiktion einer Einheit und Ganzheit \Quote{aller Menschen} ist nicht nur
\Quote{der Wunsch der Vater des Gedanken}. Vielmehr ist die bereits im vorigen
Kapitel ausführlich dargestellte abstrakte Generalisierung der
Gemeinschaftsidee ein pseudoreligiöser Kunstgriff, der notwendig von den
Bestimmungen seines Gegenstandes absehen muss. Abstraktionen in der
Wirklichkeit geltend zu machen, bedeutet aber letztlich, \Cite{Wirklichkeit zu
  zerstören}.\Footnote{Es mag befremdlich
  anmuten, das dieses Hegelsche Diktum hier so lässig auf ein Ideologem des
  frühen 20. Jahrhunderts angewendet wird, doch in der Tat scheint es mir ein
  sehr passendes allgemeines Argument für das begriffliche Dilemma der
  \Quote{O Mensch}-Expressionisten zu sein.} 
Da das Zerstören der Wirklichkeiten, die der Abstraktion \Cite{Menschheit} 
-- ebenso wie jedem anderen abstrakt absolut gesetzten Gemeinschaftskonzept --
real entgegenstehen, in der Praxis nur mit Gewalt und unter Ausblendung der
Entscheidungs- und Selbstbestimmungsfreiheit der beteiligten Subjekte zu
machen ist, muss die absolute \Quote{Menschheitsidee} an der Realität
notwendig scheitern.

Wenn die Verfechter des Humansozialismus also in den praktischen Wettstreit um
die Gunst des \Cite{Volkes} eintreten, müssen sie entweder auf
\emph{argumentative Konsensbildung} unter Bezugnahme auf die Interessen ihres
Publikums setzen -- wie dies die Frau in ihren Plädoyers vereinzelt
macht -- oder sie begeben sich in die heikle Sphäre des mehr oder weniger manipulativen
bis gewalttätigen Kampfes um die \emph{Führung eines unmündigen Gefolges}. Die
real gegebenen sozialen Gegensätze und Teilsysteme der modernen
Industriegesellschaft kann keine noch so rhetorisch begabte
Führergestalt abstrahierend aus der Welt schaffen.     

In der \Quote{autoritären}, auf Führergestalten orientierten Variante
politischen Agierens ist der \Quote{geistige} Humansozialismus dem brachialen,
sich kraftvoll gebärenden \Quote{Massensozialismus} logischerweise
unterlegen. Was einem Verbrüderungsidealisten bleibt, ist die moralische Kritik
der Gewaltmittel, die Dämonisierung der \Quote{falschen Führer} und die
Betrauerung der doch eigentlich so sehnsüchtig erwünschten Gefolgsleute als
verirrte, bewusstlose \Cite{Masse}.

Das Ideendrama \Cite{Masse Mensch} bewegt sich zirkulär auf dieser
Reflexionsstufe, weil das
Grundkonzept der messianischen Erweckung bis zuletzt beibehalten wird.
Die vollständige Isolation der Führerin von ihren Mitmenschen steht am Ende in
diametralem Gegensatz zu ihrer Gemeinschaftsidee.\Footnote{Die Arbeiter haben
  sie als Verräterin verstoßen, vom \Cite{Staatsystem} wird sie als Anführerin
  verhaftet und zum Tode verurteilt. Vgl. \BibRef{chen.98}{61}: \Cite{Deserted
    both by the bourgeoisie and the proletariat, the Woman experiences
    unspeakable loneliness.}}  
Das Wertesystem der Autornorm lässt als Ausweg nur den Tod der Isolierten zu:
Im \emph{heroischen Fatalismus des Selbstopfers} wird das pseudoreligiöse
Ritual einer verkannten Messiasfigur gestaltet. Der rein symbolische Akt kann
jedoch nicht überdecken, dass der verkündete Erweckungsoptimismus bereits
durch ein latentes Verzagen geschwächt ist.\Footnote{\Abr{Vgl} auch
  \BibRef{buetow.75}{141}: \Cite{Selbst im
  Schlußteil von \Quote{Masse Mensch} macht sich die Skepsis Tollers
  insofern bemerkbar, als dieser Glaube weder szenisch noch sprachlich so
  deutlich ausformuliert ist wie in der \Quote{Wandlung}.} Sigurd Rothstein
  geht noch weiter, indem er \Cite{die Relativierung der expressionistischen
    Erlösungsbotschaft} per \Cite{schrittweise[r] Demontage der Hauptfigur}
  diagnostiziert. Siehe \BibRef{rothst.87}{147}.}

\HeadingThree{Der Begriff der \Cite{Masse}}

In der Industriegesellschaft des \Nth{19} und frühen \Nth{20} Jahrhunderts
ergab sich aus den Prozessen der Urbanisierung, Industrialisierung und dem
Bevölkerungswachstum der Effekt einer massenhaften Anballung von Proletariat
in den ökonomischen Zentren.\Footnote{\Abr{Vgl} auch \BibRef{schnei.95}{453}:
  \Cite{Die Erfahrung der Moderne deckt sich in einer wesentlichen Schicht mit
    der Erfahrung der Masse und des Massenhaften.}}
Dieses Phänomen der gesellschaftlichen Moderne
wurde von zeitgenössischen Intellektuellen zweifellos nicht unproblematisch
empfunden. Das Auftreten großer Mengen \Quote{einfacher Leute} in den Städten
evozierte bei Wissenschaft und Intelligenz einen mit Ängsten und Vorbehalten
aufgeladenen Benennungsbedarf. Der passende Begriff der \Cite{Masse} wurde
wesentlich durch das wirkungsmächtige, 1895 erschienene Werk 
\Cite{Psychologie des Foules} von Gustave Le Bon geprägt.\Footnote{Deutsche
  Ausgabe: Gustave Le Bon: Psychologie der Massen. Stuttgart 1950.}  
Darin wird die \Cite{Masse} sinngemäß charakterisiert als \Cite{ungestaltes,
  furchterregendes Wesen, das
  aus dem Schoß der Geschichte aufstieg und ihren Gang verhängnisvoll zu
  beeinflussen drohte.}\Footnote{So Peter Uwe Hohendahl in seiner Analyse des
  Phänomens der \Cite{Masse} im expressionistischen Drama. Siehe
  \BibRef{hohend.67}{124}.}

Die ursprüngliche Massentheorie im Sinne Le Bons wird im Expressionismus um
den Aspekt der historischen Bedingtheit ergänzt. Wie Peter Uwe Hohendahl
in seiner Erörterung von 1967 herausarbeitet, versucht der Expressionismus
\Cite{die Masse als notwendige Folge einer fehlgeleiteten Entwicklung} zu
verstehen. An einer differenzierten Sicht auf das Phänomen hindert die
Expressionisten allerdings ihr \Cite{humanitär-individualistisches
  Menschenbild}, das dem der älteren Massenpsychologie ähnlich sei. Dem
Expressionismus müsse deswegen \Cite{die Massenhaftigkeit} als \Cite{Symptom
  einer kulturlosen, enthumanisierten Zivilisation} erscheinen, die zu
überwinden sei.\Footnote{\BibRef{hohend.67}{124}.}

\Title{Masse Mensch} ist laut Hohendahl ein Musterbeispiel für die
expressionistische Darstellung von \Cite{Masse}: Sie erscheint nicht als
gegliederte Menge von Einzelfiguren, sondern als ein \Cite{geschlossener,
  blockartiger Körper, der auch seine eigene Sprache spricht}. Wie im konsequenten
expressionistischen Massendrama üblich, existiert in dem Stück \Cite{mit
  Ausnahme der Führergestalt das Individuum nicht mehr}.\Footnote{\Abr{Ebd},
  \Page{127}.} 

Das expressionistische Verständnis von \Cite{Masse} führt zu einem Zwiespalt:
Wie das Beispiel \Cite{der Frau} zeigt, wird der \Cite{gestaltlosen} Masse
eine Führerfigur gegenübergestellt, die den \Cite{Mensch in Masse befrein}
\SourceRef{II}{108} will und dazu auf eine Mischung aus moralischem Appell
und Argument setzt. In der expressionistischen Attribuierung des
Massenbegriffs steht jedoch fest, dass \Cite{der Massenmensch unter der
  Herrschaft des Irrationalen [steht]}.\Footnote{\Abr{Ebd}, \Page{125}.} 
Der angestrebten \Quote{Kommunikation} zwischen Masse und Führer zwecks
\Cite{Überwindung} des Massencharakters fehlt somit eine grundlegende
Voraussetzung: Die Möglichkeit der vernünftigen Interaktion.   

Dieses Problem rührt nicht zuletzt daher, das im Begriff der \Cite{Masse} zwei
Lesarten vermischt werden: \emph{Psychologisch}
wirksame Masseneffekte, die jeweils \Cite{aktuell} das Verhalten einer
größeren Menschenmenge beeinflussen, einerseits -- die  \emph{sozialgeschichtliche}
Herausbildung eines städtischen Proletariats als Voraussetzung einer
\Cite{latenten Masse} andererseits. 

Die expressionistischen Begriffe \Cite{Volk} und \Cite{Gemeinschaft} filtern
den \Cite{aktuellen}, massenpsychologisch bedingten Lenkbarkeitsaspekt
gewissermaßen aus und ersetzen das Moment der irrationalen Triebkräfte durch die
erwünschte Erweckungsbereitschaft. Aus dem Massenbegriff übernommen wird der
sozialhistorisch bedingte Aspekt. Er fließt in das Bild der \Cite{latent}
mobilisierbaren, leidenden Volksmasse mit ein.

Entscheidend für die Bewertung der \Quote{Modernität}
des expressionistischen Verständnisses von \Cite{Masse} und \Cite{Volk} ist
dabei die durchgängige Reduzierung der proletarischen Großgruppen auf die Rolle des
belehrten Publikums, das von vordenkenden Führergestalten seine
\Quote{Marschrichtung} zu empfangen hat.\Footnote{Die Auffassung von Bozena
  Choluj, dass \Cite{die Frau} in Wirklichkeit \Cite{für
    eine bewusste, individualisierte Masse, \AbrPair{d}{h} für das werktätige
    Volk, für die Arbeiter} plädiere, kann ich nur sehr bedingt
  nachvollziehen, vgl. dazu \BibRef{choluj.91}{71}. 
  Ohne ihre Einschätzung schlüssig zu begründen, behauptet Choluj, dass die    
  proletarische \Cite{Bereitschaft zu einem selbständigen politischen Handeln}   
  ein ideeller Kernpunkt von \Title{Masse Mensch} sei:
  \BibRef{choluj.91}{9}.
  Nach meiner Auffassung wird ein individuell reflektiertes, selbständiges
  politisches Handeln in der Leitnorm des Stückes aber ausschließlich der
  Protagonistin zugestanden und bleibt auch dort in einen überindividuell
  absolut gesetzten Imperativ \Quote{humanen Gewissens} eingekapselt.}
Die Logik des dramatischen Ideenkonflikts bleibt in \Title{Masse Mensch} bis
zuletzt in einem
\Quote{gegenmodern} zu nennenden Volk-Führer-Schema befangen. 
Das Dilemma der abstrakt eingeforderten Entsprechung von Vordenker und Gefolge
wird mit den idealisierten Begriffen von \Cite{Volk} und \Cite{Gemeinschaft}
nicht aufgelöst, sondern lediglich einer moralisch anspruchvolleren
Schablone subsummiert.

%%% Local Variables: 
%%% mode: latex
%%% TeX-master: "/home/oldo/TOLLER/MAIN"
%%% End: 





