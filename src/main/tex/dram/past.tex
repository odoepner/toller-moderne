% Pastor Hall

\HeadingOne{Aufrechter Untergang in vaterländischer Barbarei:\\
  \Cite{Pastor Hall}}{\Cite{Pastor Hall}}

Ernst Tollers letztes Drama entstand 1938 in \Cite{New York, Barcelona,
  Cassis} \SourceRef{III}{246} und wurde noch bis Anfang 1939 für die
Veröffentlichung überarbeitet. Es ist 

\begin{BlockQuote}
Gewidmet dem Tag, an dem dieses Drama in Deutschland gespielt werden
darf. \SourceRef{III}{246}
\end{BlockQuote}
Doch Toller sollte schon die Uraufführung der englischen Version im
November 1939 nicht mehr erleben. Zur ersten Aufführung der deutschen Fassung
kam es erst im Januar 1947 im Deutschen Theater in Berlin. Als
antifaschistisches Exildrama ist es von den Kritikern meist in politischer
Hinsicht gelobt, dramaturgisch aber eher als \Cite{schwach} beurteilt
worden.\Footnote{\Abr{Vgl} \AbrPair{z}{B} \BibRef{benson.87}{138}:
  \Cite{\Title{Pastor Hall} ist in der Tat Tollers schächstes Stück.} Siehe
    auch \BibRef{reimer.00}{321}.} 
  
\HeadingTwo{Inhaltliche Zusammenfassung}

Im Zentrum des Dramas steht die bedrohte Welt des evangelischen Pastors
Friedrich Hall, der sich, seine Familie und seine Gemeinde als Bastion des
Anstands gegen den Einfluss der übermächtigen NS-Herrschaft zu verteidigen
versucht.

Entgegen den ängstlichen Anpassungserwägungen einzelner Gemeindemitglieder und
dem halblegalen Pragmatismus seiner Frau vertritt Hall einen \Cite{Weg der
Wahrheit} \SourceRef{III}{261}, der ethische Prinzipien über taktische
Erwägungen stellt. Er will
sich die Freiheit des kritischen Wortes bewahren und fordert von seinem Umfeld
dabei unbedingte Unterstützung.

Durch einen illegalen Geldtransfer, den seine Frau Ida ohne sein Wissen
veranlasst hat, werden die Halls erpressbar. Der NS-Mann Fritz Gerte, ein
Bekannter von Frau Hall, versucht daraus Nutzen zu ziehen. Die Veruntreuung
von brisanten Briefen durch eine Haushälterin führt zur Verhaftung des
Pastors.

Im KZ erlebt Hall die volle Respektlosigkeit der Nazis vor der Religion und
den Verlust seiner Sonderrolle als Pastor. Im heterogenen Gemisch der
Gefangenen, das von überzeugten politischen Aktivisten bis zu angepassten
Denunzianten reicht, kommt er mit dem sozialistischen Arbeiter Hofer ins
Gespräch. Gemeinsamkeiten des christlichen und proletarischen
Widerstandsethos, aber auch unterschiedliche Einstellungen zum Einsatz von
Gewalt werden deutlich.

NS-Mann Gerte ist inzwischen zum KZ-Leiter aufgestiegen. Nachdem Hall
wiederholt die vorbehaltlose Unterordnung unter das Wort \Cite{des Führers}
verweigert und Gerte sogar der Sünde beschuldigt, verurteilt dieser ihn zu harter
Strafe. Die Angst vor der Folter überlagert seinen Stolz. Mit Hilfe
eines SS-Manns, der dabei erschossen wird, gelingt ihm die Flucht.

Zu Hause im Kreis der Familie erlangt er seinen Stolz zurück. Nachdem der
Verfolger Gerte aus dem Haus verjagt werden kann, begibt sich Hall unter
Begleitung seiner Familie zu einer letzten Predigt vor seine Gemeinde, um
eine \Cite{Beispiel} \SourceRef{III}{316} zu geben.

\HeadingTwo{Die Diktatur als Bedrohung der ethischen Freiheit}

Gemessen an dem normativen Modernitätsbegriff, der in der vorliegenden
Arbeit zugrunde gelegt wird, ist der Nationalsozialismus als antimodernes
System zu kennzeichnen: Die herrschende Ideologie ist absolut gesetzt und wird
gegen jeden Widerstand \Quote{wahr gemacht}. Das Führerprinzip gilt in
voller Härte, und die Autonomie der Einzelsubjekts ist durch die Forderung
unbedingten Gehorsams auf ein Minimum beschränkt. Ernst Toller reagierte
darauf mit der dramatisierten Erörterung der Frage, wie ein
selbstbestimmtes, den eigenen Überzeugungen entsprechendes Handeln in einem
solchen Umfeld überhaupt noch möglich ist. 

Das Stück zeigt die Verhältnisse in Nazideutschland, die von staatlichen
Repressionsmethoden und grausamem KZ-Terror geprägt sind, aus der Perspektive
eines selbstbestimmten Anstandsdenkens, das seine ethischen Maßstäbe nicht dem
totalitären Machtanspruch des \Cite{Führers} und seiner Schergen beugen will.
Zentrale Begriffe sind dabei die bürgerlichen \Cite{Freiheiten} der Meinung
und des Glaubens, die vor einem falschen Kompromiss des \Cite{Schweigens}
\SourceRef{III}{296}
bewahrt werden sollen. Die lähmende Wirkung der \Cite{Furcht} wird als das
wesentliche Hilfsmittel der Machthaber thematisiert, ihre Überwindung ist der
positive Weg, den das Drama vorzuzeichnen vorsucht.\Footnote{Damit entspricht
  die Zielrichtung des Dramas sehr stark den Exilreden Tollers. In einer
  dieser Reden heißt es \AbrPair{z}{B}: \Cite{Kündet der Welt die
    Wahrheit. Fürchtet euch nicht. Wer die Furcht überwunden hat, hat den Tod
    überwunden.} Siehe \SourceRef{I}{207}.}

Die Verteidigung der persönlichen Freiheit ist die logische Folge jener
politischen Leitlinie der aktiven Teilhabe jedes Einzelnen an den Prozessen der
gesellschaftlichen Meinungs- und Willensbildung, die Toller unter anderem in
\Title{Hoppla, wir leben!} entwickelt hatte. Die NS-Herrschaft beschnitt diese
Grundfreiheiten auf ein Maß, das nur noch den Gehorsam gegenüber der
Parteilinie duldete. 

Die Macht der Nazis bedroht den Protagonisten in seinem Ethos als
Gemeindeoberhaupt und Familienvater. Er predigt für ein \Quote{echtes
  Christentum} und ist damit als Staatsfeind ins Visier
geraten.\Footnote{\SourceRef{III}{251}: \Cite{\DramName{Fritz Gerte}: Wir
    wissen längst, daß Ihr Mann die evangelische Kanzel dazu benützt, um deutsche
    Familien gegen unseren Führer aufzuhetzen.} Das Ende von Halls Freiheit
  scheint für den NS-Mann nur eine Zeitfrage zu sein: \Cite{[..] Ihn haben wir,
    wann wir wollen.}}  
Auf der persönlichen Ebene äußert sich diese Bedrohung als
\Quote{Anfechtung} der Wohlmeinenden: Halls Frau versucht mit halblegalem
Pragmatismus unter dem vorgeblichen Protektorat des \Cite{Sturmbannführers}
Fritz Gerte die Überzeugungen
ihres Mannes zu überspielen. Auch der Gemeindevertreter Pipermann ist bereit,
die Freiheit des Glaubens mit Zugeständnissen zu relativieren: Man müsse eben
\Cite{die Lippen versiegeln und das Wort im Herzen behalten}
\SourceRef{III}{263}.

Durch diese furchtsame Taktiererei und Denunziationsangst wird das
freiheitliche Ethos des Pfarrers sogar in seinem direkten Umfeld latent
untergraben. Die Furcht seiner Frau überdeckt familiäre Gefühle\Footnote{Von
  den befreundeten Grotjahns distanziert sie sich gegenüber Gerte 
  mit dem prophylaktischen Hinweis, es sei ihr auch nicht \Cite{angenehm},
  wenn ihre Tochter \Cite{in eine verdächtige Familie} heirate
  \SourceRef{III}{263}.}  
und verkehrt den Wert der Freiheit in sein
Gegenteil.\Footnote{Die Haftanstalt erscheint ihr als
  \Cite{sichere} Verwahrstelle: \Cite{\DramName{Ida Hall}: O ich    
    möchte schon im Gefängnis sein. Im Gefängnis ist Sicherheit und
    Frieden. [..]} \SourceRef{III}{263}.}  
Auch in der Gemeinde führt die Furcht vor finanziellen
Sanktionen\Footnote{\SourceRef{III}{263}: \Cite{\DramName{Traugott Pipermann}:
    Es heißt [..], daß uns die staatliche Unterstützung entzogen werden soll, [..]
    die Nazis drohen sogar mit einem Boykott unserer Geschäfte [..].}}  
dazu, dass das \Cite{Fundament} \SourceRef{III}{264} des Glaubens ins
Hintertreffen gerät und Kompromissbereitschaft
\Cite{systemunterstützend}\Footnote{\BibRef{unger.96}{299}.} wird.

Der Freiheitsverlust ergibt sich in \Title{Pastor Hall} als
Abhängigkeit und Erpressbarkeit, die aus dem Zusammenwirken
einer allgemeinen Bedrohung durch den Staat, gezielten Repressalien durch die
Partei und persönlicher Vorteilnahme des vermittelnden Nazis vor Ort
entsteht.\Footnote{\Abr{Vgl} \BibRef{unger.96}{294}: \Cite{[..] mit der
    häßlichen Verbindung von formaler Pflichterfüllung gegen den Führer und
    korruptem Eigennutz [gelangt] ein zusätzlicher Aspekt der Willkür der Macht in
    den Blick.}}  
Thorsten Unger charakterisiert dies als eine \Cite{faktische Auflösung der
  Rechtssicherheit}\Footnote{\BibRef{unger.96}{295}.}, die bei den
Figuren des Dramas zunächst mit Reaktionsmustern der Furcht oder apolitischer
Gleichgültigkeit beantwortet werde. Einschüchterungsresistenz und moralisches
Beharrungsvermögen führen dagegen zur kriminalisierenden Aussonderung. Die
Bewahrung der Gesinnungsfreiheit führt zum Verlust der äußeren Freiheit. Die
Figur Paul von Grotjahn spricht dies aus: \Cite{Ein anständiger Mensch ist
  heute im Gefängnis.} \SourceRef{III}{267}. 

In den KZ-Szenen wird die Härte dieses Zusammenhangs noch brutaler
klargestellt. Der staatliche Antisemitismus beschränkt sogar die Freiheit der
Heimatliebe,\Footnote{Das \Cite{Heimweh} eines Halbjuden wird im Lager
  höhnisch kommentiert: \Cite{\DramName{Egon Freundlich}: Dir sollte man eine
    Kugel in dein heimwehkrankes Herz schießen. Geht der Idiot ins Vaterland
    zurück, wo er unerwünscht, eine Rassenschande [..]  ist.}
  \SourceRef{III}{280}.} 
die NS-Ideologie duldet kein anderes Glaubensbekanntnis neben sich und
beansprucht \Cite{größer als der Mensch} \SourceRef{III}{282} zu sein. In
dieser Formel kommt die umfassende Reduzierung des freien Individuums auf eine
Haltung des Dauergehorsams zum Ausdruck.\Footnote{Den Idealtypus des
  NS-Deutschen fasst im KZ der \Cite{Stubenälteste} zusammmen:
  \Cite{\DramName{Egon Freundlich}: [..] Das Hirn in den Füßen, den Gehorsam im
    Blut, und das Hakenkreuz im Herzen.} \SourceRef{III}{287}.}  
Der Freiheitsverlust des Einzelnen wird als Leitnorm des totalitären Staates
ausgesprochen. Spätestens hier bewahrheitet sich Grotjahns Diagnose: \Cite{Im
  totalen Staat gibt es nur eine private Affäre: den Tod.} \SourceRef{III}{271}.
Für den, der sich kritisch äußern will, scheint sogar die semantische
Grundlage der (politischen) Verständigung gefährdet: Die Freiheit des Wortes
leidet nach Ansicht Halls darunter, dass \Cite{die Diktatoren} zentrale
Begriffskategorien \Cite{gestohlen und um ihren Sinn gebracht} haben
\SourceRef{III}{274}. 

Als Beispiel einer aufrechten, resistenten Umgangsweise mit dieser Entrechtung
ist in den KZ-Szenen der Arbeiter Peter Hofer gestaltet. Mit der Prämisse, dass
sein \Cite{Vaterland} dort sei, \Cite{wo Freiheit ist}
\SourceRef{III}{288}, zeigt er seine Unabhängigkeit von der Nazi-Ideologie. Er
bleibt sich treu, obwohl seine Genossen ihn im Stich gelassen haben, und hat
erkannt, dass Freiheit nicht nur eine \Cite{kleinbürgerliche Phrase} ist, wenn
man in einem System der \Cite{Sklaverei} lebt \SourceRef{III}{289}. Im
Gespräch mit Pastor Hall erzählt er vom Beispiel
Erich Mühsams, der trotz Nazi-Terror \Cite{an die Freiheit glaubte}
\SourceRef{III}{290} und bis zuletzt seinen Stolz bewahrte. Hofer ist
überzeugt, dass auch Hall ein \Cite{Rebell} sei \SourceRef{III}{291} und stärkt
damit dessen Widerstandsgeist.

Das Gespräch mit Hofer stellt einen Wendepunkt dar, ab dem die Gefährdung der
inneren Freiheit weniger bedrohlich erscheint. Die \Quote{Überwindung der
  Furcht} ist im Beispiel Erich Mühsams vorbildhaft repräsentiert: Statt dem
Horst-Wessel-Lied singt er seinen Henkern die Internationale. Im unbeugsamen
Stolz und der aktiv beanspruchten Macht des freien Wortes werden so die Mittel
vorgestellt, die jeder Einzelne dem Terror des System entgegen setzen
könnte.\Footnote{Laut \BibRef{spalek.89}{1748}, ist dies der Ausgangspunkt einer
  dramatischen \Cite{Kettenreaktion} von Ermutigungswirkungen, die der
  Vorstellung Tollers von einem \Cite{Effekt des Wellenschlagens}
  vorbildhafter Heldentaten entspreche.}
 
\HeadingTwo{Familie und Gemeinde: Bastionen des Anstands}

Friedrich Halls kleine Welt des geistigen Widerstandes erstreckt sich auf
seine Familie und die von ihm geleitete Gemeinde. In diesem
Umfeld besitzen er und seine mahnenden Worte eine gewisse Autorität:

\begin{BlockQuote}
  Ich stehe jeden Sonntag auf der Kanzel, ein Prediger in der Wüste, und
  verteidige die Lehre Christi gegen die Lehre meiner Widersacher. Ich lasse
  mich nicht einschüchtern durch ihre Drohungen und ihre Versprechungen,
  [..]. \SourceRef{III}{259}\Footnote{\Abr{Vgl} auch \BibRef{unger.96}{293}:
    \Cite{Pastor Hall predigt in seiner Gemeinde im Sinne der \Quote{Bekennenden
        Kirche}. Viele Gemeindemitglieder vertrauen ihm [..].}}
\end{BlockQuote}
Doch dieser Freiraum ist gefährdet. Die Wirkungen der Furcht untergraben den
Glauben und die Integrität derer, die den Pastor umgeben. Wie bereits im vorigen
Abschnitt dargestellt wurde, sind unter anderem seine Frau und der Gemeindevorstand
zu gefährlichen Kompromissen mit der Macht bereit.

Als gesinnungsfeste, aber etwas leichtfertig handelnde Stütze erweist sich Halls
Jugendfreund Paul von Grotjahn, der als ehemaliger General das konservative
Militär repräsentiert. Er ist aus Ehrgefühl zu keiner Unterordnung unter das
NS-Regime bereit, \Cite{Gehorchen und Maulhalten} lehnt er
ab.\Footnote{\Abr{Vgl} auch \SourceRef{III}{266}: \Cite{Ich bin ja ein Mensch
    vom alten Schlag. Mir können sie das Maul nicht verbinden.}}  
Seine Behauptung, er sei als Ehepartner \Cite{ein Ausbund von Treue}
\SourceRef{III}{270}, lässt sich ohne Weiteres auf die Ebene des Hallschen
Ethos übertragen. Die beiden Figuren bilden damit quasi als Achse aus
\Quote{wahrem Christentum} und \Quote{anständigem Militär} das Rückgrat des
gemäß der Autornorm anzustrebenden Widerstandsbündnisses.

Neben der wankelmütigen Gemeinde, die in allwöchentlichen Predigten vor der
\Cite{Wüste} der Barbarei bewahrt werden muss, hat auch die jüngere Generation
mahnende Worte nötig. Der geplanten Eheschließung seiner Tochter Christine
mit Grotjahns Junior widmet Hall eine vielsagende Tischrede:

\begin{BlockQuote}
  Zwei Menschen, die einander lieben, sind eine Welt, nicht Haß noch Verleumdung
  noch Gewalt können sie erschüttern, [..] Heißt das nun, der Verantwortung, den
  Lasten, den Aufgaben der Zeit entfliehen, die Türen verschließen und die
  Fenster verhängen vor dem Draußen? Nein. [..] \SourceRef{III}{276}
\end{BlockQuote}
Der jugendlichen Unbedarftheit der Tochter und der gesinnungslosen
\Cite{Objektivität} \SourceRef{III}{305} Werner von Grotjahns wird so eine
dezente Läuterung verabreicht, die zu widerstandsfähigem
Verantwortungsbewusstsein jenseits der reinen Privatsphäre auffordert.  
Den symbolträchtigen Rahmen der Zeremonie liefert die Hallsche Spieluhr: Das
Erbstück, das \Cite{schon so alt [ist], daß sie spielt, wenn sie will},
intoniert ein Loblied auf \Cite{den kühnen Mute} und \Cite{den Zorn der freien
  Rede} \SourceRef{III}{275}.

Einen Kontrast zu dieser Szenerie des Mutes und Zusammenhalts bildet die
unmittelbar folgende Verhaftung Halls und sein Abtransport in das KZ. 
Die Gemeinschaft ist auf eine Bewährungsprobe gestellt, Während Halls
Haftzeit kristallisiert
sich die Kerngruppe des Widerstandes heraus: Christine weist ihren Werner
von sich, nachdem er zu keiner klaren Parteinahme für den Pastor bereit ist.
Der alte Grotjahn setzt sich mit Pipermann -- der Verkörperung des zaghaften
Duckmäusers - auseinander und verweist ihn des Hauses.\Footnote{Grotjahn zu Pipermann:
  \Cite{Sie sind ein Jammerlappen, denn ein Judas ist aus anderm Holz
    gewachsen.} \SourceRef{III}{303}.} 
Ida Hall wandelt sich vom furchtsamen Nazi-Opfer zur mutigen Provokateurin und
Unterstützerin ihres Mannes.

So trifft Hall nach seiner Flucht aus dem KZ auf eine neu formierte Gruppe
engster Verbündeter. Seine Frau steht dem Verfolgten geistesgegenwärtig bei,
Halls Tochter und Grotjahn stoßen im richtigen Moment hinzu, so dass die
Gruppe gemeinsam den Verfolger Gerte verjagen und schließlich unter Halls
Führung den Weg in die Kirche antreten kann.\Footnote{Sigurd Rothstein sieht
  in dieser Entwicklung die späte Einlösung jener Gemeinschaftsideale, die
  sich wie ein roter Faden durch die Tollersche Dramatik ziehen: \Cite{In
    \EmphQuote{Pastor Hall} schließlich wird der Traum von der Gemeinschaft
    wahr, aber ironischerweise vorerst nur im kleinsten Kreise.} \Abr{Vgl}
  \BibRef{rothst.87}{286}.}

Die abschließende rebellische Wendung verweist auf die Prognose Peter Hofers
zurück, der Hall bereits im
KZ als \Cite{Rebell} bezeichnet hatte. So sind in dieser Gemeinschaft
der Mutigen zuletzt die humanistischen Kräfte der Geistlichkeit, des kaiserlichen
Militärs, der Arbeiterbewegung und des einfachen Bürgertums symbolträchtig
vereint. Der gemeinsame Gang zur Predigt verbindet die Perspektive einer
potentiellen Ausweitung des Widerstands mit der Betonung der Kraft des
freien Wortes und aktiviert die \Quote{Bastionen des Anstands} zu
einem finalen \Quote{showdown}.

\HeadingTwo{Der Protagonist und sein \Cite{Weg der Wahrheit}}

Die Haltung Friedrich Halls wird als \Cite{betont gerade}, seine Wirkung als
\Cite{nie pathetisch}\Footnote{Tom Kuhn
  hält diese Merkmalszuschreibung in den Regieanweisungen für fragwürdig:
  \Cite{[..] yet some of his speeches seem to be designed, almost, for their
  honest pathos.} Siehe \BibRef{kuhn.92}{172}.}
beschrieben. Er ist Pazifist, aber kein Revolutionär. Im ersten Weltkrieg
hat er \Cite{Friedenspredigten} gehalten und dafür später geringere
Aufstiegschancen in Kauf genommen \SourceRef{III}{251}. Die auf christlichen
Werten basierende kritische Haltung Halls, die seine Frau gelegentlich
\Cite{Fanatismus} nennt, ist
kämpferisch. Der Mensch sei \Cite{nicht auf dieser Welt, um der Gefahr
  auszuweichen} \SourceRef{III}{256}, lautet einer seiner Wahlsprüche.  

Pastor Hall hat erkannt, dass der Zusammenhalt der Menschen durch
wechselseitigen, vertrauensvollen Austausch eine Stärkung erfährt, die den auf
Gehorsam angewiesenen Machthabern Sorgen bereiten
kann.\Footnote{\SourceRef{III}{256}:
  \Cite{Solange der Mensch Vertrauen zum Mitmenschen hat, und die Nöte seiner
    Seele bekennt, ist der Staat seiner nicht sicher.}}
In seiner Gemeinde ist er Anlaufstelle und Fürsprecher dieses offenen
Austauschs. Seine \Cite{Pfarrkinder} \SourceRef{III}{255} teilen ihm
Beschwerden über das
Nazi-Regime mit, die sie öffentlich nicht aussprechen würden. Sein Engagement
gegen die \Cite{krankhafte Furcht} \SourceRef{III}{258} ist begleitet von
einer prinzipiellen
Gesetzestreue, die im Zweifel höher steht als das Wohl seiner Familie.
Seine Prinzipientreue geht so weit, dass er darin eine potentielle Belastung für
seine Angehörigen vermutet:

\begin{BlockQuote}
  Ich habe mich oft gefragt, ob ein Mensch, wie ich, der den Weg der Wahrheit
  sucht und nichts als die Wahrheit finden will, das Recht hat, die
  Verantwortung für Frau und Kind zu übernehmen und sie zu belasten mit seiner
  Bürde. \SourceRef{III}{261}
\end{BlockQuote}
Mit dieser Kompromisslosigkeit begegnet er seiner Frau, aber auch dem
Gemeinderat, dessen knieweiche Haltung gegenüber den Nazis er notfalls mit der
\Cite{Vertrauensfrage} \SourceRef{III}{264} beantworten würde. Zu Halls
\Cite{Weg der Wahrheit}
gehört es, von den für wahr gehaltenen Prinzipien kein Stück abzuweichen, sie
im eigenen Handeln konsequent zu befolgen und diese Überzeugungen darüber
hinaus klar zu formulieren. So mag er weder die \Cite{billigen Ausflüchte}
politischer Witzereißer \SourceRef{III}{267} noch die einseitigen Forderungen
nach Tugenden, die man selbst nicht einhält.\Footnote{Er würde eher
  \Cite{die Menschen beklagen, die versagen, weil sie Gerechtigkeit fordern,
    ohne gerecht zu sein} als irgendwelche politischen Systeme. \Abr{Vgl}
  \SourceRef{III}{261}.} 

Das  Ideal der furchtlosen Unbedingtheit konstituiert aber
zugleich Friedrich Halls persönliche Verfehlung. Er, der seine Schwierigkeiten
mit den Nazis und die Erpressungsversuche Gertes als eine \Cite{Prüfung}
\SourceRef{III}{260} auffasst,
die er klaglos auf sich nehmen will, erliegt in der KZ-Internierung zunehmend
der Selbstüberschätzung. Er glaubt, \Cite{auf das Schlimmste}
\SourceRef{III}{295} gefasst zu sein,
doch die drohende Folter versetzt auch ihn in Angst. Die ersehnte Flucht aus
dem Lager gelingt schließlich nur, weil er die Hilfe eines Wächters in Anspruch 
nimmt, der dabei sein Leben lässt.\Footnote{Sigurd Rothstein sieht darin den
  spezifischen \Quote{Egoismus} Halls:
  \Cite{Friedrich Hall [der] seinem Gewissen unbedingt verpflichtet
  ist, ist eben deswegen nicht, wie er selbst erkennt, frei von egoistischen
  Zügen.} \Abr{Vgl} \BibRef{rothst.87}{250}.}

Hall erkennt seine Schwäche. Er weiß, dass er \Cite{stolz war}
\SourceRef{III}{309} und \Cite{versagt hat} \SourceRef{III}{310}. Erst die
Gemeinschaft von Familie und Ehefrau, die er zuvor seinem Stolz unterordnete,
lässt ihn wieder erstarken und gibt ihm den Mut, seinem Peiniger Fritz Gerte 
vollmundig entgegen zu treten:

\begin{BlockQuote}
\DramName{Friedrich Hall}: Auch ich war feige. Ich wollte der Prüfung, die mir
auferlegt war, ausweichen. Ich stelle mich Ihnen. Die Zelle wird meine Stimme
nicht ersticken. Noch der Block, auf den Sie mich spannen, wird eine Kanzel
sein, und die Gemeinde so mächtig, daß keine Kirche der Welt sie fassen
könnte. \SourceRef{III}{313}\Footnote{Dies ist zweifellos eine der Stellen,
  die Hall keinesfalls \Cite{unpathetisch} erscheinen lassen.}
\end{BlockQuote}
Statt sich zu verstecken, wählt er den Weg der (verbalen)
Konfrontation. Nachdem Gerte vertrieben werden kann und die Option der Flucht
sich erneut auftut, wählt er abermals den Weg des entschiedenen Wortes. Statt zu
\Cite{desertieren} \SourceRef{III}{315}
begibt er sich als sozusagen als \Quote{Soldat der Wahrheit}
in den \Quote{Gefechtsstand} auf seiner Kirchenkanzel.

Wie sich damit zeigt. liegt in der letzten Konsequenz von Halls \Cite{Weg der
  Wahrheit} eine Opferbereitschaft, die im Zweifelsfall auch den Tod nicht
scheut. Ernst Toller musste recht stark \Cite{ins Register der Märtyrerkonventionen
und Christusstilisierungen}\Footnote{\BibRef{unger.96}{297}.} greifen, um das Konzept der
Unbedingtheit, das hier in der Vorreiterfigur des Protagonisten verdichtet
ist, mit der entsprechenden Verbindlichkeit auszustatten. Zu der
\Cite{Wahrheit}, die sich damit ihren \Cite{Weg} brechen sollte, gehörte immerhin die
Überzeugung, dass kein Geringerer als der \Cite{Antichrist}
  \SourceRef{III}{315} Adolf Hitler in
seiner Schreckensherrschaft zu stürzen sei. 
Trotzdem läuft die Dramenhandlung hier Gefahr, zu sinnlos überhöhter
Märtyrerhandlung zu verkommen. Während der Protagonist sich entrückt als
Zündfunke eines Feuers der Empörung imaginiert, deuten die
Hintergrundgeräusche ein rasches Ende an. Man hört \Cite{den Marschtritt einer
  sich nähernden Kolonne. Stärker. Drohender.} \SourceRef{III}{315}.

\HeadingTwo{Der Appellcharakter des Stückes}

\Title{Pastor Hall} ist eine Reaktion auf die Übermacht des
Nationalsozialismus, in dem Toller die ins Extrem geführte Weiterführung jenes
Nationalismus erkennen musste, dem er als freiwilliger Teilnehmer des
Ersten Weltkriegs selbst vorübergehend erlegen war. Die überwunden geglaubte,
brutale Ideologie des \Cite{Vaterlandes} hatte ihn eingeholt. Toller, der als
erklärter Gegner der Nazis zu den Ersten gehörte, die durch das neue Regime
ausgebürgert wurden, musste ohnmächtig zusehen, wie Verwandte und Freunde
schikaniert und verfolgt wurden.\Footnote{\Abr{Vgl} \BibRef{dove.93}{236ff.}
  und \BibRef{rothe.83}{118f.}.}

Ernst Tollers letztes Drama ist also ein
verzweifelter Versuch, den Siegeszug jenes gegenmodernen Systems des
Rassismus, der totalitäten Führerherrschaft und bedingungslosen Militarisierung 
in seiner Gefährlichkeit aufzudecken und ein dramatisches Beispiel dafür zu
formen, dass das Wort und der Mut des Einzelnen auch und gerade in dieser Situation eine
besondere Bedeutung haben. Die in \Title{Hoppla, wir leben!} vertretene
Forderung nach aktiver politischer Mitarbeit jedes Einzelnen bei der Gestaltung
der gesellschaftlichen Realität hatte angesichts der totalitären Verhältnisse in
Nazideutschland ihre Voraussetzungen eingebüßt. Das Hauptthema von
\Title{Pastor Hall} ist deswegen die Frage, wie unter den Umständen des
Terrors die wesentliche Grundlage für selbstbestimmte gesellschaftliche Teilhabe
-- nämlich die persönliche Gesinnungsfreiheit -- bewahrt werden kann. 

Angesichts der Aussichtslosigkeit der politischen Lage, die der
Exilant nicht ohne Grund schon bald mit seinem Freitod quittieren sollte,
verwundert es nicht, dass Toller in der Gestaltung seines Widerstandsexempels auf die
pathetischen Motive von Selbstopfer und hehrem Idealismus zurückgriff, die
man seit der expressionistischen Frühphase immer weniger in seinem Werk
hatte antreffen können. So sind in dem Drama vorbildhafte Figuren als
Ideenträger gestaltet, die zumindest Erinnerungen an Tollers expressionistische
\Quote{Helden} aufkommen lassen.\Footnote{Evelyn Röttger betont \AbrPair{z}{B}, dass
  \Cite{nicht nur Hall, sondern auch Hofer [..] Vertreter der unbedingten
    Hingabe an die \Cite{Idee}} seien. Siehe \BibRef{roettg.96}{257}.}

Dabei handelt es sich aber nicht um die entrückte Ideendramatik
humanrevolutionärer Prägung. Vielmehr konstruiert Toller seine Vorbildfiguren 
in bewusster Ausrichtung auf das anvisierte Publikum.
Es war abzusehen, dass das Stück zuerst im englischsprachigen
Ausland aufgeführt werden würde und \Cite{ein eher konservatives,
  bürgerliches, gebildetes Publikum, an dessen moralisches Bewusstsein
  appelliert werden konnte}\Footnote{\BibRef{reimer.00}{335}.} zu
erwarten war.\Footnote{\Abr{Vgl} auch Hermann Kesten: Meine Freunde, die
  Poeten. Wien 1953. \Page{159}: \Cite{[In seinem Drama \EmphQuote{Pastor
      Hall} versucht Ernst Toller] die Verschiebung der politischen
    Kräfteverhältnisse nach 1933, die Methoden der neuen Machthaber und das
    mutige Verhalten einzelner Bürger ausländischem Publikum anschaulich zu
    machen.} Zitiert nach \BibRef{ruehle.74}{836}.}

So ist wohl zu erklären, dass zum ersten Mal in Tollers
Dramatik ein Vertreter des institutionalisierten Christentums als Träger der
ethischen Leitnorm figuriert, nachdem sonst stets die
Rechtfertigungsfunktion der Kirche im Machtgefüge des Staates betont worden
war.\Footnote{\Abr{Vgl}
  \BibRef{rothst.87}{260}: \Cite{Bedeutsam ist, daß in \EmphQuote{Pastor
      Hall} der Gegensatz zwischen institutionalisiertem und 'echtem'
    Christentum in der Figur Friedrich Hall überwunden ist. Er ist die
    Kontrastfigur zu den negativ dargestellten Pfarrerfiguren früherer Dramen.}}
Der bürgerlich konservativen Figur des Pastors ist darüber hinaus mit Paul von
Grotjahn ein Mann des
Militärs (!) solidarisch zur Seite gestellt und die bei Toller sonst so wichtige
Bezugnahme auf die politische Aktivität der Arbeiterbewegung ist mit der Figur
Peter Hofer zwar präsent, aber eher dezent und selbstkritisch gehalten. 

Es ist in der Forschung relativ klar herausgearbeitet worden, dass Tollers
politische Ansichten mit der Figurenperspektive Friedrich Halls nicht völlig identifiziert
werden können.\Footnote{So kommt beispielsweise Evelyn Röttger zu der
  Einschätzung, dass \Cite{Tollers Selbstverständnis von
    mehreren Figuren her} gefaßt werden müsse. Dabei nennt
sie \Cite{insbesondere Hall, Hofer und v. Grotjahn} als maßgebliche
  Ideenträger. \Abr{Vgl} \BibRef{roettg.96}{259}.}
Im Zusammenspiel der humanistisch-liberal inspirierten Figuren wird die
Vorbildrolle des Protagonisten mit Sinn gefüllt. Der zunächst zu stolze, die
eigene Willensstärke überschätzende Pastor wird erst im Laufe der Handlung
gleichsam zur \Quote{Speerspitze} des gemeinsamen Widerstands.  

Die überhöhte Märtyrerrolle des Protagonisten hat wiederholt Vergleiche mit
Tollers \Title{Wandlung} hervorgerufen.\Footnote{Siehe unter anderem
  \BibRef{benson.87}{139}, oder \BibRef{reimer.00}{341}.}  
Die Vornamensgleichheit der
Protagonisten ist dabei noch das banalste Indiz, während die prophetische
Stilisierung und der Jesusvergleich am Dramenschluss von \Title{Pastor Hall}
tatsächlich typisch messianische Züge tragen.\Footnote{\Abr{Vgl}
  \BibRef{unger.96}{294}: \Cite{Mit biblischem Vokabular und in prophetischem
    Futur stilisiert sich Friedrich Hall hier zu einem Beispiel und ruft
    gleichsam andere Christen zur Nachfolge auf. [..] [Das Stück] hat damit
    gewisse Züge eines Märtyrerdramas.} Thorsten Unger sieht darin die Absicht
  einer \Quote{Aufrüttelung} des Zuschauers.}
Es wäre allerdings abwegig, Ernst Toller in seinem letzten Stück die Rückkehr
zum Verkündigungspathos und damit die ewige Verhaftung mit seinen
literarischen Ursprüngen nachweisen zu wollen.\Footnote{Anklänge einer solchen
  Behauptung finden sich beispielsweise bei \cite{benson.87}.}
Neben der Vielfalt an
dramatischen Mitteln, die in anderen Stücken Tollers künstlerische
Weiterentwicklung dokumentieren, lassen sich auch in \Title{Pastor Hall}
Argumente gegen eine solche Einschätzung finden. Zunächst ist zu beachten,
dass gerade der pathetische Schluss erst auf Anraten von Freunden in seine
märtyrerhaft missionarische Form gebracht wurde. In der ersten Textversion hatte Hall
in der Schlusszene einen Herzanfall erlitten, der heldenhafte Gang zur Predigt
fehlte somit.\Footnote{\Abr{Vgl} \BibRef{reimer.00}{320}: \Cite{Toller änderte
    diesen Schluß, nachdem er am 12. Januar 1939 Mitexilanten sein Stück
    vorgelesen hatte, die den Ausgang kritisierten [..].}}   
Zudem ist das Handeln Pastor Halls eine persönliche Entscheidung, dem die
Gefährten aus eigener Entscheidung folgen. Sie wird motiviert als notwendiger 
Widerstand gegen eine systematisch lebensbedrohende Macht. Die Überwindung der
Furcht wird als Voraussetzung dargestellt, die eigene Überzeugung offen
aussprechen zu können:

\begin{BlockQuote}
Während der Friedrich der \Title{Wandlung} der Prophet einer anderen,
utopischen Welt ist, ist Friedrich Hall der Anwalt der Wahrheit. Seine Rolle
ist eine wesentlich diesseitigere.\Footnote{\BibRef{reimer.00}{341}.}
\end{BlockQuote}
Die \Cite{Wahrheit} Friedrich Halls ist dabei eine Mischung aus einfacher
Tatsachenaufdeckung, die gewisse Machenschaften des NS-Regimes entlarven
würde, und
moralischen Grundüberzeugungen, die zum Kernbestand eines humanistischen
Wertekanons gezählt werden können. Die kämpferische Rede ist dem Erhalt
persönlicher Freiheit gewidmet und unterscheidet sich damit sehr stark von den
Verschmelzungsutopien des expressionistischen
\Cite{Menschheitskultes}.\Footnote{Hye Suk Kim weist auf die grundlegend
  veränderte Rolle der Protagonisten in der späten Dramatik Tollers hin:
  \Cite{[..] ihr Ziel ist nicht, als idealer Mensch anderen vorzustehen und
    sie zu führen, sondern nur, ein Beispiel des Widerstandes zu geben.}
  \BibRef{kim.98}{344}.}

Toller wollte mit dem Stück offensichtlich ein Bild von jenem \Quote{anderen
  Deutschland} entwerfen, von dem er so oft in seinen Vorträgen im Exil gesprochen
hatte.\Footnote{Exemplarisch sei auf Tollers Rede \Cite{Unser Kampf um
    Deutschland} von 1937 verwiesen. Hier ist die Anekdote von Erich
  Mühsams aufrechtem Tod ein zentrales Beispiel für das
  \Quote{andere Deutschland}. Die Darstellung ähnelt auch im Sprachduktus sehr
  stark der Erzählung Peter Hofers in \Title{Pastor Hall}. \Abr{Vgl}
  \SourceRef{I}{204}.}
Damit sollte die Idee der \Cite{Volksfront} als Bündnis eben dieser Widerstandskräfte
forciert und das demokratische Ausland zu einer geschlossenen Haltung gegen
Nazi-Deutschland bewegt werden.\Footnote{Thorsten Unger sieht hier eine
  \Cite{Analogie in der appellativen Folgerung des Dramas}: \Cite{Wie es im
    Innern darauf ankommt, daß die Bürger ihre Furcht überwinden, so müssen auch
    die freiheitlichsten Staaten ihre Furcht überwinden, um von außen [..]  auf
    Nazi-Deutschland einzuwirken.} \BibRef{unger.96}{296}.}

Toller greift dabei nicht auf das hohle Pathos der Menschheitsverbrüderung
zurück und bedient sich auch nicht nur des Mittels der vorbildhaften
Dramenfigur. Wie Thorsten Unger überzeugend herausarbeitet, geht es in
\Title{Pastor Hall} neben der Bewahrung persönlicher Integrität unter dem
Druck der NS-Herrschaft auch um die \Cite{kulturelle Integrität} des
\Quote{anderen Deutschlands}, die in erster Linie \Cite{auf dem Wege der
  intertextuellen Erinnerung} vermittelt werde.\Footnote{Für die folgende 
  Argumentation \Abr{vgl} \BibRef{unger.96}{302f.}.}
 Das im Nationalsozialismus
zwecks Ausblendung alles Unerwünschten intensiv betriebene \Cite{kulturelle
  Vergessen} bringe einen umfangreichen Bestand verdrängten Kulturguts mit
sich, auf den in Tollers Drama an verschiedenen Stellen Bezug genommen werde. 

So wird das \Cite{kritische Einverständnis} der beiden Hauptfiguren Hall und
Grotjahn anhand eines \Cite{verbotenen} Gedichts Heinrich Heines klar gemacht. Die
Linie der kulturellen Erinnerung erstreckt sich weiter über das
\Cite{subversive Potential} der Niemöller-Anekdote, die Grotjahn erzählt,
und die stolzen Klänge der Spieluhr.\Footnote{Dabei handelt es sich allerdings
  um einen Ausschnitt aus dem \Cite{Vaterlandslied} Ernst Moritz Arndts, dessen
  patriotische Ambivalenz dadurch entschärft werden muss, dass der
  \Quote{anständige General} Paul von Grotjahn den Text \Cite{ohne Pathos}
  \SourceRef{III}{275} rezitiert.} Die im KZ
erfolgende Erzählung vom Widerstand Erich Mühsams stellt sogar eine
mehrstufige Form der \Cite{Vergegenwärtigung} von \Cite{Gegenkultur} dar, da
hier von einer Symbolfigur des Widerstandes berichtet wird, die ihrerseits
einen antifaschistischen \Cite{Schlüsseltext}, nämlich die
\Quote{Internationale} anstelle des \Quote{Horst-Wessel-Liedes} zum Besten
gibt. Ein weiteres gegenkulturelles Beispiel ist schließlich das
\Cite{Moorsoldatenlied}, in dem die KZ-Häftlinge ihre Hoffnung auf eine
bessere Zukunft zum Ausdruck bringen.\Footnote{Das gleiche Lied ist auch von
  Bert Brecht in \Title{Furcht und Elend des Dritten Reiches} verwendet
  worden. \Abr{Vgl} \BibRef{unger.96}{307}.}

Die Methode der kulturellen Erinnerung unterlegt die um Friedrich Hall
gruppierte, von den Nazis an den gesellschaftlichen Rand gedrängte Welt des
humanistischen Widerstands mit einer Reihe von Traditionsbezügen, die das
Bild einer \Quote{menschlichen} Gegenkultur entstehen lassen. In diesem Sinne
ist in \Title{Pastor Hall} die Randstellung des Protagonisten positiv
aufgewertet und gleichsam mit einem Schimmer der Hoffnung ausgestattet. Im
Gegensatz zu den anderen Werken Tollers, in denen die gesellschaftliche oder
ideelle Isolation der Hauptfigur stets als mehr oder weniger tragisches Problem
durchexerziert wird, ist Halls überzeugtes \Quote{Nicht-dazu-Gehören}
ein ethisches Qualitätsmerkmal. War in den früheren Werken Tollers der
moralische Dissens mit der Welt immer auch ein Anlass für die Verunsicherung
des Protagonisten,\Footnote{Man vergleiche sie \Cite{Suche} Friedrichs in der
  \Title{Wandlung}, die Schuldfrage der \Cite{Frau} in \Title{Masse Mensch}
  und das \Quote{Irren an der Welt} bei Karl Thomas in \Title{Hoppla, wir
    leben!}.}
so ist die Frontstellung hier ganz klar: Wo \Cite{der Vater der Lüge}
\SourceRef{III}{266} die Losungen der offiziellen Weltsicht ausgibt und
\Cite{der Antichrist regiert} \SourceRef{III}{316}, ist der \Cite{der Weg der
  Wahrheit} eine der wenigen Alternativen, die die Autonomie des ethischen
Subjekts noch zu erhalten versprechen. In einem antimodernen System, das den geistigen Raum
totalitär auszufüllen versucht, ist der lautstarke appellative Ungehorsam die
Grundlage jeder späteren Pluralität. Und so war Ernst Toller in  der
Konfrontation mit der Barbarei auch der Rückgriff auf
eine märtyrerhafte Figurenkonzeption recht. 

Die Konzeption des Selbstopfers ist
sicherlich keine moderne Konzeption. Berücksichtigt man aber, dass die
konkrete
Handlungsalternative nur in der inneren oder äußeren Emigration -- also in dem
faktischen Zurückweichen vor einem antimodernen System -- bestanden hätte und
dass Toller mit seinem Drama auf eine gemeinsame, und eben möglichst nicht auf
Märtyrerniveau verbleibende Widerstandsbewegung hinwirken wollte, so muss
man dem Stück zweifellos eine moderne Stoßrichtung attestieren. 


