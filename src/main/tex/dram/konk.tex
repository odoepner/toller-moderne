\HeadingOne{Zusammenfassung und Schlussbemerkungen}{ZUSAMMENFASSUNG}

Im Folgenden werde ich die wesentlichen Ergebnisse meiner Untersuchung 
kurz zusammenfassen und dann zu dem im Titel dieser Arbeit angedeuteten
Spannungsverhältnis von \Cite{Weltverbesserung und Isolation} einige
Schlussbemerkungen machen.

\HeadingTwo{Die Ergebnisse im Überblick}

Ausgangspunkt der Überlegungen war das -- aus Prozessmerkmalen der
gesellschaftlichen Moderne abgeleitete -- Phänomen der \Cite{Identitätskrise
  des modernen Subjekts}, das durch Ernst Tollers \Cite{soziale Randlage}
noch verstärkt wird. Die Frage der \emph{sinngebenden sozialen Zugehörigkeit}
erwies sich daher als ein Grundthema der untersuchten Texte. 

In der \Title{Wandlung} sind die Auflösungserscheinungen der normativen
Verbindlichkeit sozialer Herkunftskontexte und der Wertverlust der religiösen
Sphäre am Beispiel des entwurzelten Protagonisten Friedrich dargestellt. Das
Drama entwickelt daraus sein zentrales Motiv der \emph{Sinnsuche}.

Die Suche ist zunächst als nationalistischer Irrweg angelegt. Das
\Cite{Vaterland} entpuppt sich als eine nur vordergründig sinngebende
Gemeinschaft und wird in seiner grausamen Realität gezeigt. 
Die Funktionalisierung von Medizin und Wissenschaft, die Instrumentalisierung
christlicher und traditioneller Werte und die Degradierung der Individuen zu
zweckdienlichen Objekten sind die wesentlichen Ansatzpunkte der Kriegskritik.
Die Charakterisierung des blutigen Irrwegs fasst sich zusammen in der Formel
vom \Cite{Tod als Feind des Geistes}.  

In konsequenter Generalisierung des Gemeinschaftsideals wird das Konzept
vom \Cite{Neuen Menschen} als Basis einer \emph{Menschengemeinschaft in
Einheit und Ganzheit} entworfen. Die Erreichung dieses Zustandes wird durch
einen Erweckungsmythos vorgezeichnet, der eine Rollenverteilung zwischen
\Cite{Geistigen} und \Cite{Volk} beinhaltet. Die Verbrüderungsidee ist
begleitet von einem Imperativ der allseitigen, gütigen Liebe.

Es konnte herausgearbeitet werden, dass der in der \Title{Wandlung} zum
Ausdruck kommende Verbrüderungsglauben ein attraktives Weltdeutungsmodell für
zeitgenössische Intellektuelle darstellte, das durch das universelle Konzept
der reinen \emph{humanitas} auch als Ersatz für verlorengegangene religiöse
Verbindlichkeit geeignet war. Mit der Perspektive auf ein potentiell
grenzenloses Publikum ergab sich eine erhebliche Aufwertung des
(künstlerischen) Intellektuellen. Dieses neue Evangelium der
\Quote{Künstler-Priester} versprach die geistige Rücknahme der
komplexen sozialen Wirklichkeit und ließ den Interaktionsprozess mit
bedrohlichen \Quote{Volksmassen} als einfachen Automatismus der Erweckung
erscheinen.

Der gegenmoderne Charakter der Modernereflexion der \Title{Wandlung} zeigt
sich somit an dem \Cite{messianisch}-totalitären Wahrheitskonzept, der
schematischen Reduktion des Subjektstatus im zugehörigen Volk-Führer-Modell
und dem alles dominierenden Bedürfnis nach gesellschaftlicher Einheit und
Ganzheit.

Eine zusätzliche Problematisierung ergab sich für die Idee der
Erweckungsrevolution durch das Scheitern der realen Räterevolution, in der mit
dem klassenkämpferischen Sozialismusverständnis ein wirkungsmächtiger
Gegenspieler auf den Plan getreten war. Das \Cite{Volk} hatte sich in seiner
Erweckungsbereitschaft als unzuverlässig erwiesen, was aus
humansozialistischer Perspektive nur als tragischer Sieg der \Cite{Masse} über
den \Cite{Menschen} aufgefasst werden konnte.

Das entsprechend benannte Drama Tollers zeigt abermals eine sozial entwurzelte
sinnsuchende Hauptfigur, die ihren Zugehörigkeitskonflikt zwischen Staat und
Proletariat diesmal unmittelbar im revolutionären Geschehen auszustehen hat. Das
Scheitern der messianischen Führungsfigur am Phänomen der \Cite{Masse} wird
hier als Schuldproblem
verhandelt. Ihre isolierte Stellung bleibt bis zum Schluss bestehen und mündet
in einen -- als Ideenrettung motivierten -- Übergang ins Jenseits. Damit ersetzt
in diesem Stück der \emph{heroische Fatalismus des Selbstopfers} die misslungene
Verschmelzung mit der Menschheit.

Die Problematik von \Title{Masse Mensch} zeigt das tiefgreifende Dilemma der
Verbrüderungsidee. Die in der \Title{Wandlung} per pseudoreligiösem Kunstgriff
hergestellte \Cite{Gemeinschaft} basiert auf einer Ausblendung des
Subjektcharakters der Volksmassen. Der praktische Realisierungsversuch dieser
Abstraktion musste an der Wirklichkeit scheitern. Weder argumentative
Konsenbildung noch gewaltsame Massenführung sind in der messianischen Idee
vorgesehen. Die Unterlegenheit gegenüber dem klassenkämpferischen
\Cite{Massen}-Sozialismus ist begleitet von moralischen Behauptungsversuchen: 
In \Title{Masse Mensch} äußert sich dies als moralische Kritik der
Gewaltmittel, als Dämonisierung der \Quote{falschen Führer}
und als Klage über die Unzurechnugsfähigkeit der \Cite{Masse}.

Moralisch zeichnet sich die Verbrüderungrevolution zwar durch
ihren prinzipiellen Gewaltverzicht aus, doch die Rolle von Masse und Volk
bleibt messianisch festgelegt: Das antimoderne Schema von Führer und
Gefolge ist beibehalten -- es wird lediglich in seinem \Cite{tragischen}
Scheitern beklagt. 

Ein bedeutender Einschnitt in der Darstellung der revolutionären Thematik ist
erst in
\Title{Hoppla, wir leben!} zu verzeichnen. Toller thematisiert anhand der
Konfrontation mit der republikanischen Gegenwart deren scheinbare
Demokratisierung ebenso wie den Verdacht ihrer reaktionären Unterwanderung. 
Restbestände des expressionistischen Revolutionsideals werden dabei
aufgegriffen und ad absurdum geführt. 

Tollers Abschied von der expressionistischen Dramatik äußert sich in der
\emph{Problematisierung des messianischen Führerideals}, einer Pluralisierung der
Figurenperspektive und individuell konturierten Figuren. Das neue
Subjektverständnis ermöglicht ein auf verschiedene Figuren verteiltes
Erkenntnispotential. Mit der Parallelisierung von messianischem
Erweckungsglauben und faschistischem Umsturzverauch wird die Gemeinsamkeit der
totalitären Ideologien herausgestellt. 

Das Drama entwickelt eine konsensorientierte Handlungsnorm der demokratischen
Teilhabe und des konstruktiven politischen Engagements. Es kann als eine
\emph{undogmatische Aufforderung zur politischen Einmischung} verstanden werden und
markiert insgesamt einen deutlichen Modernisierungsschritt im Werk Tollers.

Das letzte der behandelten Dramen, \Title{Pastor Hall}, ist eine Reaktion auf die
gesellschaftliche Realität des antimodernen NS-Staats. Die Marginalisierung
des ethischen Subjekts durch die Nazis beantwortete Toller mit seiner
Darstellung des \Quote{aufrechten Gangs der Anständigen}. Das Stück ist ein Appell 
an mögliche Unterstützer des Widerstandes und versucht mit den Mitteln der
kulturellen Erinnerung, die Gegenkultur eines \Cite{anderen Deutschlands}
aufzugreifen.

\Title{Pastor Hall} ist der eindringliche Versuch, die Bedrohung des freien
Individuums durch das NS-Regime aufzuzeigen. Das dramatische Grundthema ist
die \emph{Bewahrung der persönlichen Handlungsfreiheit} durch die Überwindung der Furcht
und die Kraft des Vorbilds. Dazu greift Toller auf das Motiv des Selbstopfers
zurück. Das Zusammenspiel mehrerer repräsentativer Figuren der politischen
\Quote{Mitte} verdeutlicht den politischen Widerstand als geforderte, aber
persönliche Entscheidung. 

Angesichts eines totalitären Systems, das die Subjektautonomie massiv
beschneidet, wird in \Title{Pastor Hall} konsequent für die persönlichen
Grundfreiheiten Stellung bezogen und damit eine essentielle Voraussetzung
jeder modernen Gesellschaft eingefordert. Trotz Rückgriff auf märtyrerhafte
Elemente kann dem Stück also eine moderne \Quote{Stoßrichtung} nicht abgesprochen
werden. 

\HeadingTwo{Zwischen Weltverbesserung und Isolation}

Die Dramatik Ernst Tollers hat sich in den vorgestellten Ausschnitten als eine
durchgehend politische, mit den gesellschaftlichen Entwicklungen ihrer Zeit
kritisch befasste künstlerische Auseinandersetzung erwiesen. Die durch die
ausgewählten Dramen abgedeckte Zeitspanne umfasst die gesamte Schaffensperiode
des Autors von den \Quote{Geburtswehen} der ersten deutschen Republik bis zum
düsteren Niedergang ihrer Errungenschaften unter der Naziherrschaft. Ernst
Toller hatte in dieser Zeitspanne die Bedeutung einer historischen Figur. Als
Politiker, Schriftsteller und Redner begab er sich immer wieder in die
Brennpunkte der Zeitgeschichte und versuchte dabei selbst eine gestaltende
Rolle zu spielen. Die für sein Leben so charakteristische Stellung zwischen
allen \Quote{Schubladen} und Etiketten äußerte sich nicht zuletzt in einem
tiefsitzenden Gefühl der Isolation und Heimatlosigkeit. In Verbindung mit
seinem nie zu befriedigenden Wunsch nach Verbesserung der politischen und
sozialen Verhältnisse  ergab sich daraus Ernst Tollers
bemerkenswerte Spannungslage zwischen Weltverbesserung und Isolation, die sich
in seiner Dramatik eindrucksvoll widerspiegelt.

Die exponierten Figuren der Tollerschen Dramatik stehen zumeist in einem
Konflikt zwischen Ideal und Gemeinschaft. Die gesonderte Stellung des
idealistischen, sensiblen Menschen ist jeweils als anspruchsvolles
Zugehörigkeitsproblem  gestaltet, in dem die Unzufriedenheit mit der
gegenwärtigen Realität von Staat und Gesellschaft zu kritischen
\Quote{Außenperspektiven}
führt. Der Wunsch, sich nur dort \Quote{einzufügen}, wo die eigenen
moralischen Maßstäbe Geltung finden, bedingt die für Tollers \Quote{Helden}
spezifische Spannung zwischen Aussonderung und Idealismus.

In der \Title{Wandlung} wird der Protagonist zunächst als ein isolierter,
\Quote{fehlgeleiteter} Sucher nach sinnvoller Gemeinschaft eingeführt. Sein Weg
zum \Quote{Seelenheil} vollzieht sich als Umsetzung eines generalisierten
Gemeinschaftsideals. In abstrakter Allumfassung \Quote{verschmilzt} der
Suchende mit \Quote{der Menschheit}. Die Verbrüderungsrevolution verbindet
das Motiv der Weltverbesserung mit der Aufhebung der Isolation. 

Die idealisierte Entsprechung von Vordenker und Volk in einem Erweckungsschema  
verdeckt jedoch die realen Probleme einer intellektuell geleiteten Revolution
und führt zu Widersprüchen, die der Führerperson als moralische Probleme
in die Quere kommen. In \Title{Masse Mensch} wird der Kontrollverlust über die
revolutionäre Bewegung als Schuldproblem erörtert und die Isolation der
Protagonistin bleibt mangels geeigneter Gemeinschaftsbildung bestehen. Die
Perspektive der Weltverbesserung muss hier in einen jenseitigen Wartestand
verlagert und die Menschheit augrund ihres Massencharakters für vorerst unreif
befunden werden. Weltverbesserungswunsch und Isolation fließen im
\Quote{Heldentod} zusammen.

Die Alternativen von humansozialistischer Verschmelzung oder
mystisch-jenseitiger Entrückung, die die ersten beiden Dramen dem isolierten
Weltverbesserer eröffnen, werden in \Title{Hoppla, wir leben!} einer
gründlichen Umbewertung unterzogen. Am Beispiel des Protagonisten wird hier
die Rolle des \Quote{Erweckers} diskreditiert und dessen Isolation gerade aus
seinem Festhalten an den überkommenen Idealen der Gefühlsrevolution
begründet. Weil politisches Handeln in diesem Drama nicht mehr als
Umwälzung durch vereinzelte \Quote{Vorbilder} funktioniert, führt die soziale
Vereinzelung der unglücklichen Führerfigur in das Extrem der Selbstötung. 
Als Integrationsperspektive wird die konstruktive Eigentätigkeit des Einzelnen
angeboten, der sich im Rahmen der Chancen einbringt, die das politische System
bietet.

Die in \Title{Hoppla, wir leben!} vorausgesetzte Möglichkeit der konstruktiven
persönlichen Einflussnahme auf das politische System, ist in der Situation
von \Title{Pastor Hall} nicht mehr gegeben. Als Staatssystem begegnet hier das
totalitäre NS-Regime, gegenüber dem die politische Aussonderung als moralisches
Gütesiegel gelten muss. Entsprechend fallen gesellschaftliche Isolation und
Weltverbesserung in diesem letzten Szenario wieder zusammen. Die Figur des
Außenseiters ist zum positiven Widerstandshelden aufgewertet, der für eine
bessere Welt im Kleinen, eine Gegenkultur steht. Der politische Kampf muss
sich hier auf die ethische Freiheit des Einzelnen konzentrieren, um 
die Grundlagen für humane Gemeinschaft überhaupt erst wieder herzustellen.

In der Entwicklung vom messianischen Expressionismus über neusachlichen
Pragmatismus bis hin zur antifaschistischen Exildramatik lässt sich also
ein  Bewusstseinwandel bei Ernst Toller erkennen, in dem sich erzwungene
politische Bescheidenheit mit geistiger Modernisierung verbindet. Der
idealistische Weltverbesserer der räterevolutionären Phase hatte die
reale Komplexität politischer Meinungs- und Willensbildung hautnah erfahren
müssen und setzte seine Hoffnungen später verstärkt auf die Demokratie. Dabei
differenzierte er seine Vorstellung vom Subjektstatus der politisch Aktiven,
relativierte den Geltungsanspruch der humansozialistischen Weltsicht und
bemühte sich um normative Konsensbildung in verschiedensten gesellschaftlichen
Gruppen. Das Erstarken des Nationalsozialismus wurde dann zur Belastungsprobe
seiner Hoffnungen auf die moderne, freiheitliche Staatenwelt und ihre
bürgerlichen Eliten. Seinen verzweifeltem moralischen Appell gestaltete er in 
\Title{Pastor Hall} und griff in dessen Figurengestaltung bewusst auf
messianisch-prophetische Mittel zurück, da er die Hoffnung auf subtilere,
pluralistischere Formen der politischen Einflussnahme angesichts der historischen
Entwicklung offensichtlich verloren hatte.

Für Toller selbst standen Weltverbesserungswünsche und Isolation 
zeitlebens in einer engen Wechselbeziehung. Die einzige längere feste
Bindung, die er mit einer Frau einging, scheiterte an seinem geradezu
zwanghaften Streben, gegen das Elend der Welt anzukämpfen. Im New Yorker Exil
fasste er 1938 den Entschluss, sich im Spanischen Bürgerkrieg zu engagieren. Die
Beziehung zu Christiane Grautoff, seiner Frau, scheint im Vergleich zu solchen
weltpolitischen \Quote{Pflichten} eher zweitrangig gewesen zu sein:

\label{grautoff}
\begin{BlockQuote}
\DramName{Ernst Toller}: Sie brauchen mich \ldots ich muß \ldots ich muß! [..] 
Du kannst das nicht verstehen - man würde es mir nicht verzeihen, wenn ich
nicht mitkämpfen würde!\\
\DramName{Christinae Grautoff}: Ich mache nicht mehr mit. Wir rennen seit
Jahren. Du musst an überhaupt keine Front. [..] Vor was läufst du überhaupt
weg? [..] Aber rennen, rennen, rennen ist ja der reinste Wahnsinn. Sind wir
nicht Flüchtlinge genug? Ist es nicht arg genug, überall ein Fremder zu sein?
Weißt du - du hast mich überhaupt erst zu einem Menschen gemacht. Mit dir,
für dich, durch dich bin ich zu einem Menschen, anständigen Menschen
geworden, aber nun will ich auch Frau sein und nicht nur hin und her jagen
für Sachen, von denen ich weiß, sie sind schlecht für
dich!\Footnote{\BibRef{FULD.96}{123}. Laut Christiane Grautoffs
  Lebenserinnerungen war dies der entscheidende letzte Dialog der
  Ehepartner. Ernst Toller wählte seinen Weg -- den Weg der Weltverbesserung
  und Isolation.}
\end{BlockQuote}


%%% Local Variables: 
%%% mode: latex
%%% TeX-master: "~/TOLLER/MAIN"
%%% End: 
