\HeadingOne{Begriffliche und theoretische Grundlagen}{Grundlagen}

In der Einleitung wurden Schlagwörter wie \Cite{Modernisierungsprozess},
\Cite{Modernereflexion} und \Cite{Literarische Moderne} verwendet, die einer
theoretischen Fundierung und begrifflichen Klärung bedürfen. Die entsprechende
Grundlegung soll in diesem Kapitel erfolgen.

\HeadingTwo{Modernisierungsprozess und Moderne}

Im Rahmen einer literaturwissenschaftlichen Arbeit ist der Begriff
\Cite{Moderne} zunächst mehrdeutig. Einerseits werden völlig verschiedene
Phasen der Literaturgeschichte als \Cite{Moderne} apostrophiert, andererseits 
mischt sich dabei die Übernahme von Selbstkennzeichnungen \Cite{moderner}
Schriftsteller mit mehr oder weniger begrifflich-analytischen Verwendungen
dieses Etiketts durch die Literaturwissenschaft.\Footnote{\Abr{Vgl}
  \AbrPair{z}{B} \BibRef{kemper.98}{97}\Abr{ff}, \BibRef{koopma.97}{1-11}, und
  \BibRef{grimin.95}{12}\Abr{ff}}

So begegnet man der älteren Begriffsverwendung in der Opposition \Cite{modern}
versus \Cite{antik} spätestens seitdem die Gültigkeit zeitenthobener
Schönheitsideale im Sinne einer unnachahmlichen Antike zum Gegenstand
ästhetiktheoretischer Problematisierung geworden war.\Footnote{Die im
  Frankreich des späten \Nth{17} Jahrhunderts ausgelöste \Cite{Querelle des
    Anciens et des Modernes} gehört ebenso zu diesem Diskursfeld wie der später
  \AbrPair{z}{B} von Schiller und Schlegel artikulierte Verlust antiker
  Idealität. \Abr{Vgl} dazu \BibRef{plumpe.95}{12}, \BibRef{koopma.97}{1}\Abr{f} oder
  \BibRef{jauss.89}{6}.}  Die implizite Aufwertung des Neuen über das Bestehende
zeigte sich seither immer wieder und mit zunehmender Intensität im
Selbstverständnis verschiedener literarischer Gruppen und \Cite{Avantgarden}:
So sahen Autoren des \Quote{Jungen Deutschlands} \Cite{das Moderne} als Moment ihrer
künstlerischen Bestrebens,\Footnote{\Abr{Vgl} \BibRef{koopma.97}{3}:
  \Cite{'Moderne' wurde um 1835 zu einem der neuen Schlagwörter, die die
    Bewegung des Jungen Deutschland kennzeichneten [..].}}  Naturalisten
publizierten als \Cite{Moderne Dichtercharaktere},\Footnote{Arendt, Wilhelm
  (Hrsg): Moderne Dichter-Charaktere. Berlin 1885.} und für die Wende zum
\Nth{20} Jahrhundert wird der Terminus \Cite{Moderne} sogar als inflationäre
Selbst- und Sammelbezeichnung verwendet, um die nicht mehr chronologisch zu
ordnende Vielfalt literarischer \Cite{Epochen} und Tendenzen dieser Zeit unter
eine gemeinsame Überschrift zu bringen.

Auch in der Postmoderne-Diskussion der letzten Jahrzehnte zeigte sich
eindrucksvoll, wie unspezifisch und vage der Begriff der \Cite{Moderne} auch
in der Debatte um das vermeintliche Ende der durch ihn bezeichneten Periode
geblieben zu sein scheint. Diese Schwierigkeiten wurden nicht ohne Grund als
Herausforderung zu einer \Cite{Präzisierung des
  Begriffs}\Footnote{\BibRef{ANZ.94}{101}.} wahrgenommen.

Die angedeuteten Varianten der Begriffsverwendung haben zwar gemeinsam, dass
sie Tendenzen oder Phasen im Laufe der Literaturgeschichte bezeichnen, bei
denen das Verhältnis von Innovation und Tradition, Neuem und Altem,
Bestehendem und Veränderung eine besondere Relevanz erhielt. 
Will man aber \Cite{die Moderne} als übergreifendes Periodisierungskonzept
definieren und damit die Vielfalt der \Cite{historisierenden}
Begriffsverwendungen sozusagen unter dem systematischen Dach einer
\Cite{gegenwartsoffenen Langzeitepoche}
ordnen,\Footnote{\BibRef{kemper.98}{101}.}
so reicht es nicht aus, dem Auftreten entsprechender Benennungen in
poetologischen Dokumenten oder sonstigen Formen literarischer
Selbstbezeichnung nachzuspüren.

Vielmehr ist die Frage zu stellen, seit wann und warum die Bestrebung,
\Cite{Tradition nicht zu wiederholen, sondern permanent zu
  brechen},\Footnote{\BibRef{plumpe.95}{32}.} zu einem bestimmenden Merkmal von
Literatur wurde. Darüber hinaus bliebe zu klären, welche Fragen man an den
derart systematisierten Gegenstandsbereich \Cite{moderner Literatur}
eigentlich stellen will, um den Begriffsaufwand in einem entsprechend
erkenntnisträchtigen Forschungsparadigma nutzbar zu machen.

Es hat sich als hilfreich erwiesen, zunächst zwischen \Cite{literarischer
  Moderne} und \Cite{gesellschaftlicher \Abr{bzw} historischer Moderne} zu
unterscheiden. Letztere ist mit Rückgriff auf Ergebnisse aus
Gesellschaftgeschichte, soziologischer Systemtheorie und (politischer)
Philosophie relativ präzise zu fassen.\Footnote{Als Bezugsquellen dienen
  hier wesentlich die umfassenden Arbeiten von Hans-Ulrich Wehler, Niklas
  Luhmann und Jürgen Habermas. Zur Begründung dieser Auswahl \Abr{vgl} auch
  \BibRef{lohmei.00.1}{1-5}.}

So kann ein Begriffsbündel zur Kennzeichnung des Prozesses der
\Cite{gesellschaftlichen Modernisierung} zusammengetragen werden, wobei die
verschiedenen Betrachtungsaspekte als Perspektiven auf einen komplexen
Gesamtvorgang zu verstehen sind. Anhand der genaueren Charakterisierung dieses
Prozesses kann dann die \Cite{gesellschaftliche Moderne} als der Zeitraum
eingegrenzt werden, in dem diese Prozessmerkmale \Cite{irreversibel} werden
und in ihrer Beschleunigung und historischen Relevanz einen epochalen Wandel
konstituieren.\Footnote{\Abr{Vgl} \BibRef{wehler.87}{21} und
  \BibRef{luhman.80}{27}.}

Aus sozialgeschichtlicher Sicht wird gesellschaftliche Modernisierung im
Wesentlichen durch die Herausbildung moderner Zentralgewalten
(\emph{Staatenbildung}), das Entstehen maschinisierter, zentralisierter und
stark arbeitsteiliger Produktionsweisen (\emph{Industrialisierung}) sowie die
gewinnorientierte, auf marktwirtschaftlichen Prinzipien basierende Aneignung
(\emph{Kapitalisierung}) gekennzeichnet. Diese politischen und ökonomischen
Tendenzen sind begleitet von \emph{Bevölkerungswachstum} und der Entstehung
städtischer Zentren (\emph{Urbanisierung}).\Footnote{\Abr{Vgl}
  \BibRef{wehler.87}{23}, 59\Abr{ff} und 218\Abr{ff}}

Die genannten sozialgeschichtlichen Prozesse bewirken gesellschaftliche
Veränderungen, die aus systemtheoretischer Sicht als \Cite{Transformation der
  stratifikatorisch differenzierten Gesellschaft Alteuropas zur funktional
  differenzierten Gesellschaft der europäischen Moderne} beschrieben
wurden.\Footnote{Die hier sehr knapp gehaltene Andeutung des
  systemtheoretischen Modernisierungsbegriffes stützt sich hauptsächlich auf
  \BibRef{lohmei.00.1}{44}\Abr{f} wobei die theoretischen Grundlagen in verschiedenen
  Arbeiten Luhmanns zu finden sind.}  Diese \emph{funktionale
  Ausdifferenzierung} beschert modernen Gesellschaften eine Vielfalt relativ
autonomer sozialer \Cite{Teilsysteme} mit jeweils eigenen Norm- und
Wertsetzungen. Auch Kunst und Literatur geraten in den Rang eines
relativierten Teilbereichs, es erfolgt eine \Cite{\emph{Autonomisierung des
    Kunstsystems}}.  Außerdem gehören die Individuen nur mehr partiell, temporär
und funktionsbezogen solchen Teilsystemen an. Mit dem Verlust statischer
Lokalisierung in angestammten Standeshierarchien erfährt auch die
\Cite{Ganzheit} menschlicher Selbstwahrnehmung und die darauf basierende
\Cite{soziale Identität} eine zunehmende Brechung.\Footnote{\Abr{Vgl}
  \BibRef{plumpe.95}{44}, \BibRef{lohmei.00.1}{3}.}

Da die funktionale Differenzierung der Gesellschaft aus systemtheoretischer
Sicht keine \Cite{Funktionsredundanz} hervorbringt, gibt es in ihr auch
\Cite{keinen Ort mehr [..], am dem sie privilegiert und zustimmungsfähig
  beschrieben werden könnte}.\Footnote{\BibRef{plumpe.95}{46}.}

Diese Feststellung leitet unmittelbar über zu den Konsequenzen, die der
Modernisierungsprozess aus philosophischer Perspektive mit sich bringt:
Gesellschaftliche Modernisierung als \Cite{fundamentale Veränderung des
  Denkens und der Normbildung}\Footnote{\BibRef{lohmei.00.1}{3}.}  ist mit
Jürgen Habermas wesentlich durch \emph{Säkularisierung},
\emph{Pluralisierung}, \emph{Selbstreflexivität} und \emph{Subjektautonomie}
gekennzeichnet: Die frühere Einheit normativ gesetzter Wahrheit im
Weltdeutungssystem der Religion verliert durch die Säkularisierung an
Verbindlichkeit, was einen Verlust an religiöser Heilsgewissheit und eine
Entwertung religiös fundierter Normen und Wertsetzungen mit sich
bringt. Dieser Vorgang der Relativierung trifft auch alle anderen Systeme
verabsolutierender Weltdeutung. Die Tendenz zum Wahrheitspluralismus setzt
solchen Absolutheitsansprüchen die Einsicht in die subjektive Perspektivierung
jeglicher Erkenntnis entgegen.\Footnote{Das ergibt sich aus der
  durch Kant eingeleiteten \Cite{transzendentalphilosophischen Wende}.} Für die
somit in die Autonomie entlassenen Subjekte moderner Gesellschaften ergibt
sich die prinzipielle Notwendigkeit, eigene Normen und Werte zu entwickeln und
für ihr soziales Miteinander Übereinkünfte auszuhandeln:

\begin{BlockQuote}
  Die Moderne kann und will ihre orientierenden Maßstäbe nicht mehr Vorbildern
  einer anderen Epoche entlehnen, \emph{sie muss ihre Normativität aus sich
    selbst schöpfen.}\Footnote{\BibRef{haberm.85}{16}.}
\end{BlockQuote}
Dieses \Cite{Aus-sich-selbst-Schöpfen} verschafft dem modernen Subjekt die
Notwendigkeit der Selbstreflexion. Die stets präsente Option der
Selbstinspektion schafft eine für modernes Bewusstsein typische
\Cite{Dissoziation}: Es entsteht nicht mehr als Einheit in vorgegebenen
Erfahrungs- und Deutungshorizonten, sondern tritt quasi neben sich,
reflektiert sich und seine Bestimmungen und kann \Abr{bzw} muss dabei Probleme
seiner \Cite{Identität} aufwerfen.

Die hiermit gegebene Begriffsbestimmung \Cite{gesellschaftlicher
  Modernisierung} und ihrer Auswirkungen auf \Cite{modernes Bewusstsein} wird
mittlerweile in vielen literaturwissenschaftlichen Publikationen in ähnlicher
Weise zusammengefasst und \Cite{verdichtet sich allmählich zu einem soliden
  Konsens}\Footnote{Siehe \BibRef{lohmei.00.1}{5}. Thomas Anz verwendet den
  Begriff der \Cite{zivilisatorischen Moderne} in nahezu deckungsgleicher
  Bedeutung: \BibRef{anz.94}{1}.}.

Neben den dargestellten Aspekten, die quasi zum begrifflichen Kernbestand
gesellschaftlicher Modernisierung zu rechnen sind, werden von verschiedenen
Autoren weitere Teilmomente genannt. So gehören \AbrPair{z}{B} für Hans
Esselborn zu den \Cite{zentralen außerliterarischen Herausforderungen} an die
Literatur der Moderne die politische Demokratisierung, der Fortschritt in den
Naturwissenschaften und die Herausbildung einer moderner
Medienkultur.\Footnote{\BibRef{esselb.94}{416-418}.} Auch Thomas Anz nennt
die \Cite{Expansion massenkommunkativer Prozesse} als Merkmal der
\Cite{zivilisatorischen} Moderne. Er betont des Weiteren die für das
\Cite{zivilisierte Subjekt} notwendig gewordene Affektkontrolle als wichtigen
sozialpsychologischen Aspekt.\Footnote{Vgl. die Bezugnahme auf Norbert Elias
  in \BibRef{anz.94}{1}.}

Basierend auf dem Bündel der angeführten Modernisierungsmerkmale lässt sich
der Epochenbegriff der \Cite{gesellschaftlichen Moderne} schlüssig aufbauen.
Interessanterweise ergibt sich aus den verschiedenen Aspekten des
Modernisierungsprozesses eine ganz ähnliche zeitliche Verortung für den Beginn
dieser historischen Epoche:

\begin{BlockQuote}
  So gibt es sowohl interdisziplinär als auch innerhalb der verschiedenen
  methodischen Ansätze der Literaturwissenschaft einen gewissen Konsens, die
  Makroepoche der Moderne in der von Reinhart Koselleck so genannten Sattelzeit
  um 1800, im weiteren Sinne zwischen 1750 und 1850 beginnen zu
  lassen.\Footnote{\BibRef{kemper.98}{101}. Siehe auch \BibRef{plumpe.95}{24-25}.}
\end{BlockQuote}
Begründung für diesen Konsens ist die Diagnose, dass die genannten
Teilprozesse in ungefähr diesem Zeitraum \Cite{irreversibel} werden.  Die
politische Zäsur durch die Französische Revolution, das Einsetzen der
industriellen Revolution, die Auswirkungen der rationalistischen Aufklärung
und die Kritik ihres Wahrheitsbegriffs in der Kantischen
Tranzendentalphilosophie sind greifbare Wendepunkte, die diesen
Periodisierungskonsens (zusätzlich) bestärken.

\HeadingTwo{Literarische Moderne}

Nachdem der Begriffskomplex der\Cite{gesellschaftlichen Modernisierung}
umrissen und die \Cite{historische Moderne} als eingrenzbare Periode
vorgestellt wurde, kann der Begriff der \Cite{literarischen Moderne} -- oder
allgemeiner: der \Cite{ästhetischen Moderne} -- daran angeschlossen werden.

Ausgehend von der Theorie der gesellschaftlichen Moderne kann das auffallende
Thematisieren und Problematisieren von Aspekten dieses gesellschaftlichen
Vorgangs in Literatur und Ästhetik sozusagen als gleichzeitiger, aber
ambivalenter Bewusstseinsprozess begriffen werden, der vor dem Hintergrund der
genannten Umwälzungen zunehmend unausweichlich zu werden scheint. Das
Bewusstsein, \Cite{in grundsätzlich veränderter Zeit} zu leben, wird seit dem
späten \Nth{18} Jahrhundert zum \Cite{beherrschenden Gesichtspunkt
  ästhetischer Reflexion und Produktion}.\Footnote{\BibRef{lohmei.00.1}{5}.} Es
liegt also nahe, \Cite{ästhetische Moderne} als \Cite{ambivalente Reaktion auf
  gesellschaftliche Modernisierungsprozesse} zu
begreifen.\Footnote{\BibRef{ANZ.94}{1}.}

Im Zuge der Anwendung dieses \Cite{literaturhistorischen
  Paradigmas}\Footnote{Ebd.} ist der Vorschlag entstanden, \Cite{Literarische
  Moderne} als \Cite{Makroepoche} zu verstehen:

\begin{BlockQuote}
  Der Makroepochenbegriff stiftet nicht nur die Zusammenhänge von
  Längsschnitten, sondern vermag auch, die synchronen inneren Zusammenhänge
  scheinbar gegensätzlicher Phänomene als unterschiedliche, in ihren
  Erscheinungsformen konträre Antworten auf ein und dasselbe Grundproblem der
  Moderne auszuweisen.\Footnote{\BibRef{VIETTA.98}{14}.}
\end{BlockQuote}
Die tradierten literaturgeschichtlichen Epochenbegriffe sollen dabei als
\Cite{Mikroepochen} beibehalten und in das Modell eingebettet
werden.\Footnote{Einen systemtheoretisch begründeten Entwurf zur völligen
  Neugliederung der \Cite{Epochen moderner Literatur} liefert dagegen
  \cite{plumpe.95}.}

Angriffspunkte für eine Kritik dieser Theorie der Literarischen Moderne liegen
sicherlich im latenten Teleologismus des prozessualen und damit
vermeintlich linear gerichteten Basisbegriffs gesellschaftlicher
Modernisierung.  Es muss außerdem betont werden, dass insbesondere die
zeitliche Festlegung der Epoche primär auf die Verhältnisse Westeuropas
zugeschnitten ist. Auch die Diskussion um die angebliche Ablösung der Moderne
durch die sogenannte \Cite{Postmoderne} ergibt gewisse Möglichkeiten der
Kritik, die aber hier nicht weiter erörtert werden können.

Generell ist die Neben- und Gegenläufigkeit von Teilsträngen des
gesellschaftlichen Modernisierungsprozesses ebenso zu beachten wie die
reflexive Bezogenheit und Verschobenheit von literarischer Moderne gegenüber
Aspekten der gesellschaftlichen Modernisierung. Die vorgestellte
Forschungsperspektive ist gerade nicht im Sinne einer irgendwie gearteten
\Cite{Widerspiegelungstheorie} zu verstehen.

\HeadingTwo{Modernetheoretische Literaturwissenschaft}

Ausgehend von dem allgemeinen Begriff literarischer Moderne als der Literatur
der letzten 200 Jahre, bleibt zu klären, welche Fragen man an diese moderne
Literatur stellen will und inwiefern man dabei zu einer Graduierung
literarischer \Cite{Modernität} gelangt.

Die ältere Forschung, die \Cite{ihren Modernebegriff mehr oder weniger aus den
  Texten selbst} bezog, sah überall dort \Cite{moderne Literatur}, wo Phänomene
des Modernisierungsprozesses literarisch aufgegriffen wurden. Der dabei
häufig anzutreffende Tenor von Verlusterfahrung und
Modernekritik galt den Interpreten gerade als das spezifisch Moderne
dieser Literatur. Vielfach wurde diese \Cite{antimoderne} Haltung sogar
schlicht reproduziert, weil sich die Forschung \Cite{im Wahrnehmungs- und
  Deutungssystem der Texte selbst} bewegte und nicht selten der Gefahr erlag,
\Cite{ihren Gegenstand lediglich zu verdoppeln}.\Footnote{Die hier zitierten
  Einschätzungen finden sich in \cite{lohmei.00.1}.}

Der Begriffsapparat, der in den vorigen Abschnitten vorgestellt wurde,
bietet nach Auffassung von Anke-Marie Lohmeier die Möglichkeit, \Cite{die
  Position identifizierender Textverdopplung aufzugeben und auf analytische
  Distanz zu den Texten zu gehen}. Denn in diesen Texten wird den
differenzierenden, pluralisierenden und sozusagen normativ befreienden
Tendenzen des Modernisierungsprozesses vielfach eine Sehnsucht nach neuer
Totalität und Verbindlichkeit entgegen gesetzt, die in ihrem ideologischen
Kern bedenkliche \Cite{Einheit- und
  Ganzheitswünsche}\Footnote{\BibRef{anz.94}{6}.} mit sich bringt. Auch Silvio
Vietta und Dirk Kemper sehen die Problematik solcher Denkfiguren:  

\begin{BlockQuote}
  Die Moderne wird den Prozess der Säkularisation selbst radikal vorantreiben,
  aber sie wird auch die durch den Metaphysikverlust enstandene Leere zu
  kompensieren versuchen und so einer höchst problematischen Geschichte der
  Utopien das Tor öffnen, den modernen Ersatzreligionen und
  Ideologien.\Footnote{\BibRef{VIETTA.98}{46}.}
\end{BlockQuote}
Aus dieser Perspektive erscheint es interessant, die Modernität von Literatur
nicht nur an innerliterarischen Fragen der Ästhetik-, Stil- und
Formengeschichte festzumachen und in diesen Bereichen nach gestalterischer
Innovation oder strukturellen Analogien zwischen modernem Werk und moderner
Welt zu suchen,\Footnote{Wie \AbrPair{z}{B} in der These, gesellschaftliche
  Fragmentarisierung als soziologischer Prozess spiegele sich wider in
  fragmentarischer Textgestalt, \Abr{vgl} \BibRef{anz.94}{4}, oder
  \BibRef{vietta.92}{235}.} sondern anhand geeigneter Kriterien die Modernität
des in den Werken zum Ausdruck kommenden Bewusstseins zu analysieren.

Dieser Zugang zur Sondierung von Modernität könnte auf einem \Cite{normativen
  Begriff von Modernität und modernem
  Bewußtsein}\Footnote{\BibRef{lohmei.00.1}{8}.} gründen, der in relativ
stringenter Weise aus den im letzten Abschnitt angeführten
Modernisierungsbegriffen der Systemtheorie und Philosophie entwickelt werden
kann. Er beruht im Wesentlichen darauf, die Wirkungen der Moderne auf den
Charakter von Wahrheitskonstruktion, sozialer Identitätsbildung, Norm- und
Wertsystemen sowie die Modifikationen des Subjektstatus als Gegebenheiten der
modernen Welt anzunehmen und die Selbstbestimmungsfähigkeit des Subjekts, die
Pluralität von Wahrheit und die Konsensbedürftigkeit gesellschaftlicher
Normativität als unhintergehbare Axiome moderner Gesellschaften zu verstehen.

Aus einem solchen Modernitätsbegriff ergibt sich eine Vielzahl von
Fragestellungen, die als Aspekte für die literaturwissenschaftliche Analyse
nutzbar gemacht werden können. Zur Anwendung auf das dramatische Werk Ernst
Tollers werden diese Aspekte zunächst konkretisiert und dann möglichst textnah
angewendet.

%%% Local Variables: 
%%% mode: latex
%%% TeX-master: "~/TOLLER/MAIN"
%%% End: 
