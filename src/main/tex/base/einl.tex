\HeadlineOne{Einleitung}{Einleitung}
\pagenumbering{arabic}

\HeadlineTwo{I. Themenstellung}

Diese Arbeit beschäftigt sich mit einer Auswahl dramatischer Werke Ernst
Tollers. Die Dramen werden aus einer \Quote{modernetheoretischen} Perspektive
daraufhin untersucht, wie der Autor sich mit den Problemen und Entwicklungen
seiner Zeit auseinander setzte. Dabei soll insbesondere herausgearbeitet
werden, wie sich das Spannungsverhältnis zwischen Tollers intensiv empfundenem
Wunsch nach Weltverbesserung und der oft von Isolation geprägten Lage des
intellektuellen Idealisten in seiner literarischen Produktion äußerte.

Ernst Tollers Schaffensperiode als Dramatiker erstreckte sich ziemlich genau
auf die Zeit zwischen den Weltkriegen. Der Schrecken des ersten Krieges prägte
seine politische Grundhaltung und machte ihn zum Pazifisten und Verfechter
eines \Quote{humanen} Sozialismus. Politische Berühmtheit erlangte er als eine der
führenden Gestalten der Münchener Räterepublik von 1918/1919. Die folgende
mehrjährige Haft, in der er einige seiner bedeutendsten Werke schrieb,
festigten seinen Ruf als Symbolgestalt der Revolution. Während der Weimarer
Republik avancierte Toller zu einem der erfolgreichsten Dramatiker deutscher
Sprache.\Footnote{\Abr{Vgl} den Beitrag von Thomas Anz über Ernst Toller in:
  Metzler Autoren-Lexikon. Deutschsprachige Dichter und Schriftsteller vom
  Mittelalter bis zur Gegenwart. Herausgegeben von Bernd Lutz. Stuttgart
  [u.a.] 1997.}
In seinem stets von politischem Engagement durchdrungenen literarischen
Werk orientierte er sich meist eng an zeitgeschichtlichen Themen. Als
klarsichtiger Kritiker des aufstrebenden Nationalsozialismus nahm er früh die
vorauseilenden Schatten eines erneuten \Cite{Erdgemetzels}\Footnote{Mit diesem
  Ausdruck hatte Toller den Ersten Weltkrieg im Nachhinein bezeichnet. Siehe
  \LongSourceRef{II}{8}.} 
wahr. Nach Jahren des Exils wählte er 1939 den Freitod.

Ernst Toller gilt als einer der \Cite{engagierten}\Footnote{\Abr{Vgl}
 \BibRef{distl.93}{14}.}  Autoren deutscher
Literatur, der die Doppelrolle des Schriftstellers und politischen Aktivisten
mit bemerkenswerter Intensität ausfüllte. Im Wechselspiel der beiden
Betätigungsfelder versuchte er, Idee und Praxis zu verknüpfen.
In allen Stücken, die in der vorliegenden Arbeit untersucht werden, geht es 
im Wesentlichen um die Frage nach der richtigen Handlungsweise des ethischen
Subjekts bei der Verbesserung der politisch-gesellschaftlichen Verhältnisse. 
Die frühen, expressionistischen Stücke thematisieren einerseits die Rolle des
vorbildlichen intellektuellen Führers bei der Revolutionierung der Menschheit
und zeigen andererseits die Grausamkeit und Inhumanität der systematischen
Gewalt im Krieg und in revolutionären Kampfhandlungen. 
In den späteren Stücken wird die Führerproblematik abgelöst durch die Frage
nach geeigneten individuellen Handlungsmöglichkeiten, die dem Einzelnen in der
scheinbar demokratischen Welt der Republik \Abr{bzw} in der totalitären
Welt der NS-Herrschaft (noch) zur Verfügung stehen. 

Durchgängiges Merkmal all dieser Dramen ist die Konfrontation relativ
isolierter Hauptfiguren mit einer problematisierten Gegenwart. In der
Motivierung, Gestaltung und Auflösung dieses Gegensatzes entwickeln die
Stücke ihre jeweiligen Eigenheiten. Dies zu analysieren und im jeweils
relevanten zeitgeschichtlichen Kontext als Versuch einer literarischen
Bewältigung von Prozessen der historischen Moderne zu verstehen, ist das
Hauptanliegen der vorliegenden Arbeit. Die ausgewählten Dramen Ernst Tollers
werden somit als Literatur der \Quote{Literarischen Moderne} in den Blick
genommen.\Footnote{Was dabei mit \Quote{Literarischer
    Moderne} und verwandten Begriffen genau gemeint sein soll, wird in dem
  folgenden Grundlagen-Kapitel expliziert.}

Aus der genannten Perspektive auf die in den Dramen erkennbaren Reflexionsmuster  
erschließt sich eine Verlaufslinie, die die Tollersche Modernereflexion in
ihrer Entwicklung beschreibt. Zugleich ergibt sich damit ein Beispiel, wie 
ein wichtiger Autor der Weimarer Republik seine \Cite{Bewältigung des
  Wirklichen}\Footnote{So der Titel der neueren Monographie von Kirsten
  Reimers, die Tollers Bild von der \Cite{Realität} und ihrer Gestaltbarkeit
  anhand umfassender Dramenanalysen untersucht. Siehe \cite{reimer.00}.}   
von expressionistischer Verkündigungsdramatik über das neusachliche Zeitstück
bis hin zum antifaschistischen Exildrama variierte. 

\HeadlineTwo{II. Methode und Aufbau}

Ernst Toller war lange Zeit ein sehr umstrittener Autor, der den Kommunisten
zu \Cite{bürgerlich} und den Konservativen zu \Cite{sozialistisch} war. Auch der
literaturwissenschaftlichen Forschung ist es oft schwer gefallen, einen
differenzierten Zugang zu Tollers Werk zu finden. Immer wieder wurde er
\AbrPair{z}{B} auf sein expressionistisches Frühwerk reduziert.  

Glücklicherweise erbrachten die vergangenen Jahrzehnte eine Reihe von
differenzierten Analysen, die sich um eine unvoreingenommene Sicht auf ihren
Gegenstand bemühen und dabei verschiedene methodische Ansätze zur Anwendung
bringen.\Footnote{Für eine gute Übersicht über den neueren Forschungsstand sei
  hier auf \BibRef{reimer.00}{13f.} verwiesen.}
Einen wesentlichen Fortschritt für die Forschungsbemühungen
brachte die Herausgabe der \Title{Gesammelten Werke} Ernst Tollers
mit sich, die 1978 durch Wolfgang Frühwald und John M. Spalek
besorgt wurde.\Footnote{Wolfgang Frühwald / John M. Spalek [Hrsg.]: Ernst
  Toller. Gesammelte Werke. Band I-V. München 1978.
  Zitate in der vorliegenden Arbeit, die sich auf die
  \Title{Gesammelten Werke} beziehen, sind durch die in Klammern stehende
  Angaben des Bandes und der entsprechenden Seitenzahl gekennzeichnet.}

Die vorliegende Arbeit versucht unter Anwendung ihres
\Quote{modernetheoretischen} Ansatzes auf normative Distanz zu den
untersuchten Texten zu gehen und damit eine reproduzierende Verdopplung der
Autornorm möglichst zu vermeiden. Die Texte werden auf die in ihnen
enthaltenen Reflexionsmuster, Wertsysteme und Handlungsperspektiven
untersucht. Dabei soll möglichst keine implizite Parteinahme für oder wider
eine bestimmte Auffassung von Humanität erfolgen. Der Autor als moralisches
Subjekt soll hinter die Denkstrukturen und Ideologeme zurücktreten, die in
seinen Stücken enthalten sind. Die Herausarbeitung dieser Strukturen ist
deswegen die zentrale Aufgabe der Arbeit.

Die Untersuchung beschränkt sich auf eine Auswahl
dramatischer Texte und berücksichtigt andere Werke und Publikationen Tollers
nur am Rande. Die aufgestellten Thesen und Analyseurteile erhalten dadurch
eine gewisse Vorläufigkeit, die im Rahmen einer solchen Arbeit aber nie zu
vermeiden ist. Die gattungsmäßige Einschränkung auf die Dramatik scheint
vertretbar, da diese den Schwerpunkt des literarischen Werks des Autors
darstellt. 

Ausgewählt wurden aus dem expressionistischen Frühwerk die
\Quote{messianischen} Dramen \Title{Die Wandlung} (1917) und \Title{Masse
  Mensch} (1919). Beide stehen in einem engen Zusammenhang und bilden den Kern
der \Quote{O Mensch}-Dramatik Tollers. Relativ ausführlich wird auch auf das 1927
entstandene
Stück \Title{Hoppla, wir leben!} eingegangen, da es einen deutlichen Schritt in
Tollers Abwendung von der expressionistischen Weltsicht dokumentiert. Das
antifaschistische Exildrama \Title{Pastor Hall} (1938) vervollständigt den
Entwicklungsbogen zumindest zeitlich: Es ist zugleich Tollers letztes
Bühnenstück.

Zur Methode der vorliegenden Arbeit gehört es, die argumentativen Aussagen
möglichst nur aus der Interpretation der dramatischen Texte abzuleiten. 
Die Dramenanalyse erfolgt textnah und zieht biographische
Informationen über den Autor nur in Ausnahmefällen hinzu. Der verengte
Blick einer biographisch argumentierenden Interpretation soll so
vermieden werden. 

Im folgenden Kapitel werden einige begriffliche Grundlagen entwickelt, um den
genannten \Quote{modernetheoretischen Ansatz} genauer zu definieren. Daraus
ergibt sich die übergreifende Perspektive der Arbeit. Für die konkrete
Dramenanalyse führt dies zu einer Reihe von Aspekten und Themenstellungen, die
in einem weiteren Kapitel umrissen werden. Die
eigentliche Analyse erfolgt dann kapitelweise für alle vier Dramen. Am Ende
dieser Kapitel habe ich mich jeweils bemüht, die Kernpunkte der
Argumentation in geeigneter Form zu bündeln. Das abschließende Kapitel dient
dann der Zusammenfassung der Ergebnisse und enthält die Schlussbemerkungen der Arbeit.

%%% Local Variables: 
%%% mode: latex
%%% TeX-master: "~/TOLLER/MAIN"
%%% End: 
