% Titelblatt und Inhaltsverzeichnis


\title{  
  {\bfseries
    {\Large\Cite{Zwischen Weltverbesserung und Isolation}\\}
    {\large Reflexionen der Moderne im dramatischen Werk\\ Ernst Tollers\\}}  
  \vspace{10mm}     
  {\large Wissenschaftliche Arbeit im Rahmen der Ersten Staatsprüfung für das
    Lehramt an Gymnasien und Gesamtschulen\\}
  \vspace{15mm}
  {\normalsize
    Universität des Saarlandes\\
    Fachrichtung 4.1 Germanistik\\
    Prof. Dr. Anke-Marie Lohmeier\\}
  \vspace{10mm}}

\author{ 
  \small vorgelegt von:\\
  \normalsize Oliver Doepner\\ Schumannstr. 11\\ 66111 Saarbrücken\\
  \vspace{10mm}}

\date{Saarbrücken, im Juli 2001} 

\maketitle

\ExtraStuff{Danksagung}{
  Die vorliegende Arbeit wurde am 27. Juli 2001 als
  Examensarbeit im Rahmen der Ersten Staatsprüfung für das Lehramt an
  Gymnasien und Gesamtschulen im Saarland eingereicht.

  Frau Professorin Dr. Anke-Marie Lohmeier gilt mein herzlichster Dank für die
  Betreuung der Arbeit. Ihre methodische und inhaltliche Kritik regten mich
  immer wieder an, meine Argumentation zu präzisieren. 
  Den Teilnehmern des Examenskolloquiums danke ich herzlich für die angeregten
  Diskussionen.   
  Den Mitarbeitern der Ernst-Toller-Gesellschaft unter Leitung von Herrn
  Dr. Dieter Distl gilt mein besonderer Dank für die vorzügliche Unterstützung
  bei der Literaturbeschaffung.

  Diese Arbeit wurde mit dem \LaTeX-Textsatzsystem und dem Texteditor GNU Emacs
  auf dem Betriebssystem GNU/Linux erstellt. Allen Entwicklern der 
  \Cite{Free Software}-Bewegung, die zur Entstehung dieser Hilfsmittel
  beigetragen haben, danke ich ebenfalls recht herzlich.

  \bigskip\noindent
  Saarbrücken, im Juli 2001 \hfill Oliver Doepner
  }

\pagenumbering{roman}

\tableofcontents


