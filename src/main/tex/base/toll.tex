% Ernst Toller: Leben und Dramatik

\HeadingOne{Leben und Werk Ernst Tollers}{Leben und Werk}

Im Folgenden wird kurz auf die verfügbare biographische Literatur zu Ernst
Toller eingegangen, um dann in geraffter chronologischer Folge relevante
Lebensdaten zusammen zu stellen.

\HeadingTwo{Toller-Biographien}

Toller hat 1933 mit \Cite{Eine Jugend In Deutschland} eine Autobiographie für
die Zeit von 1893 bis 1924 veröffentlicht. Diese Quelle ist an vielen Stellen
von Selbststilisierung geprägt und nur nach objektivierender Prüfung
wissenschaftkich verwertbar.\Footnote{Vgl. Dove und Rothe.}
Aus der Perspektive von 1933 liefert Toller einen Aufriss der politischen
Entwicklungen seiner Jugend bis zur Entlassung aus der Haft im Jahr 1924. Die
Darstellung ist geprägt durch den Rückblick aus gereifter Erfahrung. Die
strukturgebende Erzähllinie ist Tollers Weg vom verblendeten Patrioten zum
Pazifisten und durch Revolutionserfahrung und Haft realistisch gewordenen
Humansozialisten. Packende Momentbeschreibung verbindet sich mit der
Aufdeckung struktureller Wurzeln der NS-Herrschaft und der Spaltung der
Arbeiterbewegung. Tollers politische Perspektive lässt familiäre und private
Aspekte dabei fast völlig in den Hintergrund treten.

In einer 1983 erschienenen Arbeit will Wolfgang Rothe \Cite{in Selbstzeugnissen
und Bilddokumenten} über Ernst Toller informieren. In dichter, pointierter
Darstellung zeichnet Rothe ein heterogenes Bild Tollers. Er charakterisiert
ihn als künstlerischen \Cite{Amateur wenn nicht Dilettant} mit Hang zur
Selbstdarstellung, betont aber sein \Cite{rhetorisches Genie} und würdigt ihn
als ehrlichen Kämpfer für Gerechtigkeit und Gewaltverzicht, der sich von
naivem Utopismus zu scharfsichtigem Realismus entwickelt habe.
Rothe stellt einige Facetten der \Cite{Tollerlegende} in Frage, die er bei
anderen Literaturwissenschaftlern kolportiert findet und belegt dies mit
entsprechenden Quellen. Für die Argumentation der vorliegenden Arbeit sind
insbesondere Rothes Aussagen zu Tollers \Cite{Menschheitsutopien} und seinem
Intellektuellenverständnis interessant.

Richard Dove hat 1990 mit \Cite{He was a German} eine Toller-Biographie
vorgelegt, die 1993 auch in deutscher Übersetzung erschien und aufgrund ihrer
Detailliertheit und Bandbreite als Referenzwerk bezeichnet werden kann. Dove bindet
stärker als Rothe den zeitgeschichtlichen und literaturhistorischen Kontext
ein und behandelt auch die Phasen der Weimarer Republik und des Exils wesentlich
ausführlicher.

Dieter Distls Arbeit \Cite{Ernst Toller - Eine politische Bibliographie} von
1993 konzentriert sich darauf, die Entwicklung Tollers politischer Ideen und
seines politischen \Cite{Engagements} nachzuzeichnen. Aufgrund dieser
Zielsetzung dominiert hier die Strukturierung und analytische Verknüpfung
bereits bekannter Lebensdaten Tollers.  

Als Ergänzung zu den genannten Schriften eignen sich die Lebens-Erinnerungen
Christiane Grautoffs, Tollers einziger längerer Partnerin und Ehefrau. Sie
wurden 1993 erstmals unter dem Titel \Cite{Die Göttin und ihr Sozialist}
herausgegeben. Die Beschreibung der Jahre 1932 bis 1939 gibt aufschlussreiche
Einblicke in die oft vernachlässigte Exilzeit Tollers.

\HeadingTwo{Leben und Werk im Überblick}

Der folgende Überblick soll ein Gerüst abgeben, das Tollers Leben und Werk in
den historischen Kontext einbettet und die Chronolgie des dramatischen Werks
im biographischem Zusammenhang darstellt.  Unterstützende Belege für die
Argumentation dieser Arbeit werden nach Möglichkeit hervorgehoben.

\HeadlineThree{Jugend im deutschen Kaiserreich}

Ernst Hugo Toller, geboren 1893, wuchs geographisch, sozial und familiär als
Außenseiter auf. Als Sohn jüdischer Kaufleute in Samotschin, einem Handelsort
in der preußischen Provinz Posen, gehörte er zu einer schwindenden Minderheit,
die zwischen christlich-deutscher Mehrheit und polnischer Landbevölkerung
relativ isoliert war. Trotz des relativen Wohlstands der Familie ist Tollers
Kindheit nicht glücklich zu nennen\Footnote{Vgl. Dove, S. 26}. 
Gefühle der \Cite{Fremdheit} und \Cite{Unzugehörigkeit} verbanden sich früh
mit einem Hang zum
träumerischen Alleinsein. Die kommunal angesehene Stellung des Vaters konnte
nicht verhindern, dass rassisch-religiöse und soziale Ressentiments die
Kindheitserfahrung des feinfühligen Knaben mit prägten. Sein Vater starb, als
er siebzehn war, und soll ihm auf dem Sterbebett Mitschuld an seinem Tod
gegeben haben. Im Verhältnis zur sehr religiösen Mutter mischte sich Innigkeit
mit Ablehnung. Über die Geschwister Hertha und Heinrich ist wenig bekannt. Mit
der Schwester, die später Vorbild für die weise Ratgeberin im Drama \Cite{Die
Wandlung}\Footnote{Siehe Wandlung, S. XX.} wurde, verband Toller eine
lebenslange Bindung.

Die Enge des preußisch-autoritären Gymnasialunterrichts kontrastierte mit
Tollers Interesse für moderne Literatur. Er las Werke von Ibsen, Strindberg,
Wedekind und Hauptmann, die in der Schule verboten waren. Früh begann er
selbst zu schreiben, verfasste schon als Jugendlicher Zeitungsbeiträge und
versuchte sich literarisch.

Nach dem Abitur folgte ein halbherziges Studium in Grenoble, wo er sich dem
chauvinistisches Gehabe und den \Cite{abenteuerlichen} Ausschweifungen anderer
deutscher Studenten anschloss. Sein Resumée \Cite{Ich lebe in Frankreich und
habe Deutschland nie verlassen}\Footnote{Jugend in Deutschland, S.XY} 
legt auch nahe, warum er umgehend ausreiste,
als Anfang August 1914 die deutsche Mobilmachung und damit der Kriegsbeginn
bevorstand.

\HeadlineThree{Vom Patriotismus zum Sozialismus}

Im \Cite{Rausch des Gefühls}\Footnote{Siehe Jugend in Deutschland} 
nationaler Begeisterung wurde Toller deutscher
Kriegsfreiwilliger. Aus Ungeduld meldete er sich an die vorderste Front, wo er
den schikanösen Gegensatz von \Cite{Oben und Unten} in den eigenen Reihen
erlebte und seine Feindbilder ins Wanken gerieten.  Das Gemetzel gab dem
Hurra-Patrioten zu denken. Die Idee, dass die feindlichen Soldaten als
Menschen doch eigentlich \Cite{Brüder}\Footnote{Siehe Jugend in Deutschland}
seien, wurde später zum Kern seiner
pazifistischen Überzeugung. Nach einem Nervenzusammenbruch kam er 1916 ins
Lazarett, dann in ein Sanatorium und wurde gegen Jahresende beurlaubt.

Er ging nach München, um wieder zu studieren, verkehrte bald in literarischen
Kreisen und lernte unter anderem Rilke und Thomas Mann kennen. Die
Kriegseindrücke versuchte er zunächst zu verdrängen.

Bei einer Tagung im Herbst 1917 lernte Toller den Sozialwissenschaftler Max
Weber kennen. Beeindruckt folgte er Weber nach Heidelberg und setzte dort sein
Studium fort. In dieser Zeit hat er wohl erstmals Gustav Landauers
\Cite{Aufruf zum Sozialismus} gelesen und bezog daraus Grundbausteine seines
Politikverständnisses.

In Heidelberg setzte Toller sich mit anderen Studenten gegen den Krieg ein. Er
publizierte wirkungsvoll den Gründungsaufruf zu einem \Cite{Kulturpolitischen
Bund der Jugend in Deutschland}, dessen \Cite{Leitlinien} Anerkennung bei
einigen Intellektuellen und Hasstiraden bei den Nationalisten evozierten. Wie
in seinem bereits teilweise niedergeschriebenen Dramen-Erstling \Cite{Die
Wandlung} propagierte Toller dabei antinationalistische Ideen: \Cite{Liebe zu
den Menschen}, \Cite{Besinnung auf Menschhaftigkeit} und \Cite{Wahrhaftigkeit
des Geistes}, die durch vorbildhafte Intellektuelle im Volk zu verbreiten
seien.\Footnote{Vgl. GW X, S.Y.}

Da in Heidelberg angeblich Verhaftung drohte, flüchtete Toller Ende 1917 nach
Berlin, wo er den USPD-Arbeiterführer Kurt Eisner traf. Der Arbeitskampf
erschien ihm als praktischer Ansatz zu politischen Veränderungen und er folgte
Eisner nach München. Beim \Cite{Munitionsarbeiter-Streik} entdeckte er sein
Talent als mitreißender Redner und verteilte Auszüge aus der \Cite{Wandlung}
als Flugblätter. Dem Zusammenbruch des Streiks folgte Militärhaft, in der er
das Drama vollendete und den Gedichtzyklus \Cite{Lieder der
  Gefangenen}\Footnote{GW X, S. Y.}
begann. Im Mai 1918 wurde er entlassen und musste seinen Militärdienst
vollenden.  Nach psychiatrischen Untersuchungen, die er später in
\Cite{Hoppla, wir leben!} verarbeitete\Footnote{Vgl.: Professor Lüdin untersucht
  Karl Thomas}, zog Toller sich zu seiner mittlerweile
in Landsberg wohnenden Mutter zurück. Nach eigenem Bekunden war er zu dieser
Zeit durch die Lektüre marxistischer Literatur \Cite{überzeugter Sozialist}
geworden.\Footnote{Jugend in Deutschland: S. FF.}

\HeadlineThree{Revolution und \Cite{Schriftstellerrepublik}}

Im November 1918 kam es zu revolutionären Unruhen in Deutschland.  Eisner
wurde als \Cite{Ministerpräsident} eines \Cite{Volksstaates Bayern}
ausgerufen. Toller erfuhr davon und eilte nach München. Er übernahm Aufgaben
in den neuen Räteorganen und stieg in den USPD-Vorsitz auf.  Nach der
USPD-Niederlage bei den Landtagswahlen im Januar 1919 und Eisners Ermordung im
Februar konkurrierte eine parlamentarisch legitimierte SPD-Regierung mit den
Rätestrukturen.

Im April geriet Toller unter turbulenten Umständen und gegen den Willen der
KPD an die Spitze einer neu verkündeten Räterepublik. Der bewaffnete Kampf war
unabwendbar, die geflüchtete SPD-Regierung rüstete zum Gegenschlag. Als die
KPD wenige Tage später die Revolutionsführung an sich riss, musste Toller sich
unterordnen. Er wurde ein Kommandant der \Cite{roten Truppen} und erreichte
vorübergehenden Ruhm als \Cite{Sieger von Dachau}. Als die Übermacht zu groß
wurde, wollte KPD-Führer Leviné bis zuletzt kämpfen lassen. Toller trat von
seinen Funktionen zurück, was ihm die Kommunisten als \Cite{Verrat}
vorwarfen. Den Streit um die blutige \Cite{Opferbereitschaft} der Revolution
verarbeitete er später als Grundproblem im Drama \Cite{Masse
  Mensch}.

Anfang Mai wurde die Münchner Räterevolution endgültig brutal besiegt.  Viele
der Führer starben. Toller tauchte unter, wurde aber wenige Wochen später
festgenommen.

\HeadlineThree{Hochverrat und \Cite{Ehrenhaft}}

Der Prozess gegen Toller spiegelte die politische Atmosphäre nach der
gescheiterten Revolution wieder: Nationalistische Militärs hatten die Fäden in
der Hand, Freikorps begingen Exzesse an den Gefangenen. Gustav Landauer wurde
zu Tode geprügelt, Eugen Leviné hingerichtet.

Es entstand eine öffentliche Kampagne für die Schonung Tollers.  Die Anklage
versuchte, Toller persönlich zu diskreditieren, ohne aber letztendlich den Befund
der \Cite{Ehrenhaftigkeit} vermeiden zu können. Die abschließenden Worte des
Angeklagten betonten sein revolutionäres Ethos.\Footnote{Siehe GW X, S. Y.} 
Er wurde zu fünf Jahren Festungshaft verurteilt.

Zu Beginn der Haftzeit kam Tollers erstes Drama \Cite{Die Wandlung} sehr
erfolgreich in Berlin zur Aufführung und machte ihn zum \Cite{berühmtesten
politischen Gefangenen Deutschlands}\Footnote{Dove, S. XX}. 
Diese Prominenz verhalf ihm 1920 sogar zu einem Gnadenangebot, das er aber ablehnte.

Den Großteil der Haft verbrachte Toller wie viele Räterevolutionäre in der
abgelegenen Festung Niederschönenfeld. Es wurden seine literarisch
produktivsten Jahre. Er nannte sie später \Cite{eine Schule des Lebens}. Sie
machten ihn zum \Cite{scharfäugigen Realisten}, der nicht mehr an
\Cite{Wandlung zu neuem Menschentum} glaubte.\Footnote{Rothe, S.XX}. Der Zweifel an
expressionistischen Politideologien manifestierte sich bereits im Drama
\Cite{Masse Mensch}, das im Oktober 1919 entstand.  Im folgenden Jahr schrieb er
das historisierende Weberdrama \Cite{Die Maschinenstürmer}.

1921 erhielt Toller ein USPD-Landtagsmandat als Nachrücker, doch ihm wurde
kein Hafturlaub gewährt. Nach der Auflösung der USPD im folgenden Jahr vermied
er jegliche parteiliche Bindung. Er hatte die fortgeschrittene Fraktionierung
der Arbeiterbewegung erkannt und sah den Proletarier zunehmend \Cite{in seiner
banalen Realität}.\Footnote{GW X, S.Y.} 
Bedauernd konstatierte er die Manipulierbarkeit der
\Cite{Massen}. Seine besondere Abneigung galt dem Typus des standpunktlosen
\Cite{Realpolitikers}.\Footnote{Vgl. dazu die Darstellung des Wilhelm Kilman
  in dem Drama \EmphQuote{Hoppla, wir leben!}}

Erfolgreiche Drameninszenierungen stärkten Tollers Reputation in den zwanziger
Jahren. Außerdem schrieb er \Cite{Massenspiele} für die jährlichen Leipziger
Gewerkschaftsfeste. Aufführungen seines umstrittensten Bühnenstücks \Cite{Der
deutsche Hinkemann} mussten 1923/24 aufgrund nationalistischer Tumulte
eingestellt werden. Das Erscheinen des lyrischen \Cite{Schwalbenbuchs}, seinem
später meistübersetzten Werk, wurde von der Justiz erschwert.

Die Haft schwächte Toller mental bis zur Schreibunfähigkeit. Eine
\Cite{verschärfte Hausordung} brachte schikanöse Haftbedingungen und häufige
Disziplinarstrafen mit sich. Er überbrückte die geistige Ödnis mit Lesen,
vertiefte seine Bildung und pflegte Briefkontakte. Das Verhältnis zu Mutter
und Schwester intensivierte sich.

Am 15. Juli 1924 wurde er schließlich an der Grenze zu Thüringen freigelassen
und aus Bayern ausgewiesen.

\HeadlineThree{Freiheit in der Republik}

Tollers neue Freiheit fand große öffentliche Aufmerksamkeit. Die Sympathien
aus Arbeiterschaft und linksliberalen Kreisen waren ihm sicher. Er genoss den
Wirbel um seine Person, empfand aber auch hohen Erwartungsdruck. Berlin wurde
sein Wohnsitz, doch er war fast ständig auf Vortragsreisen. Erstmals konnte er
seine eigenen Stücke auf der Bühne sehen. Beim Leipziger Gewerkschaftsfest
trat er als umjubelter Festredner auf und man gab sein neuestes Massenspiel.

Der \Cite{Triumphzug} der ersten Monate setzte sich 1925 \Cite{nahtlos} fort
\Footnote{So Dove, Seite XY}.  
Aber auch nationalistische Störaktionen und hetzerische Kritik von
den Kommunisten nahmen zu. In der \Cite{relativen Stabilität} der Weimarer
Republik verfolgte Toller praktische politische Ziele wie Haftgerechtigkeit
und Zensurabschaffung. Die Lebensrealität entfernte ihn von der
Arbeiterklasse, obwohl er weiterhin für Klassenkampf und humanen Sozialismus
eintrat. Wichtig war ihm die Abgrenzung seiner literarischen Produktion von
Propaganda und Parteilinie.

Wiederkehrende Schreibkrisen und Selbstzweifel waren mit ein Grund für
verstärkte Koproduktion mit anderen Autoren. Die Zwiespältigkeit seiner
Prominenz zeigte sich im Wechsel von Publizitätsdrang und
Zurückgezogenheit. Toller entwickelte zeitweise ein gewisses Interesse für
Sport und moderne Technik. Auch der Einsatz neuer technischer Medien im
Theater reizte ihn.

Neben der Dokumentarschrift \Cite{Justiz}, die Beweise für Klassenjustiz
liefern sollte, und regelmäßigen Zeitschriftsbeiträgen veröffentlichte Toller
von 1924 bis 1932 allein fünf Dramen. Nachdem \Cite{Der entfesselte Wotan}
1926 uraufgeführt worden war, arbeitete er 1927 mit Erwin Piscator an
\Cite{Hoppla, wir leben!}. Trotz Streit über die Konzeption des Stücks, wurden
die Hamburger Aufführungen ein großer Erfolg, der sich international
fortsetzte. Das 1929 mit Walter Hasenclever verfasste Stück \Cite{Bourgeois
bleibt Bourgeois} blieb dagegen ein \Cite{Flop}. Mit \Cite{Feuer aus den
Kesseln} lieferte Toller 1930 ein \Cite{herausragendes Beispiel politischen
Dokumentartheaters}\Footnote{Dove, S. 21}, doch der große Publikumserfolg blieb auch
hier aus. Das politische Klima wurde ungünstiger für zeitkritisches
Theater. Die nach 1930 geschriebenen Stücke \Cite{Wunder in Amerika} (mit
Hermann Kesten) und \Cite{Die blinde Göttin} hatten, obwohl thematisch relativ
unpolitisch, an deutschen Theatern kaum noch Chancen.

Toller schaffte es, literarische Projekte mit häufigen Reisen und vielfältigem
politischen Engagement zu verbinden. Er bereiste unter anderem Palästina,
Nordafrika und viele europäische Länder. In den USA und Sowjetrussland
gesammelte Eindrücke verarbeitete er in kritischen Notizen, die 1930 in dem
Sammelband \Cite{Quer durch} erschienen.

Toller besuchte Kongresse des PEN-Clubs, engagierte sich bei den
\Cite{Revolutionären Pazifisten}, der \Cite{Deutschen Liga für
Menschenrechte}, dem \Cite{Kongress gegen koloniale Unterdrückung}, dem
\Cite{Internationalen Komitee der Freunde Sowjetrusslands} und der
\Cite{Weltliga für Sexualreform}. 1929 hielt er die Gedenkrede zum
10. Todestag Kurt Eisners.

Dem erstarkenden Nationalsozialismus stellte sich Toller sehr früh.  Bereits
in den späten zwanziger Jahren zeigte er sich als klarsichtiger Warner vor
der nationalsozialistischen Gefahr. Im Gegensatz zu SPD und KPD erkannte er
Hitlers rücksichtslosen Willen zur Macht und forderte als Gegenmittel eine
Einheitsfront der deutschen Arbeiterbewegung mit Wahlbündnissen und
Streikaktionen. Doch die Warnungen fanden kein Gehör.

\HeadlineThree{Antifaschistischer Kampf und Exil}

Nach dem Reichstagsbrand von 1933 weilte Toller in der Schweiz und entging
damit einem Verhaftungsversuch in Berlin. Seine gesamte Habe in Deutschland
wurde beschlagnahmt. Im Mai hielt er beim PEN-Kongress in Dubrovnik eine
Anklagerede gegen die \Cite{Gleichschaltung} der deutschen Kultur und bemühte
sich um Politisierung des Kongresses gegen Hitler-Deutschland.  Sein Name
stand auf einer der ersten Ausbürgerungslisten, mit denen das NS-Regime sich
kritische Intellektuelle vom Hals schaffen wollte. Der Kampf für ein
\Cite{anderes Deutschland} wurde fortan zum Fokus seiner politischen
Bemühungen. Seine Stellungnahme im Londoner Untersuchungsverfahren zum
Reichstagsbrand war ebenso wie 1933 beim Querido-Verlag erschiene
Autobiographie \Cite{Eine Jugend in Deutschland}, von diesem Anliegen geprägt.

Im Dezember 1933 starb seine Mutter. Tollers Beziehung mit Christiane Grautoff
hatte sich 1934 so weit entwickelte, das sie mit ihm nach London
übersiedelte. Toller wurde dort Mitbegründer des Exil-PEN und arbeitete mit
Hermann Kesten an einer Ausgabe seiner Gefängnisbriefe. Außerdem
schrieb er an der pessimistischen Komödie \Cite{No more Peace!}. Im Sommer
verband er eine mehrwöchige Russlandreise mit der Teilnahme am 1. Kongress des
sowjetischen Schriftstellerverbands in Leningrad.

1935 heirateten Christiane Grautoff und Ernst Toller in London. Mit dem
Erscheinen seiner gesammelten Dramen erreichte Tollers literarischer Ruf in
Großbritannien \Cite{wohl seinen Höhepunkt}\Footnote{Dove, S.250}. Er nutzte wie so
oft seine Reputation für politische Kampagnen, bemühte sich um die Freilassung
deutscher Intellektueller, die Bewahrung der \Cite{verbrannten} Literatur und
die Rechte von Exilanten. Trotz der zunehmenden Gefährdung durch Nazi-Agenten,
waren die Jahre in London wohl die glücklichsten des Exils. Ungefähr seit
1936 relativierte Toller seine pazifistische Grundhaltung, da er fest mit
einem Krieg in Europa rechnete. Er setzte sich für ein gemeinsame
Wehrhaftigkeit der demokratischen Staaten und der \Cite{Volksfront} aller
Exilanten gegen die Diktatur ein.

Bei der Uraufführung von \Cite{No more peace!} in London spielte seine Frau
die Hauptrolle. Nach einer
mehrmonatigen Vortragsreise durch die USA und Kanada, bei der er über
Nazideutschland aufzuklären versuchte, folgte 1937 die vollständige
Umsiedelung in die USA. Der Versuch, in der amerikanischen Theaterszene und
später als Drehbuchautor in Hollywood zu landen, blieb weitgehend
erfolglos. Wiederholt quälten ihn manisch-depressive Krisen. 1938 kehrte er
nach New York zurück und war bald in ständiger psychiatrischer Behandlung. Die
Ehe zerbrach, Christiane konnte seine Krisen und seine zwanghafte Unstetigkeit
nicht mehr ertragen. Als er erklärte, nach ins Bürgerkriegsland Spanien reisen
zu wollen, verliess sie ihn.\Footnote{Vgl. die Erörterung auf Seite
  \pageref{grautoff}.}
Die Arbeit am letzten Drama \Cite{Pastor Hall},
das als \Cite{Märtyrertragödie} bezeichnet werden kann, war sein letztes
literarisches Projekt. In Spanien entschloss er sich zu einer Hilfsaktion für
die hungernden Bürgerkriegsopfer und begann die demokratischen Regierungen in
Europa und USA zu kontaktieren.  Die Bemühungen scheiterten durch den Sieg
Francos. Tollers Kräfte zehrten sich auf, die Depressionen wurden
übermächtig. 
Verzweifelt angesichts des Zerbrechens seiner Ideale an der weltpolitischen 
Wirklichkeit, nahm er sich am 22. Mail 1939 in seinem New Yorker Hotelapartment das
Leben. Eine Trauerfeier in Manhattan mit 500 Gästen ehrte sein
Andenken. Weltweit erschienen Nachrufe.


