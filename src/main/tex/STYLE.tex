% Schriftgröße und Dokumenttyp
\documentclass[12pt,twoside]{report}

% Use Times font for more portable PDf output
\usepackage{times}

% Deutsch und Latin1 benutzen
\usepackage[utf8]{inputenc}
\usepackage[ngerman]{babel}
\usepackage{t1enc}
\fontencoding{T1}

% Fußnotenstil ohne Einrückung
\usepackage[bottom,marginal]{footmisc}

% Überschriftenstil klein
\usepackage[tiny]{titlesec}

% Seitenzahlen und Kapitelname im Seitenkopf
%\pagestyle{myheadings}

% Blocksatz erleichtern
\sloppy

% Seitenmaße
\setlength{\topmargin}{0cm}
\setlength{\headheight}{0cm}
\setlength{\headsep}{1cm}
\setlength{\textheight}{230mm}
% \setlength{\oddsidemargin}{0mm}

% Zählungstiefe
\setcounter{secnumdepth}{4}

% Zeilenabstand
\renewcommand{\baselinestretch}{1.2}


% Textmodi für Überschriften, Textkörper, usw.
\newcommand{\DenseStyle}{\renewcommand{\baselinestretch}{1}\normalsize\small}
\newcommand{\TextStyle}{\renewcommand{\baselinestretch}{1.2}\small\normalsize}
\newcommand{\BibStyle}{\DenseStyle}
\newcommand{\FootnoteStyle}{\renewcommand{\baselinestretch}{1}\normalsize\footnotesize}

% Hervorhebungen und Zitate
\newcommand{\Strong}[1]{\textbf{#1}}
\newcommand{\Emph}[1]{\emph{#1}}
\newcommand{\Cite}[1]{``#1''}
\newcommand{\EmphCite}[1]{\emph{``#1''}}
\newcommand{\Quote}[1]{`#1'}
\newcommand{\EmphQuote}[1]{\emph{`#1'}}
\newcommand{\DramName}[1]{\textsc{#1}}
\newcommand{\DramHint}[1]{\textit{#1}}
\newcommand{\Title}[1]{\EmphQuote{#1}}

% Abkürzungen
\newcommand{\Abr}[1]{#1.}
\newcommand{\Nth}[1]{#1.}
\newcommand{\AbrPair}[2]{#1.\,#2.}
\newcommand{\AbrCont}[2]{#1.~#2}
\newcommand{\Page}[1]{\AbrCont{S}{#1}}

% Hilfsbefehle für Kopfzeile und Inhaltsverzeichnis
\newcommand{\Header}[1]{\markright{\centerline{\textup{#1}}}}
%\newcommand{\Header}[1]{\markright{\rightline{\footnotesize{\textup{{#1}}}}\ }}
\newcommand{\BibHeader}{\Header{\bibname}}
\newcommand{\ChapterHeader}[1]{\Header{\chaptername\ \thechapter. #1}}
\newcommand{\TocEntry}[1]{\addcontentsline{toc}{chapter}{#1}}
\newcommand{\BibTocEntry}{\TocEntry{Literaturverzeichnis}}

% Nummerierte Kapitel
\newcommand{\HeadingOne}[2]{\chapter{#1}\ChapterHeader{#2}}
\newcommand{\HeadingTwo}[1]{\section{#1}}
\newcommand{\HeadingThree}[1]{\subsection{#1}}
\newcommand{\HeadingFour}[1]{\subsubsection{#1}}

% Sonstige Kapitel
\newcommand{\ExtraStuff}[2]{\chapter*{#1} #2}
\newcommand{\HeadlineOne}[2]{\chapter*{#1}\TocEntry{#1}\Header{#2}}
\newcommand{\HeadlineTwo}[1]{\section*{#1}}
\newcommand{\HeadlineThree}[1]{\subsection*{#1}}
\newcommand{\HeadlineFour}[1]{\subsubsection*{#1}}


% Zitatblöcke
\newenvironment{BlockQuote}{\DenseStyle\begin{quote}}{\end{quote}\TextStyle}

% Literaturverzeichnis
\newenvironment{BibList}[1]{\BibTocEntry\BibHeader\BibStyle\begin{thebibliography}{#1}}
  {\end{thebibliography}\TextStyle}

% Literaturverweise
\newcommand{\BibRef}[2]{\cite{#1}, \Page{#2}}
\newcommand{\SourceRef}[2]{\nolinebreak(#1/#2)}
\newcommand{\LongSourceRef}[2]{GW #1, \Page{#2}}

% Fußnoten
\newcommand{\Footnote}[1]{\footnote{\FootnoteStyle#1}}




